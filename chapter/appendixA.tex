\setcounter{chapter}{0}
\renewcommand{\thechapter}{\Alph{chapter}}

\titleformat{\chapter}[hang]{\hwyk\LARGE}
{}{0mm}{\hspace{-0.4cm}\myappendix}
 
\titleformat{\section}[hang]{\hwyk\LARGE}
{}{0mm}{\hspace{-0.5cm}\appendixsection}

\titleformat{\subsection}[hang]{\hwyk\large}
{}{0mm}{\hspace{-0.5cm}\appendixsubsection}

\titleformat{\subsubsection}[hang]{\hwyk\large}
{}{0mm}{\hspace{-0.5cm}\appendixsubsubsection}

\chapter{Gauss单位制与国际单位制}\label{Gauss单位制与国际单位制}

\section{有理化单位制和非理化单位制}

Gauss单位制与国际单位制之间,一个差别是在一些方程里的因子$4\pi$。国际单位制被分类为“有理化单位制”,因为Maxwell方程组里没有因子$4\pi$,而Coulomb定律\footnote{描述点电荷之间力的作用规律的定律。}和Biot-Savart定律\footnote{描述电流元之间力的作用规律的定律。}的方程里都含有因子$4\pi$。采用Gauss单位制的状况则完全相反,Maxwell方程中含有因子$4\pi$,但是Coulomb定律和Biot-Savart定律的方程里都没有因子$4\pi$。因此Gauss单位制被分类为“非理化单位制”。

对比Gauss单位制与国际单位制,电荷单位的定义有很大的区别。国际单位制特别为电现象设置一个基本单位——安培(Ampere),这造成电荷的量纲称为物理量的一种独特量纲,而不能用机械单位(\si{\kilo\gram}、\si{\m}、\si{\s})来表达。Coulomb定律方程为
\begin{equation*}
	\mbf{F} = \frac{1}{4\pi\epsilon_0} \frac{e_1e_2}{r^3}\mbf{r}
\end{equation*}
其中,$\mbf{F}$是Coulomb力,$\epsilon_0$是真空介电常数,$e_1$和$e_2$是两个相互作用的电荷,$r$是这两个电荷之间的距离,$\mbf{r}$为它们之间的相对位矢。

其中真空介电常数$\epsilon_0$的量纲为\si{\ampere^2\cdot \second^4\cdot \kilo\gram^{-1}\cdot\meter^{-3}}。而在Gauss单位制内,$\epsilon_0$并不存在,Coulomb定律方程为
\begin{equation*}
	\mbf{F} = \frac{e_1e_2}{r^3}\mbf{r}
\end{equation*}
因而在Gauss单位制中,电荷的单位\si{\statC}可以完全以机械单位写为
\begin{equation*}
	\SI{1}{\statC} = \SI{1}{\kilo\gram^{\frac12}\cdot\meter^{\frac32}\cdot\second^{-1}}
\end{equation*}

\section{Gauss单位制与国际单位制之间的方程对比}

\begin{table}[htbp]
  \centering
  \caption{Gauss单位制与国际单位制之间的方程对比}
	\label{Gauss单位制与国际单位制之间的方程对比}
    \begin{tabular}{|r|r|r|}
    \hline
         & \multicolumn{1}{c|}{Gauss单位制} & \multicolumn{1}{c|}{国际单位制} \bigstrut\\
    \hline
    \multirow{4}[2]{*}{Maxwell方程} & \multicolumn{1}{l|}{$\bnb \cdot \mbf{E} = 4\pi\rho$} & \multicolumn{1}{l|}{$\bnb \cdot \mbf{E} = \dfrac{\rho}{\epsilon_0}$} \bigstrut[t]\\
         & \multicolumn{1}{l|}{$\bnb \cdot \mbf{H} = 0$} & \multicolumn{1}{l|}{$\bnb \cdot \mbf{H} = 0$} \\
         & \multicolumn{1}{l|}{$\bnb \times \mbf{E} = -\dfrac1c \dfrac{\pl \mbf{H}}{\pl t}$} & \multicolumn{1}{l|}{$\bnb \times \mbf{E} = -\dfrac{\pl \mbf{H}}{\pl t}$} \\
         & \multicolumn{1}{l|}{$\bnb \times \mbf{H} = \dfrac{4\pi}{c}\mbf{j}+\dfrac1c \dfrac{\pl \mbf{E}}{\pl t}$} & \multicolumn{1}{l|}{$\bnb \times \mbf{H} = \mu_0\mbf{j}+\dfrac{1}{c^2} \dfrac{\pl \mbf{E}}{\pl t}$} \bigstrut[b]\\
    \hline
    Lorentz力 & \multicolumn{1}{l|}{$\mbf{F}=e\left(\mbf{E}+\dfrac1c\mbf{v}\times \mbf{H}\right)$} & \multicolumn{1}{l|}{$\mbf{F}=e\left(\mbf{E}+\mbf{v}\times \mbf{H}\right)$} \bigstrut\\
    \hline
    Coulomb力 & \multicolumn{1}{l|}{$\mbf{F} = \dfrac{e_1e_2}{r^2}$} & \multicolumn{1}{l|}{$\mbf{F} = \dfrac{1}{4\pi\epsilon_0} \dfrac{e_1e_2}{r^3}\mbf{r}$} \bigstrut\\
    \hline
    电场强度 & \multicolumn{1}{l|}{$\mbf{E} = -\bnb\phi-\dfrac1c\dfrac{\pl \mbf{A}}{\pl t}$} & \multicolumn{1}{l|}{$\mbf{E} = -\bnb\phi-\dfrac{\pl \mbf{A}}{\pl t}$} \bigstrut\\
    \hline
    磁场强度 & \multicolumn{1}{l|}{$\mbf{H} = \bnb\times \mbf{A}$} & \multicolumn{1}{l|}{$\mbf{H} = \bnb\times \mbf{A}$} \bigstrut\\
    \hline
    \end{tabular}%
\end{table}%

在Gauss单位制和国际单位制下,各种公式的对照如表\ref{Gauss单位制与国际单位制之间的方程对比}所示。

\section{转换单位的一般定则}

若要将任何方程从Gauss单位制转换至国际单位制,只要将其中以Gauss单位制表达的物理量直接替换为相应的国际单位表达式即可。反之亦然。为了简化其中涉及的关系,可能需要应用关系式$\epsilon_0\mu_0=\dfrac{1}{c^2}$。其中两种表达式可在表\ref{转换单位的一般定则}中找到。

\begin{table}[htbp]
  \centering
  \caption{转换单位的一般定则}
  \label{转换单位的一般定则}
    \begin{tabular}{|c|r|r|}
    \hline
    物理量  & \multicolumn{1}{c|}{Gauss单位制} & \multicolumn{1}{c|}{国际单位制} \bigstrut\\
    \hline
    电场、标势 & \multicolumn{1}{l|}{$\mbf{E}, \phi$} & \multicolumn{1}{l|}{$\sqrt{4\pi\epsilon_0}(\mbf{E}, \phi)$} \bigstrut\\
    \hline
    电荷、电荷密度 & \multicolumn{1}{l|}{$q, \rho$} & \multicolumn{1}{l|}{$\dfrac{1}{\sqrt{4\pi\epsilon_0}}(q, \rho)$} \bigstrut\\
    \hline
    磁场强度、矢势 & \multicolumn{1}{l|}{$\mbf{B}, \mbf{A}$} & \multicolumn{1}{l|}{$\sqrt{\dfrac{4\pi}{\mu_0}}(\mbf{B}, \mbf{A})$} \bigstrut\\
    \hline
    \end{tabular}%
\end{table}%
