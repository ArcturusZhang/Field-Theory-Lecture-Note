\setcounter{chapter}{0}
\renewcommand{\thechapter}{\Alph{chapter}}

\titleformat{\chapter}[hang]{\hwyk\LARGE}
{}{0mm}{\hspace{-0.4cm}\myappendix}
 
\titleformat{\section}[hang]{\hwyk\LARGE}
{}{0mm}{\hspace{-0.5cm}\appendixsection}

\titleformat{\subsection}[hang]{\hwyk\large}
{}{0mm}{\hspace{-0.5cm}\appendixsubsection}

\titleformat{\subsubsection}[hang]{\hwyk\large}
{}{0mm}{\hspace{-0.5cm}\appendixsubsubsection}

\chapter{Gauss单位制与国际单位制}

\section{有理化单位制和非理化单位制}

Gauss单位制与国际单位制之间,一个差别是在一些方程里的因子$4\pi$。国际单位制被分类为“有理化单位制”,因为Maxwell方程组里没有因子$4\pi$,而Coulomb定律\footnote{描述点电荷之间力的作用规律的定律。}和Biot-Savart定律\footnote{描述电流元之间力的作用规律的定律。}的方程里都含有因子$4\pi$。采用Gauss单位制的状况则完全相反,Maxwell方程中含有因子$4\pi$,但是Coulomb定律和Biot-Savart定律的方程里都没有因子$4\pi$。因此Gauss单位制被分类为“非理化单位制”。

对比Gauss单位制与国际单位制,电荷单位的定义有很大的区别。国际单位制特别为电现象设置一个基本单位——安培(Ampere),这动作的后果是,电荷是一种物理数量的一种独特量纲,($\SI{1}{\coulomb}=\SI{1}{\ampere \cdot \second}$)不能用机械单位(\si{\kilo\gram}、\si{\m}、\si{\s})来表达。Coulomb定律方程为
\begin{equation*}
	\mbf{F} = \frac{1}{4\pi\epsilon_0} \frac{e_1e_2}{r^3}\mbf{r}
\end{equation*}
其中,$\mbf{F}$是Coulomb力,$\epsilon_0$是真空介电常数,$e_1$和$e_2$是两个相互作用的电荷,$r$是这两个