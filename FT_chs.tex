\documentclass[hyperref,UTF8,a4paper,openany]{ctexbook}
%\usepackage{ctex}
\usepackage{zhnumber}
\usepackage{graphicx}
\usepackage{subfigure}

\usepackage{latexsym}
\usepackage{amsmath}
\usepackage{amssymb}
\usepackage{amsfonts}
\usepackage{mathrsfs}
\usepackage{amsthm}
\usepackage{siunitx}

\usepackage{cases}

\usepackage{multirow}
\usepackage{textcomp}
\usepackage[top=2.54cm, bottom=2.54cm, left=3.18cm, right=3.18cm]{geometry}
\usepackage[toc]{multitoc}

\usepackage{asymptote}

%\usepackage{breqn}

\usepackage{hyperref}
\hypersetup{colorlinks,
        linkcolor=black,
        filecolor=black,
        urlcolor=blue,
        citecolor=black,
        pdftitle={《场论》},
        pdfauthor={张大鹏},
        pdfsubject={相对论力学, 电动力学, 广义相对论},
        pdfkeywords={相对论力学, 电动力学, 广义相对论},
        pdfproducer={XeLaTeX},
		pdfborder=0 0 1}

\setcounter{secnumdepth}{3} %使subsubsection也有编号

\usepackage{fancyhdr}	%页眉页脚

\usepackage{enumitem}
\setlist[enumerate,1]{leftmargin=0pt,itemindent=2em,itemsep=0ex,listparindent=2em,parsep = 0ex,topsep = 1ex}
\setlist[enumerate,2]{leftmargin=0pt,itemsep=0ex,listparindent=2em,parsep = 0ex,topsep = 1ex}
\setlist[enumerate,3]{leftmargin=0pt,itemsep=0ex,listparindent=2em,parsep = 0ex,topsep = 1ex}
\setlist[itemize,1]{itemsep = 0ex, parsep = 0ex, topsep = 1ex}
\setlist[description]{listparindent=2em, itemsep=0ex, parsep = 0ex,topsep = 1ex}
% \setlist[description,1]{leftmargin=0pt,itemindent=2em}
% \setlist[enumerate,1]{leftmargin=0pt,itemindent=2em}
% \setlist[enumerate,2]{leftmargin=1em,itemindent=2em}
% \setlist[enumerate,3]{leftmargin=1em,itemindent=2em}

\ctexset{figurename={\kaishu 图},
	tablename={表},
	contentsname={目\quad 录},
	listfigurename={插图目录}}
\newcommand{\eps}{\varepsilon}
\newcommand{\mbf}{\boldsymbol}
\newcommand{\mrm}[1]{\boldsymbol{\mathnormal{#1}}}
\newcommand{\bnb}{\mbf{\nabla}}
\newcommand{\pl}{\partial}
\newcommand{\arcosh}{\mathrm{arcosh}\,}
\newcommand{\sn}{\mathrm{sn}\,}
\newcommand{\cn}{\mathrm{cn}\,}
\newcommand{\tn}{\mathrm{tn}\,}
\newcommand{\dn}{\mathrm{dn}\,}
\newcommand{\am}{\mathrm{am}\,}
\newcommand{\mathd}{\mathrm{d}}
\renewcommand{\bf}{\heiti}
\renewcommand{\bfseries}{\heiti}

\newtheoremstyle{theoremwithoutdot}% 类型名
  {}%                   Space above, empty = `usual value'
  {}%                   Space below
  {\kaishu}%                   Body font
  {}%         Indent amount (empty = no indent, \parindent = para indent)
  {\heiti}%          Thm head font
  {}%                   Punctuation after thm head
  {1em}%                Space after thm head
  {\thmname{#1}\thmnumber{~#2}\thmnote{~(#3)}}%                   Thm head spec
  
 \newtheoremstyle{solutionstyle}% 类型名
  {}%                   Space above, empty = `usual value'
  {}%                   Space below
  {}%                   Body font
  {}%         Indent amount (empty = no indent, \parindent = para indent)
  {\heiti}%          Thm head font
  {}%                   Punctuation after thm head
  {1em}%                Space after thm head
  {\thmname{#1}\thmnumber{~#2}\thmnote{~(#3)}}%                   Thm head spec

\theoremstyle{theoremwithoutdot}
\newtheorem{defi}{定义}[section]
\newtheorem{theorem}{定理}[section]
\newtheorem{lemma}[theorem]{引理}
\newtheorem{prop}[theorem]{命题}
\newtheorem{coro}{推论}[theorem]
\newtheorem{remark}{注}
\newtheorem{example}{例}[chapter]
\newtheorem{question}{题}[section]
\newtheorem{property}[theorem]{性质}
%\renewcommand{\thequestion}{\arabic{chapter}.\arabic{section}.\arabic{question}}
\theoremstyle{solutionstyle}
\newtheorem*{solution}{解}
%去掉证明后面的点
\makeatletter
\renewenvironment{proof}[1][\proofname]{\par%
\pushQED{\qed}%
\normalfont \topsep6\p@\@plus6\p@\relax%
\trivlist%
\item[\hskip\labelsep%
#1]\ignorespaces%
}{%
\popQED\endtrivlist\@endpefalse%
}
\makeatother
\renewcommand{\proofname}{{\heiti 证明}}

%\renewcommand{\thefootnote}{\fnsymbol{footnote}}
%带圈脚注
\usepackage{pifont}
\usepackage[perpage,stable]{footmisc}  %每页脚注重新编号
\renewcommand{\thefootnote}{\ding{\numexpr191+\value{footnote}}}
% 脚注中的脚注序号不用上标,正文中的脚注号保持不变
\makeatletter
\def\my@makefnmark{\hbox{\normalfont\@thefnmark\space}}
\let\my@save@makefntext\@makefntext
\long\def\@makefntext#1{{%
  \let\@makefnmark\my@makefnmark
  \my@save@makefntext{#1}}}

\allowdisplaybreaks[3]

\title{《场论》笔记}
\author{张大鹏}
%\date{}

\makeatletter
\def\cleardoublepage{\clearpage\if@twoside \ifodd\c@page\else
\hbox{}
\vspace*{\fill}

\vspace{\fill}
\thispagestyle{empty}
\newpage
\if@twocolumn\hbox{}\newpage\fi\fi\fi}
\makeatother

%在插图目录图编号前加上“图”
%\usepackage{titletoc}
%\titlecontents{figure}[0.5cm]{\songti}{\figurename~\thecontentslabel\quad}{\hspace*{-1.5cm}}{\titlerule*[0.12cm]{.}\contentspage}[\addvspace{6pt}]

%旧的样式
%\usepackage{titlesec}
%\titleformat{\section}{\bf \xiaosan}{\thesection}{1em}{}

%新的样式
\usepackage{titlesec}
\usepackage{xcolor,colortbl}
\usepackage{tgpagella}
\usepackage[T1]{fontenc}
 
\definecolor{titlecolor}{rgb}{0.503906,0.503906,0.734375}
\newcommand{\hwyk}{\heiti}
 
\newcommand{\mytitle}[1]{
\begin{tabular}{p{0.01\textwidth}p{0.99\textwidth}}
\cellcolor{black} &\cellcolor{titlecolor} \textcolor{white}
{\newline\hwyk\LARGE 第\zhnumber{\thechapter}章 \  \ #1}
\end{tabular}
\arrayrulewidth=0.4pt
}
 
\newcommand{\mycont}[1]{
\vspace{-0.5cm}
\begin{tabular}{p{0.01\textwidth}p{0.99\textwidth}}
\cellcolor{black} &\cellcolor{titlecolor} \textcolor{white}
{\newline\hwyk\LARGE #1}
\end{tabular}
\arrayrulewidth=0.4pt
\vspace{-2.0cm}
}
 
\newcommand{\mysection}[1]{
\setlength\arrayrulewidth{1pt}\arrayrulecolor{titlecolor}
\begin{tabular}{p{0.01\textwidth}p{0.99\textwidth}}
\hline
\cellcolor{titlecolor} &  \textcolor{black}{\hwyk\LARGE \thesection ~ #1}
\end{tabular}
\arrayrulewidth=0.4pt
\vspace{-0.7cm}
}

\newcommand{\mysubsection}[1]{
\setlength\arrayrulewidth{1pt}\arrayrulecolor{titlecolor}
\begin{tabular}{p{0.01\textwidth}p{0.99\textwidth}}
\hline
\cellcolor{titlecolor} &  \textcolor{black}{\hwyk\Large \thesubsection ~ #1} \\
\hline
\end{tabular}
\arrayrulewidth=0.4pt
\vspace{-0.5cm}
}

\newcommand{\mysubsubsection}[1]{
\setlength\arrayrulewidth{1pt}\arrayrulecolor{titlecolor}
\begin{tabular}{p{0.01\textwidth}p{0.95\textwidth}p{0.01\textwidth}}
\hline
\cellcolor{titlecolor} &  \textcolor{black}{\hwyk\Large \thesubsubsection ~ #1} & \cellcolor{titlecolor} \\
\hline
\end{tabular}
\arrayrulewidth=0.4pt
\vspace{-0.5cm}
}

\begin{document}

\maketitle

\frontmatter

\pagestyle{fancy}
\renewcommand{\chaptermark}[1]{\markboth{第\zhnumber{\thechapter}章\ \ #1}{}}
\renewcommand{\sectionmark}[1]{\markright{\thesection\ \ \rm #1}{}}	%这两个命令第一次必须出现在\pagestyle{fancy}之后,否则\pagestyle{fancy}会覆盖掉其效果
\markboth{\leftmark}{\rightmark}
\fancyhf{}
\fancyhead[CO]{\rightmark}
\fancyhead[LE,RO]{$\cdot$\, \thepage\, $\cdot$}
\fancyhead[CE]{\leftmark}
\renewcommand{\headrulewidth}{0.4pt}

\titleformat{\chapter}[hang]{\hwyk\LARGE}
{}{0mm}{\hspace{-0.4cm}\mycont}

\tableofcontents

\clearpage

\mainmatter
 
\titleformat{\chapter}[hang]{\hwyk\LARGE}
{}{0mm}{\hspace{-0.4cm}\mytitle}
 
\titleformat{\section}[hang]{\hwyk\LARGE}
{}{0mm}{\hspace{-0.5cm}\mysection}

\titleformat{\subsection}[hang]{\hwyk\large}
{}{0mm}{\hspace{-0.5cm}\mysubsection}

\titleformat{\subsubsection}[hang]{\hwyk\large}
{}{0mm}{\hspace{-0.5cm}\mysubsubsection}
 
\titlespacing{\chapter}
{0pc}{1.5ex plus .1ex minus .2ex}{.2pc}

\chapter{相对性原理}

\section{相互作用的传播速度}

为了描述发生的事件,必须有一个所谓的{\bf 参考系}。与经典力学中的参考系不同的是,现在的参考系不仅需要一个坐标系,还需要一个固定在其中的钟。此处坐标系用来刻画其中物体的位置,而钟用来指示时间。

经典力学中已经提到过所谓{\bf Galileo相对性原理}:力学定理在所有惯性系中形式相同。一切惯性系等价,无法通过力学实验区分哪个惯性系更基本。换句话说,表示力学定律的方程对于由一个惯性系到另一个惯性系的时间和坐目标各种变换来说是不变的。也就是说,描述力学定律的方程,用不同的惯性系的坐标与时间写出来,将具有相同的形式。而且,有理由认为:所有的自然定律在所有惯性系中都是相同的,即所谓的{\bf 相对性原理}。

粒子间的相互作用在经典力学中是瞬时传播的,然而,实验表明,瞬时的相互作用在自然界中是不存在的。实际上,如果相互作用的物体中的一个发生任何变动,仅仅在过了一段时间之后才能影响到其他物体。用这两个物体之间的距离除以这段时间,就得到{\bf 相互作用的传播速度}。

这个速度,严格地说,应该称为相互作用的最大传播速度。这个速度仅仅决定于某一物体的变动开始在第二个物体上表现出来所需要的时间。而相互作用的最大传播速度的存在,也就暗示着,自然界中物体运动的速度一般不可能大于这个速度。假如真的有这种运动存在,我们可以利用这种运动实现一个相互作用,使得此相互作用的传播速度大于前面的最大传播速度。从一个粒子向另一个粒子传播的相互作用往往叫做“信号”,它由第一个粒子发出,将第一个粒子所经历的变化“通知”第二个粒子。因此相互作用的传播速度也可以称为{\bf 信号速度}。根据相对性原理可以推断相互作用的传播速度在所有惯性系中都是相同的,即相互作用的传播速度是一个普适常数。

后面将会证明,这个恒定速度就是光在真空中的速度。{\bf 光速}常用$c$来表示,其值等于
\begin{equation*}
	c = \SI{2.998e8}{\metre/\second}
\end{equation*}
这个速度很大,因此经典力学在大多数情况下都足够精确。

把相对性原理和相互作用传播速度的有线性结合起来,就得到{\bf Einstein相对性原理},它不同于Galileo相对性原理,Galileo的相对性原理基于无限大的相互作用传播速度。

以Einstein相对性原理(简称为相对性原理)为基础的力学,称为{\bf 相对论力学}。在运动物体的速度远小于光速的极限情形下,可略去相互作用传播速度有限对于运动的影响,此时相对论力学就变成{\bf 经典力学}。在相对论力学的公式中,取$c \to \infty$的极限,就可由相对论力学在形式上过渡到经典力学。

很容易说明,绝对时间的概念是与Einstein相对性原理完全冲突的。在以绝对时间为基础的经典力学中,速度合成的通用法则是有效的,即复合运动的速度等于这个运动的各个速度的矢量和。将此法则用于相互作用的传播即可得出,互相作用在不同的惯性系中的传播速度必然是不同的,这就与相对性原理冲突了。但是,实验则完全证实了相对性原理,1881年Michelson-Morley实验表明,光速与其传播方向并无关系;然而,按照经典力学,光速在与地球运动相同的方向上,应该比在与地球运动方向相反的方向上小。

因此,相对性原理直接导出时间不是绝对的,即在不同的参考系中,时间的流逝也是不同的。所以,“两个不同的事件之间有一定的时间间隔”这样的陈述,仅在指定了参考系的情况下才有意义。特别地,在某一个参考系中同时发生的事件,对另一个参考系来说并不一定是同时的。

设有两个惯性系$K$和$K'$,其坐标轴分别为$xyz$和$x'y'z'$,而$K'$则相对于$K$沿$x$和$x'$轴向右运动(如图\ref{chapter8:同时的相对性}所示)。

\begin{figure}[htb]
\centering
\begin{asy}
	texpreamble("\usepackage{xeCJK}");
	texpreamble("\setCJKmainfont{SimSun}");
	usepackage("amsmath");
	import graph;
	size(250);
	//同时的相对性
	pair O,OO,i,j,k;
	O = (0,0);
	OO = (1,0.5);
	i = dir(0);
	j = dir(45-180);
	k = dir(90);
	draw(Label("$x$",EndPoint),O--5*i,Arrow);
	draw(Label("$y$",EndPoint),O--1*j,Arrow);
	draw(Label("$z$",EndPoint),O--4*k,Arrow);
	draw(Label("$x'$",EndPoint),OO--OO+4*i,Arrow);
	draw(Label("$y'$",EndPoint),OO--OO+(1-0.5/Sin(degrees(j)))*j,Arrow);
	draw(Label("$z'$",EndPoint),OO--OO+3.5*k,Arrow);
	real l;
	l = 0.2;
	draw(Label("$B$",EndPoint),OO+i--OO+i+l*k);
	draw(Label("$A$",EndPoint),OO+2*i--OO+2*i+l*k);
	draw(Label("$C$",EndPoint),OO+3*i--OO+3*i+l*k);
	draw(OO+1.8*i+0.5*l*k--OO+1.2*i+0.5*l*k,Arrow);
	draw(OO+2.2*i+0.5*l*k--OO+2.8*i+0.5*l*k,Arrow);
\end{asy}
\caption{同时的相对性}
\label{chapter8:同时的相对性}
\end{figure}

设信号从$x'$轴上某一点$A$向两个相反的方向发出,既然信号在$K'$系中的传播速度在所有的惯性系中都一样,在两个方向上都等于$c$,因此$K'$系中的观察者将观察到信号同时到达$B$点和$C$点。

然而在$K$系的观察者看来,同样的两个事件(即信号到达$B$点和$C$点)并不是同时的。按照相对性原理,信号相对于$K$系的速度也是$c$,而且因为$B$点在$K$系中向着信号发出的位置移动,而$C$点则背离信号发出的位置,所以在$K$系中,信号到达$B$点要比到达$C$点早。

\section{间隔}

\subsection{事件的间隔}

一个事件是由其发生的地点及其发生的时间来描述的,因此在某一实物粒子上所发生的事件可由粒子的三个坐标以及事件发生的时间来决定。

为表述便利起见,建立一个假象的四维空间,在此四维空间的四个轴中,三个用来刻画位置坐标,一个用来标示时间。在这个空间内,事件可用点来表示,这个点称为{\bf 世界点}。在这个假象的四维空间内,每个粒子都对应于一条线,称为{\bf 世界线}。这条在线的各点决定了粒子在所有时刻的坐标。

现在考虑用数学形式来表示光速不变原理。为此,我们考虑两个彼此以恒定速度作相对运动的参考系$K$和$K'$。其中两个坐标系取为$x$轴和$x'$轴重合,而$y$轴和$z$轴则分别与$y'$轴和$z'$轴平行,并以$t$和$t'$分别表示在$K$和$K'$参考系内的时间。

设第一个事件是:在$K$系内的$t_1$时刻从具有坐标$x_1,y_1,z_1$(在同一参考系中)的点发出一个光速传播的信号。再设第二个事件是:信号在$t_2$时刻到达点$x_2,y_2,z_2$。由于信号传播的速度是$c$,因此可得两个事件坐目标关系
\begin{equation}
	(x_2-x_1)^2+(y_2-y_1)^2+(z_2-z_1)^2-c^2(t_2-t_1)^2 = 0
	\label{chapter8:K系中两个事件坐目标关系}
\end{equation}
而同样的两个事件,也可以在$K'$系中观察。设第一个事件在$K'$内的坐标为$x'_1,y'_1,z'_1,t'_1$,而第二个事件的坐标则为$x'_2,y'_2,z'_2,t'_2$。根据光速不变原理,在$K'$系中信号传播的速度也是$c$,因此可有
\begin{equation}
	(x'_2-x'_1)^2+(y'_2-y'_1)^2+(z'_2-z'_1)^2-c^2(t'_2-t'_1)^2 = 0
	\label{chapter8:K'系中两个事件坐目标关系}
\end{equation}

假如$x_1,y_1,z_1,t_1$和$x_2,y_2,z_2,t_2$是任意两个事件的坐标,则
\begin{equation}
	s_{12} = \sqrt{c^2(t_2-t_1)^2 - (x_2-x_1)^2 - (y_2-y_1)^2 - (z_2-z_1)^2}
	\label{chapter8:间隔的定义}
\end{equation}
称为这两个事件的{\bf 间隔}。因此,由光速不变原理可得,如果两个事件的间隔在某一个坐标系中为零,那么它在所有其他坐标系中均为零。

如果两个事件彼此无限接近,那么其间隔$\mathrm{d}s$将满足
\begin{equation}
	\mathrm{d} s^2 = c^2 \mathrm{d} t^2 - \mathrm{d} x^2 - \mathrm{d} y^2 - \mathrm{d} z^2
	\label{chapter8:无限小间隔}
\end{equation}
从数学形式上来看,式\eqref{chapter8:间隔的定义}和\eqref{chapter8:无限小间隔}的形式可以设想为四维空间内两点之间的距离(该空间的四个轴分别为$x,y,z$和$ct$)。但是与普通几何的法则之间有一个根本区别,即间隔的平方和中沿不同轴的坐标差平方是以相异而非相同的运算符号求和的。\footnote{二次式\eqref{chapter8:无限小间隔}所描述的四维几何,是Minkowski为相对论而引入的。这种几何称为{\bf 伪Euclid几何}。有些文献中将四维空间的四个坐标轴记作$x,y,z$和$\mathrm{i}ct$,这样间隔将在形式上表示为平方和的形式。}

前面已经证明,如果在某一惯性系中有$\mathrm{d}s=0$,则在任意其他惯性系中同样有$\mathrm{d}s'=0$。此外,$\mathrm{d}s$和$\mathrm{d}s'$为同阶的两个无穷小量。由此可知,$\mathrm{d}s^2$与$\mathrm{d}s'^2$必须成比例,即
\begin{equation}
	\mathrm{d}s^2 = a\mathrm{d}s'^2
\end{equation}
而且其中系数$a$仅与两个惯性系的相对速度大小有关。系数$a$不可能与坐标或时间有关系,否则就违背了空间和时间的均匀性。系数$a$也不可能与惯性系的相对速度方向有关,否则就违背了空间的各向同性。

考虑三个参考系$K,K_1,K_2$,令$V_1,V_2$为$K_1,K_2$相对于$K$的速度,此时有
\begin{equation*}
	\mathrm{d}s^2 = a(V_1) \mathrm{d}s_1^2 ,\quad \mathrm{d}s^2 = a(V_2) \mathrm{d}s_2^2 
\end{equation*}
以及
\begin{equation*}
	\mathrm{d}s_1^2 = a(V_{12}) \mathrm{d}s_2^2
\end{equation*}
式中$V_{12}$是$K_2$相对于$K_1$速度的大小。比较这三组关系可得
\begin{equation}
	\frac{a(V_2)}{a(V_1)} = a(V_{12})
	\label{chapter8:间隔之间的比例系数关系}
\end{equation}
而$V_{12}$不仅依赖于矢量$\mbf{V}_1$和$\mbf{V}_2$的大小,还依赖于它们之间的夹角。\footnote{虽然在相对论下,速度的合成公式于经典力学下不同,但两个速度的合成速度依然是依赖于两个速度的大小和它们的夹角。}但这个夹角在式\eqref{chapter8:间隔之间的比例系数关系}的左端并未出现,因此显然式\eqref{chapter8:间隔之间的比例系数关系}只有当函数$a(V)$为常数时才成立,而且该常数必然为$1$。因此,
\begin{equation}
	\mathrm{d} s^2 = \mathrm{d} s'^2
	\label{chapter8:无限小间隔相等}
\end{equation}
再从无限小间隔相等可得有限间隔相等,即$s=s'$。

因此,我们得到一个很重要的结论:任意两个事件的间隔在所有惯性系中都是相等的,即当由一个惯性系变换到另一个惯性系时,这两个事件的间隔是不变的。这个不变性就是光速不变的数学表示。

\subsection{类时间隔、类空间隔和类光间隔}

再次假设$x_1,y_1,z_1,t_1$和$x_2,y_2,z_2,t_2$是在某一个惯性系$K$内的两个事件的坐标,是否存在一个惯性系$K'$,使得这两个事件在同一位置发生?在$K$系中,两个事件之间的间隔为
\begin{equation*}
	s^2 = c^2(t_2-t_1)^2 - (x_2-x_1)^2 - (y_2-y_1)^2 - (z_2-z_1)^2 = c^2 t_{12}^2 - l_{12}^2
\end{equation*}
在$K'$系中,两个事件之间的间隔则为
\begin{equation*}
	s'^2 = c^2 t'^2_{12} - l'^2_{12}
\end{equation*}
根据间隔的不变性,可得
\begin{equation*}
	c^2 t_{12}^2 - l_{12}^2 = c^2 t'^2_{12} - l'^2_{12}
\end{equation*}
在$K'$系中两个事件在同一位置发生,即要求$l'^2_{12}=0$,此时即有
\begin{equation*}
	s^2 = c^2 t_{12}^2 - l_{12}^2 = c^2 t'^2_{12} > 0
\end{equation*}
由此可得,如果$s^2>0$,即如果两个事件的间隔是实数的话,则存在一个惯性系,使得两个事件在同一位置发生。实数间隔称为{\bf 类时间隔}\footnote{在一些其他文献上,间隔的符号与此处相反,因此间隔的类时性和类空性判定也是相反的。}。

因此,若两个事件的间隔是类时的,那么就有这样一个参考系存在,在其中两个事件发生于同一位置。若任何两个事件在同一物体上发生,那么它们之间的间隔必然是类时的。

依然假设$x_1,y_1,z_1,t_1$和$x_2,y_2,z_2,t_2$是在某一个惯性系$K$内的两个事件的坐标,是否存在一个惯性系$K'$,使得这两个事件在同一时刻发生?于之前的讨论相同,同样有
\begin{equation*}
	c^2 t_{12}^2 - l_{12}^2 = c^2 t'^2_{12} - l'^2_{12}
\end{equation*}
在$K'$系中两个事件在同一时刻发生,即要求$t'^2_{12}=0$,此时即有
\begin{equation*}
	s^2 = c^2 t_{12}^2 - l_{12}^2 = -l'^2_{12} < 0 
\end{equation*}
因此,仅当两个事件的间隔是虚数的情况下,才存在一个惯性系使得两个事件同时发生。虚数间隔称为{\bf 类空间隔}。

因此,若两个事件的间隔是类空的,那么就有这样一个参考系存在,在其中两个事件同时发生。

恒为零的间隔称为{\bf 类光间隔}。

由于间隔的不变性,将间隔分为类时间隔、类空间隔和类光间隔具有绝对的意义,即间隔的类空性、类时性或类旋光性与参考系无关。

取某一个事件$O$作为时间及空间坐目标原点,现在我们来研究所有其他事件对于本事件$O$的关系。首先只考虑一维空间的情况,如图\ref{chapter8:光锥}所示。一个当$t=0$时经过$x=0$点的粒子的匀速直线运动可以用一条直线来表示,这条直线过$O$点,对$t$轴的斜率等于粒子的速度。因为物体最大的速度为$c$,所以这条直线与$t$轴成角也有一个最大值。图\ref{chapter8:光锥}中有两条直线,代表两个信号经过事件$O$(即当$t=0$时经过$x=0$)以光速向相反的两个方向传播。所有代表粒子运动的直线只能在$aOc$和$dOb$两个区域内。显然,在直线$ab$及$cd$上有$x=\pm ct$。世界点位于区域$aOc$中的那些事件满足$c^2t^2-x^2 > 0$。即,这个区域内的任何事件与事件$O$之间的间隔都是类时的。在这个区域内$t>0$,即其中所有的事件都发生在事件$O$之“后”。但是两个事件若被类时间隔所分开,无论在哪一个参考系中都不可能同时发生。因此也不可能找到一个参考系使得$aOc$区域中的任何事件会在事件$O$之“前”发生,即在$t<0$时发生。因此,$aOc$区域内的所有事件对$O$来说在任何参考系中都是未来的事件。所以,这个区域对事件$O$来说可称为{\bf 绝对未来}。

\begin{figure}[htb]
\centering
\begin{asy}
	texpreamble("\usepackage{xeCJK}");
	texpreamble("\setCJKmainfont{SimSun}");
	usepackage("amsmath");
	import graph;
	size(250);
	//光锥与因果性
	real theta;
	theta = 70;
	draw(Label("$x$",EndPoint),1*dir(180)--1*dir(0),Arrow);
	draw(Label("$t$",EndPoint),2*dir(-90)--2*dir(90),Arrow);
	draw(2*dir(theta-180)--2*dir(theta),Arrow);
	draw(2*dir(-theta)--2*dir(-theta+180),Arrow);
	label("$O$",(0,0),SW);
	label("$a$",2*dir(-theta+180),N);
	label("$b$",2*dir(-theta),S);
	label("$c$",2*dir(theta),N);
	label("$d$",2*dir(theta-180),S);
	label("绝对未来",1.5*dir(90));
	label("绝对过去",1.5*dir(-90));
	label("绝对",0.5*dir(0),N);
	label("分割",0.5*dir(0),S);
	label("绝对",-0.5*dir(0),N);
	label("分割",-0.5*dir(0),S);
\end{asy}
\caption{光锥与因果性}
\label{chapter8:光锥}
\end{figure}

完全类似地,所有在区域$bOd$内的事件对$O$来说都是{\bf 绝对过去},即本区域内的事件,无论在何参考系中都在事件$O$之前发生。

最后,再研究$dOa$及$cOb$两个区域。本区域内的任何事件与事件$O$的间隔都是类空的。这些事件在任何参考系中都发生在空间的不同位置。因此,这些区域对于$O$来说都可称为{\bf 绝对分割}。但是,关于这些事件的“同时”、“较早”和“较晚”等概念都是相对的。对这些区域中的任何事件来说,在一些参考系中,此事件在事件$O$之后发生;在另一些参考系中,此事件在事件$O$之前发生;也存在一个参考系,在其中此事件与事件$O$同时发生。

如果考虑三个空间坐标轴,那么图\ref{chapter8:光锥}中两条相交的直线,在四维空间坐标系$x,y,z,t$中将称为一个“圆锥体”$x^2+y^2+z^2-c^2t^2 = 0$,圆锥体的轴与$t$轴重合,称为{\bf 光锥}。光锥内的区域即为事件$O$的绝对未来和绝对过去。

两个事件仅仅在其间隔是类时间隔的情况下,彼此才能有因果关系;这可以由相互作用的传播速度不能大于光速这一事实直接推出来。也只有对这些事件来说,“较早”和“较晚”才具有绝对意义,而这一点又是使因果概念具有意义的必要条件。

\section{固有时}

假设我们在某一惯性参考系(不妨称之为静止系)中观察一只钟,这只钟相对于我们可作任意形式的运动。在各个不同的时刻,该钟的运动可以认为是匀速的。因此,在每一时刻,我们可以引入一个固连于运动钟上的坐标系,这个坐标系也是一个惯性系。

在无限小的时间间隔$\mathrm{d}t$内,运动的钟前进的距离是$\sqrt{\mathrm{d}x^2 + \mathrm{d}y^2 + \mathrm{d}z^2}$。而在与运动钟固连的坐标系内,钟是静止的,即$\mathrm{d}x'= \mathrm{d}y' = \mathrm{d}z' = 0$,由间隔的不变性,可得
\begin{equation*}
	\mathrm{d}s^2 = c^2 \mathrm{d}t^2 - \mathrm{d}x^2 - \mathrm{d}y^2 - \mathrm{d}z^2 = c^2 \mathrm{d} t'^2
\end{equation*}
由此可得在与钟固连坐标系中的时间为
\begin{equation*}
	\mathrm{d}t' = \mathrm{d}t \sqrt{1-\frac{\mathrm{d}x^2 + \mathrm{d}y^2 + \mathrm{d}z^2}{c^2 \mathrm{d}t^2}}
\end{equation*}
再由
\begin{equation*}
	\frac{\mathrm{d}x^2 + \mathrm{d}y^2 + \mathrm{d}z^2}{\mathrm{d}t^2} = v^2
\end{equation*}
其中$v$为运动钟在静止系中的速度,所以有
\begin{equation}
	\mathrm{d}t' = \frac{\mathrm{d} s}{c} = \mathrm{d}t\sqrt{1-\frac{v^2}{c^2}}
	\label{chapter8:无穷小固有时}
\end{equation}
将上式积分,我们可以得到,当静止的钟所行走的时间为$t_2-t_1$时,运动的钟所指示的时间为
\begin{equation}
	t'_2 - t'_1 = \int_{t_1}^{t_2} \mathrm{d}t\sqrt{1-\frac{v^2}{c^2}}
	\label{chapter8:固有时}
\end{equation}
随着某一给定物体一同运动的钟所指示的时间,称为该物体的{\bf 固有时}。由式\eqref{chapter8:无穷小固有时}和\eqref{chapter8:固有时}可知,一个运动物体的固有时永远比在静止系内相对应的时间间隔小。换句话说,运动的钟比静止的钟走得慢些。

假设有一只钟相对于某一惯性系$K$作匀速直线运动。同这只钟固连着的参考系$K'$也是惯性系。从$K$系的观察者来看,$K'$系的钟走得慢。反过来说,从$K'$系的观察者来看,$K$系的钟走得慢。为了是我们相信这是不矛盾的,可以注意下面的事实。为了确定$K'$系内的钟比$K$系内的钟慢,我们必须按下述方法来做。假设在某一时刻,$K'$内的那只钟经过$K$内的一只钟旁边,而在这一时刻,两只钟所指示的时间恰好一样。为了比较$K$及$K'$内钟的快慢,我们必须再次将$K'$内同一只动钟的读数与$K$内的钟的读数作比较。但是在新的时刻,$K'$内的钟将从$K$内的另一些钟旁边经过,现在我们就将运动的钟与那些钟比较。这时我们发现,$K'$内的钟比藉以比较的$K$内的钟走得慢。由此可见,为了比较两个参考系内的钟的快慢,我们需要在一个参考系内有几只钟,而在另一个参考系内只有一只钟。因此这种过程对这两个参考系来说,并不是对称的。与另一参考系内不同的一些钟比较的那一只钟总是走得慢的钟。

假设我们有两只钟,其中之一描绘一闭合路径,又回到出发点(即静止的钟所在之点),显然,与静止的钟相比,运动的钟慢了。相反,若设想运动的钟静止(即以运动的钟为参考系),就不能做上面的推论了。因为那只钟既然描绘了一条闭合曲线,它的运动就不是匀速直线运动,因而与之相连的参考系就不再是惯性系。因为自然定律只有在惯性系内才是一样的,与静止的钟相连的系统(惯性系)及与运动的钟相连的系统(非惯性系)具有不同的特性,因而导致静止的钟应当变慢这个结论的论证就不对了。

一只钟所只是的时间间隔,等于沿着钟的世界线而取的积分$\displaystyle \dfrac{1}{c}\int\mathrm{d}s$。假如钟是静止的,则它的世界线是一条与$t$轴平行的直线;假如钟在闭合路径上作非匀速运动而且又回到出发点,那么它的世界线是一条曲线,这条曲线经过静止钟的直的世界线的起点和终点。另一方面,可以看到,静止钟所指示的时间间隔永远比运动的钟所指示的大。因此,我们得到一个结论,在两个世界点之间所取的积分$\displaystyle \int\mathrm{d}s$,只有在连接这两点之间的直线上能取得其最大值。\footnote{当然,需假设$a$及$b$两点和连接它们的曲线满足该曲线上所有的线元$\mathrm{d}s$都是类时的。这个性质与四维几何的伪Euclid特性有关。在Euclid空间中,这个积分沿直线当然取得最小值,此处定义的Minkowski空间的间隔与Euclid空间的间隔(距离)相比,多了一个负号,因此积分在直的世界在线取得最大值而非最小值。}

\section{Lorentz变换}

在经典力学中,已知一个惯性系$K$中事件的坐标$x,y,z,t$,获得相对$K$系以速度$V$匀速运动的$K'$系中的坐标$x',y',z',t'$是很简单的,称为{\bf Galileo变换}。即
\begin{equation}
\begin{cases}
	\mbf{r} = \mbf{r}'+\mbf{V}t \\
	t = t'
\end{cases}
\end{equation}
很容易证明,Galileo变换不能满足相对论的要求,即这个变换不能满足事件之间的间隔不变。

两个事件之间的间隔,可以认为是四维Minkowski空间中两个世界点之间的距离。因此,满足相对论要求的变换(即在四维Minkowski空间中的平移和旋转),必须保持四维空间$x,y,z,ct$内的所有距离不变。对于平移,这是显然的。因此只需考虑四维空间中的旋转即可。

四维空间内的一切转动可以分解为六个分别在六个平面$xy,yz,zx,tx,ty,tz$内的转动,其中前三个转动仅仅变换空间坐标,它们对应通常的空间转动,只需将$K$系和$K'$系的坐标轴取得相互平行即可避免。

现在研究在$tx$平面内的转动,这时$y$和$z$坐标是不变的。具体地说,这个变换必须使差值$c^2t^2-x^2$,即点$(ct,x)$到原点的“距离”的平方保持不变。因此,新旧坐目标关系最一般的表达式为
\begin{equation}
\begin{cases}
	x = x'\cosh \psi + ct' \sinh \psi \\
	ct = x'\sinh \psi + ct' \cosh \psi
\end{cases}
\label{chapter8:四维空间中旋转的一般形式}
\end{equation}
式中$\psi$为转动角。式\eqref{chapter8:四维空间中旋转的一般形式}与坐标轴转动变换通常公式的不同之处在于,此处的旋转公式中为双曲函数而非三角函数。这就是Euclid几何与伪Euclid几何之间的差别。假设$K'$系以速度$V$沿$x$轴相对$K$系作匀速运动,则$x,t$与$x',t'$之间的变换关系式中转动角仅与相对速度$V$有关。

研究$K'$系的原点在$K$系中的运动。这时$x'=0$,而式\eqref{chapter8:四维空间中旋转的一般形式}可写成
\begin{equation*}
	x = ct'\sinh \psi,\quad ct = ct'\cosh \psi
\end{equation*}
由此可得
\begin{equation*}
	\frac{x}{ct} = \tanh \psi
\end{equation*}
而$\dfrac{x}{t}$显然是$K'$相对$K$的速度$V$,因此有
\begin{equation*}
	\tanh \psi = \frac{V}{c}
\end{equation*}
由此可得
\begin{equation}
\begin{cases}
	x = \dfrac{x'+Vt'}{\sqrt{1-\dfrac{V^2}{c^2}}} \\
	y = y' \\
	z = z' \\
	t = \dfrac{t'+\dfrac{V}{c^2}x'}{\sqrt{1-\dfrac{V^2}{c^2}}}
\end{cases}
\label{chapter8:Lorentz变换}
\end{equation}
这组变换称为{\bf Lorentz变换},是今后讨论的基础。其逆变换只需将$V$替换为$-V$即可得到\footnote{当然直接求解方程组\eqref{chapter8:Lorentz变换}也能得到相同的结果。}。

由式\eqref{chapter8:Lorentz变换}可见,取$c\to\infty$的经典力学极限,Lorentz变换事实上就过渡到Galileo变换了。当$V>c$时,式\eqref{chapter8:Lorentz变换}中的$x,t$变成虚数,这与运动速度不可能大于光速的事实符合。此外我们也不可以用以光速运行的参考系,因为在这种情形下,式\eqref{chapter8:Lorentz变换}的分母为零。

当$V$比光速小很多时,我们可以用下面的近似公式代替式\eqref{chapter8:Lorentz变换}
\begin{equation}
\begin{cases}
	x = x'+Vt' \\
	y = y' \\
	z = z' \\
	t = t'+\dfrac{V}{c^2}x'
\end{cases}
\label{chapter8:近似的Lorentz变换}
\end{equation}

假设在$K$系中有一根平行于$x$轴的静止杆,假定它在$K$系内测定的长度为$\Delta x=x_2-x_1$,在$t'$时刻杆两端的坐标$x'_1,x'_2$与$x_1,x_2$的关系可由Lorentz变换\eqref{chapter8:Lorentz变换}获得
\begin{equation*}
	x_1 = \frac{x'_1 + Vt'}{\sqrt{1-\dfrac{V^2}{c^2}}},\quad x_2 = \frac{x'_2 + Vt'}{\sqrt{1-\dfrac{V^2}{c^2}}}
\end{equation*}
记杆在$K'$中的长度为$\Delta x'= x'_2-x'_1$,则有
\begin{equation*}
	\Delta x = \frac{\Delta x'}{\sqrt{1-\dfrac{V^2}{c^2}}}
\end{equation*}
杆的{\bf 固有长度}是它在与其相对静止的惯性系中的长度。如果用$l_0 = \Delta x$表示这个固有长度,用$l$表示它在任何其他参考系$K'$中的长度,即有
\begin{equation}
	l = l_0\sqrt{1-\frac{V^2}{c^2}}
	\label{chapter8:固有长度}
\end{equation}
因此,杆在与其相对静止的惯性系中最长,在相对于杆以速度$V$运动的惯性系中,它的长度就要减少一个因子$\sqrt{1-\dfrac{V^2}{c^2}}$。相对论的这个结果称为{\bf Lorentz收缩}。因为物体的横向尺度都不因运动而改变,所以它的体积$\mathscr{V}$也按照相似的公式收缩,即
\begin{equation}
	\mathscr{V} = \mathscr{V}_0\sqrt{1-\frac{V^2}{c^2}}
	\label{chapter8:固有体积}
\end{equation}

Lorentz变换与Galileo变换另外一个不同点在于Galileo变换具有可对易性,即连续两次Galileo变换(具有不同的速度$\mbf{V}_1$和$\mbf{V}_2$)的结果与施行变换的顺序无关。然而连续两次Lorentz变换的结果一般依赖于它们的顺序。可类比于刚体的有限转动,作为四维空间中的有限转动,同样具有不可对易性,除非$\mbf{V}_1$和$\mbf{V}_2$相互平行。

\section{速度的变换}

假设$K'$系相对于$K$系以速度$V$沿$x$轴运动,则根据Lorentz变换\eqref{chapter8:Lorentz变换}可有
\begin{equation*}
\begin{cases}
	\mathrm{d} x = \dfrac{\mathrm{d}x'+V\mathrm{d}t'}{\sqrt{1-\dfrac{V^2}{c^2}}} \\
	\mathrm{d} y = \mathrm{d} y' \\
	\mathrm{d} z = \mathrm{d} z' \\
	\mathrm{d} t = \dfrac{\mathrm{d}t'+\dfrac{V}{c^2} \mathrm{d}x'}{\sqrt{1-\dfrac{V^2}{c^2}}}
\end{cases}
\end{equation*}
考虑到
\begin{equation*}
	\mbf{v} = \frac{\mathrm{d} \mbf{r}}{\mathrm{d} t},\quad \mbf{v}' = \frac{\mathrm{d} \mbf{r}'}{\mathrm{d} t}
\end{equation*}
则有
\begin{equation}
\begin{cases}
	v_x = \dfrac{\mathrm{d}x}{\mathrm{d}t} = \dfrac{v'_x+V}{1+\dfrac{V}{c^2}v'_x} \\[1.5ex]
	v_y = \dfrac{\mathrm{d}y}{\mathrm{d}t} = \dfrac{v'_y\sqrt{1-\dfrac{V^2}{c^2}}}{1+\dfrac{V}{c^2}v'_x} \\[1.5ex]
	v_z = \dfrac{\mathrm{d}z}{\mathrm{d}t} = \dfrac{v'_z\sqrt{1-\dfrac{V^2}{c^2}}}{1+\dfrac{V}{c^2}v'_x}
\end{cases}
\label{chapter8:速度的变换关系}
\end{equation}
这些公式就决定了速度的变换,它们是相对论中的速度合成法则。在$c\to\infty$的极限情形下,它们就变为经典力学中的公式:
\begin{equation*}
	v_x = v'_x+V,\quad v_y = v'_y,\quad v_z = v'_z
\end{equation*}

在粒子沿$x$轴运动的特殊情况下,$v_x=v,\,v_y=v_z=0$。那么,$v'_x=v',\,v'_y=v'_z=0$,并且有
\begin{equation}
	v = \frac{v'+V}{1+v'\dfrac{V}{c^2}}
\end{equation}

很容易证明,如果两个速度各小于或等于光速,其合成速度根据这组公式也不会大于光速。

假如速度$V$比光速$c$小很多($v$可以是任意的),我们将近似地(精确到$\dfrac{V}{c}$的项)得到
\begin{equation}
\begin{cases}
	\displaystyle v_x = v'_x+V\left(1-\frac{v'^2_x}{c^2}\right) \\[1.5ex]
	\displaystyle v_y = v'_y-v'_xv'_y\frac{V}{c^2} \\[1.5ex]
	\displaystyle v_z = v'_z-v'_xv'_z\frac{V}{c^2}
\end{cases}
\end{equation}
这3个公式可以合写为一个矢量公式
\begin{equation}
	\mbf{v} = \mbf{v}' + \mbf{V} - \frac{1}{c^2}(\mbf{V}\cdot \mbf{v}') \mbf{v}'
\end{equation}

我们可以指出,在相对论的速度合成公式\eqref{chapter8:速度的变换关系}中,相加的两个速度$\mbf{v}'$和$\mbf{V}$是以不对称的方式引入的(如果它们都不沿$x$轴方向的话)。这个事实同Lorentz变换的非对易性有关,将在下节提到。

让我们这样来选择坐标轴,使粒子的速度在给定时刻是在$xy$平面内。这时粒子在$K$系内的速度分量是$v_x=v\cos \theta,\,v_y=v\sin \theta$,而在$K'$系内则为$v'_x=v'\cos \theta',\,v'_y=v'\sin \theta'$,其中$v,v'$为速度在$K$和$K'$系中的大小,$\theta,\theta'$为速度与$x$轴和$x'$轴的夹角。利用速度合成公式\eqref{chapter8:速度的变换关系},可得
\begin{equation}
	\tan \theta = \frac{v'\sqrt{1-\dfrac{V^2}{c^2}}\sin \theta'}{v'\cos \theta'+V}
	\label{chapter8:速度的方向在参考系之间的变换}
\end{equation}
这个公式决定了速度的方向从一个参考系变换到另一个参考系时的改变。

让我们来详尽地研究这个公式的另一个重要特例,即光由一个参考系变换到另一个参考系时的偏差,即所谓的{\bf 光行差}现象。在这种情形下$v=v'=c$,因而式\eqref{chapter8:速度的方向在参考系之间的变换}化为
\begin{equation}
	\tan \theta = \frac{\sqrt{1-\dfrac{V^2}{c^2}}}{\dfrac{V}{c}+\cos \theta'}\sin \theta'
\end{equation}
再由式\eqref{chapter8:速度的变换关系}可得
\begin{equation}
	\sin \theta = \frac{\sqrt{1-\dfrac{V^2}{c^2}}}{1+\dfrac{V}{c}\cos \theta'}\sin \theta',\quad \cos \theta = \frac{\dfrac{V}{c}+\cos \theta'}{1+\dfrac{V}{c}\cos \theta'}
	\label{chapter8:光行差角度变化关系}
\end{equation}
如果$V \ll c$,由式\eqref{chapter8:光行差角度变化关系}可得精确到数量级为$\dfrac{V}{c}$项的公式如下:
\begin{equation*}
	\sin \theta - \sin \theta' = -\frac{V}{c}\sin \theta'\cos \theta'
\end{equation*}
若引入光行差角$\Delta \theta = \theta'-\theta$,我们就可得到同级的近似公式
\begin{equation}
	\Delta \theta = \frac{V}{c} \sin \theta'
\end{equation}
这就是著名的光行差的基本公式。

\section{四维矢量和四维张量}

\subsection{四维矢量}

一个事件的坐标$(ct,x,y,z)$可以看成是四维空间中一个四维径向矢量的分量。我们将它的分量记为$x^i$,这里指标$i$取值为$0,1,2,3$,而且
\begin{equation*}
	x^0 = ct,\quad x^1 = x,\quad x^2 = y,\quad x^3 = z
\end{equation*}
该径向四维矢量“长度”的平方由下式给出
\begin{equation*}
	(x^0)^2-(x^1)^2-(x^2)^2-(x^3)^2
\end{equation*}
它在四维坐标系的任意转动下不变,特别是,它在Lorentz变换下不变。

更一般地,如果4个量$A^0,A^1,A^2,A^3$,在四维坐标系的变换下像四维径向矢量的分量$x^i$那样变换,我们就将这4个量的集合称为{\bf 四维矢量}$A^i$。在Lorentz变换下,
\begin{equation}
\begin{cases}
	\displaystyle A^0 = \frac{A'^0+\dfrac{V}{c}A'^1}{\sqrt{1-\dfrac{V^2}{c^2}}} \\[1.5ex]
	\displaystyle A^1 = \frac{A'^1+\dfrac{V}{c}A'^0}{\sqrt{1-\dfrac{V^2}{c^2}}} \\[1.5ex]
	A^2 = A'^2 \\[1.5ex]
	A^3 = A'^3 
\end{cases}
\end{equation}
与四维径向矢量的平方类似,任一四维矢量数值的平方定义为:
\begin{equation*}
	(A^0)^2-(A^1)^2-(A^2)^2-(A^3)^2
\end{equation*}
为表示方便起见,我们引入四维矢量分量的两种“类型”,用带上标和下目标符号$A^i$和$A_i$来标记它们。两者之间的关系是
\begin{equation}
	A_0 = A^0,\quad A_1 = -A^1,\quad A_2 = -A^2,\quad A_3 = -A^3
\end{equation}
量$A^i$称为四维矢量的{\bf 逆变分量},$A_i$称为四维矢量的{\bf 协变分量}。四维矢量的平方则取为形式
\begin{equation*}
	\sum_{i=0}^3 A^iA_i = A^0A_0+A^1A_1+A^2A_2+A^3A_3
\end{equation*}
人们通常略去求和号,将这样的求和简单记作$A^iA_i$,也就是约定遍历所有重复指标求和,而把求和号省去\footnote{这个约定称为“Einstein求和约定”。}。每对指标中必须一个为上标,另一个为下标。这种遍历“傀”指标求和的约定非常方便,可大大简化公式的书写。我们将用拉丁字母$i,k,l,\cdots$表示四维指标,其取值为$0,1,2,3$。

与四维矢量的平方类比,我们可以构造两个不同四维矢量的{\bf 标积}:
\begin{equation*}
	A^iB_i = A_iB^i = A^0B_0+A^1B_1+A^2B_2+A^3B_3
\end{equation*}
显然,这既可以写为$A^iB_i$,也可以写为$A_iB^i$,结果是相同的。一般来说,一对傀指标中的上标和下标总是可以交换的。

积$A^iB_i$是一个{\bf 四维标量}——它在四维坐标系的转动下是不变的。验证这点首先需明确四维矢量协变分量需满足的变换法则,即
\begin{align*}
	A_0 = A^0 & = \frac{A'^0+\dfrac{V}{c}A'^1}{\sqrt{1-\dfrac{V^2}{c^2}}} = \frac{A'_0-\dfrac{V}{c}A'_1}{\sqrt{1-\dfrac{V^2}{c^2}}} \\
	A_1 = -A^1 & = -\frac{A'^1+\dfrac{V}{c}A'^0}{\sqrt{1-\dfrac{V^2}{c^2}}} = \frac{A'_1-\dfrac{V}{c}A'_0}{\sqrt{1-\dfrac{V^2}{c^2}}} \\
	A_2 = -A^2 & = -A'^2 = A'_2 \\
	A_3 = -A^3 & = -A'^3 = A'_3
\end{align*}
由此可得协变分量在Lorentz变换下的变换法则
\begin{equation}
\begin{cases}
	\displaystyle A_0 = \frac{A'_0-\dfrac{V}{c}A'_1}{\sqrt{1-\dfrac{V^2}{c^2}}} \\[1.5ex]
	\displaystyle A_1 = \frac{A'_1-\dfrac{V}{c}A'_0}{\sqrt{1-\dfrac{V^2}{c^2}}} \\[1.5ex]
	\displaystyle A_2 = A'_2 \\[1.5ex]
	\displaystyle A_3 = A'_3
\end{cases}
\end{equation}
据此,便可直接验证
\begin{align*}
	A^iB_i & = A^0B_0+A^1B_1+A^2B_2+A^3B_3 \\
	& = \frac{A'^0+\dfrac{V}{c}A'^1}{\sqrt{1-\dfrac{V^2}{c^2}}} \frac{B'_0-\dfrac{V}{c}B'_1}{\sqrt{1-\dfrac{V^2}{c^2}}} + \frac{A'^1+\dfrac{V}{c}A'^0}{\sqrt{1-\dfrac{V^2}{c^2}}} \frac{B'_1-\dfrac{V}{c}B'_0}{\sqrt{1-\dfrac{V^2}{c^2}}} + A'^2B'_2 + A'_3B'_3 \\
	& = A'^0B'_0+A'^1B'_1+A'^2B'_2+A'_3B'_3
\end{align*}

分量$A^0$称为四维矢量的{\bf 时间分量},$A^1,A^2,A^3$称为四维矢量的{\bf 空间分量}(与四维径向矢量类比)。四维矢量的平方可以为正、负或零,相应的四维矢量分别称为{\bf 类时矢量}、{\bf 类空矢量}和{\bf 类光矢量}\footnote{类光矢量也称为各项同性矢量。}。

在纯空间转动(即不影响时间轴的变换)下,四维矢量$A^i$的三个空间分量构成一个三维矢量$\mbf{A}$。该四维矢量的时间分量(在这些变换下)是一个三维标量。为了列举四维矢量的分量,我们常将其写为
\begin{equation*}
	A^i = (A^0,\mbf{A})
\end{equation*}
同一四维矢量的协变分量为$A_i=(A^0,-\mbf{A})$。该四维矢量的平方为
\begin{equation*}
	A^iA_i = (A^0)^2 - \mbf{A}^2
\end{equation*}
因此,对于四维径向矢量:
\begin{equation*}
	x^i = (ct,\mbf{r}),\quad x_i = (ct,-\mbf{r}),\quad x^ix_i = c^2t^2- \mbf{r}^2
\end{equation*}
对于三维矢量,没有必要区分逆变和协变分量。只要能够做到这一点而不至引起混淆,我们将用希腊字母作为下标把这些分量记为$A_\alpha\,(\alpha=x,y,z)$。对于重复的希腊字母指针我们将假设其遍历$x,y,z$求和,例如
\begin{equation*}
	\mbf{A} \cdot \mbf{B} = A_\alpha B_\alpha
\end{equation*}

\subsection{四维张量}

二阶{\bf 四维张量}是16个量$A^{ik}$的集合,它在坐标变换下像两个四维矢量分量的积那样变换。类似地,可以定义更高阶的四维张量。一个二阶张量的分量可以写为三种形式:协变分量$A_{ik}$、逆变分量$A^{ik}$和混合分量$A^i{}_k$及$A_i{}^k$。不同类型分量之间的联系由一下通则决定:升或降一个空间指标($1,2,3$)改变分量的正负号,而升或降时间指标($0$)则不变号。因此有
\begin{align*}
	& A_{00} = A^{00},\quad A_{01} = -A^{01},\quad A_{11} = A^{11},\cdots \\
	& A^0{}_0 = A^{00},\quad A_0{}^1 = A^{01},\quad A^0{}_1 = -A^{01},\quad A^1{}_1 = -A^{11},\quad 
\end{align*}
在纯空间变换下,9个量$A^{11},A^{12},\cdots$构成一个三维张量。三个分量$A^{01},A^{02},A^{03}$和三个分量$A^{10},A^{20},A^{30}$构成三维矢量,而分量$A^{00}$是一个三维标量。

如果$A^{ik} = A^{ki}$,张量$A^{ik}$称为{\bf 对称的};如果$A^{ik} = -A^{ki}$,张量$A^{ik}$称为{\bf 反对称的}。在反对称张量中,所有对角分量(即分量$A^{00},A^{11},A^{22},A^{33}$)都是零。对于一个对称张量$A^{ik}$,混合分量$A^i{}_k$和$A_k{}^i$相等,在这样的情形下,我们把一个指标置于另一个上方,简单地记为$A^i{}_k$。

在每个张量方程中,等号两边所含的自由指标(区别于傀指针)必须字母相同且位置相同(即上或下)。张量方程中的自由指标可以上移或下移,但必须对方程中所有的项同时进行。让不同张量的协变和逆变分量相等是“非法的”,这样的方程即便碰巧在特定参考系中成立,在变换到另一个参考系时也会失效(即其不具有“协变性”)。

通过对张量$A^{ik}$的分量求和可以形成一个标量
\begin{equation*}
	A^i{}_i = A^0{}_0 + A^1{}_1 + A^2{}_2 + A^3{}_3
\end{equation*}
这个和称为{\bf 张量的迹},求得它的运算称为{\bf 缩并}。缩并任何一对指标会使张量的阶减去$2$。例如二阶张量$A^iB_k$缩并后$A^iB_i$为四维标量,而四阶张量$A^i{}_{klm}$的缩并$A^i{}_{kli}$是一个二阶张量,等等。

单位四维张量$\delta^i_k$满足如下条件:对于任意四维矢量$A^i$,有
\begin{equation}
	\delta^i_k A^k = A^i
\end{equation}
因此,这个张量的分量显然是
\begin{equation}
	\delta^i_k = \begin{cases} 1,& \text{当}\,i=k \\ 0, &\text{当}\,i\neq k \end{cases}
\end{equation}
它的迹是$\delta^i_i = 4$。通过在$\delta^i_k$中升一个指标或者降一个指标,我们就得到逆变张量$g^{ik}$或协变张量$g_{ik}$,称之为{\bf 度规张量}。张量$g^{ik}$和$g_{ik}$具有相同的分量,可以用矩阵表示为
\begin{equation}
	\begin{pmatrix} g^{ik} \end{pmatrix} = \begin{pmatrix} g_{ik} \end{pmatrix} = \begin{pmatrix} 1 & 0 & 0 & 0 \\ 0 & -1 & 0 & 0 \\ 0 & 0 & -1 & 0 \\ 0 & 0 & 0 & -1 \end{pmatrix}
\end{equation}
由此,显然有
\begin{equation}
	g_{ik} A^k = A_i,\quad g^{ik} A_k = A^i
\end{equation}
两个四维矢量的标积因而可以写成形式
\begin{equation}
	A^i A_i = g_{ik}A^iA^k = g^{ik}A_iA_k
\end{equation}

张量$\delta^i_k$、$g_{ik}$和$g^{ik}$的特别之处在于,它们的分量在所有坐标系中都相同。四阶的{\bf 全反对称单位张量}$e^{iklm}$具有同样的性质。这个张量的分量在交换任一对指标时变号,其非零分量为$\pm 1$。从反对称性可知,有两个指标相同的所有分量均为零,所以仅有的非零分量是那些所有4个指标都不同者。我们令
\begin{equation}
	e^{0123} = 1
\end{equation}
于是,所有其他的非零分量$e^{iklm}$等于$1$或$-1$,依$i,k,l,m$这几个数能经偶数还是奇数次换位排成$0,1,2,3$而定。由于
\begin{equation*}
	e_{0123} = (-1)^3e^{0123} = -1
\end{equation*}
以及$e^{iklm}$的非零分量数为$4!=24$,可得
\begin{equation}
	e^{iklm}e_{iklm} = 24(-1)^3(e^{iklm})^2 = -24
\end{equation}

对于坐标系的转动而言,$e^{iklm}$诸量的特性与张量分量的特性相同,但是如果我们改变1个或3个坐目标正负号,分量$e^{iklm}$并不改变,因为按定义,它们在所有坐标系中都相同,而张量的分量在这种情况下是应当变号的。所以,严格地说,$e^{iklm}$并不是张量,而是一个{\bf 赝张量}。任意阶的赝张量,特别是赝标量,在所有的坐标变换下都具有张量的性质,只有那些不能归结为转动的变换,即反射(不能归结为转动的坐标正负号改变)是例外。

乘积$e^{iklm}e^{prst}$构成一个8阶四维张量,它是一个真正的张量,通过缩并其一对或多对指标可以得到6阶、4阶和2阶张量。所有这些张量在所有坐标系中具有相同的形式。所以,它们的分量必须表示为单位张量$\delta^i_k$(其分量在所有坐标系中都相同的唯一真张量)分量乘积的组合。%从指标排列所必须具有的对称性出发,这些组合是不难求得的。

记$\sigma$为一个4阶{\bf 置换},其作用在$0,1,2,3$中的某数将得到其重排之后所对应的数。由于$e^{iklm}$的值只有在其指标互不相同时才非零,因此有
\begin{equation*}
	i,k,l,m = \sigma(1),\sigma(2),\sigma(3),\sigma(4)
\end{equation*}
由此即有
\begin{equation*}
	e^{iklm}e_{prst} = e^{\sigma(1)\sigma(2)\sigma(3)\sigma(4)} e_{\gamma(1)\gamma(2)\gamma(3)\gamma(4)} = -\mathrm{sgn}\,\sigma \mathrm{sgn}\,\gamma
\end{equation*}
此处$\mathrm{sgn}\,\sigma$表示置换$\sigma$的{\bf 符号},如果置换后的元素经过偶数次交换可以恢复原本顺序,则该置换符号值为$1$,否则为$-1$。根据全反对称单位张量的定义,即有$e^{\sigma(1)\sigma(2)\sigma(3)\sigma(4)} = \mathrm{sgn}\,\sigma$。考虑到指标排列所必须具有的对称性,可有
\begin{align*}
	e^{iklm}e_{prst} & = e^{\sigma(1)\sigma(2)\sigma(3)\sigma(4)} e_{\gamma(1)\gamma(2)\gamma(3)\gamma(4)} = -\mathrm{sgn}\,\sigma \mathrm{sgn}\,\gamma = -\mathrm{sgn}\,\sigma \mathrm{sgn}\,\gamma \begin{vmatrix} 1 & 0 & 0 & 0 \\ 0 & 1 & 0 & 0 \\ 0 & 0 & 1 & 0 \\ 0 & 0 & 0 & 1 \end{vmatrix} \\
	& = -\mathrm{sgn}\,\sigma \mathrm{sgn}\,\gamma \begin{vmatrix} \delta^0_0 & \delta^0_1 & \delta^0_2 & \delta^0_3 \\ \delta^1_0 & \delta^1_1 & \delta^1_2 & \delta^1_3 \\ \delta^1_0 & \delta^2_1 & \delta^2_2 & \delta^2_3 \\ \delta^3_0 & \delta^3_1 & \delta^3_2 & \delta^3_3 \end{vmatrix} = - \begin{vmatrix} \delta^{\sigma(0)}_{\gamma(0)} & \delta^{\sigma(0)}_{\gamma(1)} & \delta^{\sigma(0)}_{\gamma(2)} & \delta^{\sigma(0)}_{\gamma(3)} \\ \delta^{\sigma(1)}_{\gamma(0)} & \delta^{\sigma(1)}_{\gamma(1)} & \delta^{\sigma(1)}_{\gamma(2)} & \delta^{\sigma(1)}_{\gamma(3)} \\ \delta^{\sigma(2)}_{\gamma(0)} & \delta^{\sigma(2)}_{\gamma(1)} & \delta^{\sigma(2)}_{\gamma(2)} & \delta^{\sigma(2)}_{\gamma(3)} \\ \delta^{\sigma(3)}_{\gamma(0)} & \delta^{\sigma(3)}_{\gamma(1)} & \delta^{\sigma(3)}_{\gamma(2)} & \delta^{\sigma(3)}_{\gamma(3)} \end{vmatrix} \\
	& = -\begin{vmatrix} \delta^i_p & \delta^i_r & \delta^i_s & \delta^i_t \\ \delta^k_p & \delta^k_r & \delta^k_s & \delta^k_t \\ \delta^l_p & \delta^l_r & \delta^l_s & \delta^l_t \\ \delta^m_p & \delta^m_r & \delta^m_s & \delta^m_t \end{vmatrix}
\end{align*}
由此,可有
\begin{align*}
	e^{iklm}e_{prst} & = \delta^i_t \begin{vmatrix}\delta^k_p & \delta^k_r & \delta^k_s \\ \delta^l_p & \delta^l_r & \delta^l_s \\ \delta^m_p & \delta^m_r & \delta^m_s \end{vmatrix} - \delta^k_t \begin{vmatrix} \delta^i_p & \delta^i_r & \delta^i_s \\ \delta^l_p & \delta^l_r & \delta^l_s \\ \delta^m_p & \delta^m_r & \delta^m_s \end{vmatrix} + \delta^l_t \begin{vmatrix} \delta^i_p & \delta^i_r & \delta^i_s \\ \delta^k_p & \delta^k_r & \delta^k_s \\ \delta^m_p & \delta^m_r & \delta^m_s \end{vmatrix} - \delta^m_t \begin{vmatrix} \delta^i_p & \delta^i_r & \delta^i_s \\ \delta^k_p & \delta^k_r & \delta^k_s \\ \delta^l_p & \delta^l_r & \delta^l_s\end{vmatrix}
\end{align*}
缩并其中的$m$和$t$可得
\begin{align*}
	e^{iklm}e_{prsm} & = \begin{vmatrix}\delta^k_p & \delta^k_r & \delta^k_s \\ \delta^l_p & \delta^l_r & \delta^l_s \\ \delta^i_p & \delta^i_r & \delta^i_s \end{vmatrix} - \begin{vmatrix} \delta^i_p & \delta^i_r & \delta^i_s \\ \delta^l_p & \delta^l_r & \delta^l_s \\ \delta^k_p & \delta^k_r & \delta^k_s \end{vmatrix} + \begin{vmatrix} \delta^i_p & \delta^i_r & \delta^i_s \\ \delta^k_p & \delta^k_r & \delta^k_s \\ \delta^l_p & \delta^l_r & \delta^l_s \end{vmatrix} - 4\begin{vmatrix} \delta^i_p & \delta^i_r & \delta^i_s \\ \delta^k_p & \delta^k_r & \delta^k_s \\ \delta^l_p & \delta^l_r & \delta^l_s\end{vmatrix} \\ 
	& = -\begin{vmatrix} \delta^i_p & \delta^i_r & \delta^i_s \\ \delta^k_p & \delta^k_r & \delta^k_s \\ \delta^l_p & \delta^l_r & \delta^l_s\end{vmatrix}
\end{align*}
由此,可有
\begin{align*}
	e^{iklm}e_{prsm} & = -\delta^i_s \begin{vmatrix} \delta^k_p & \delta^k_r \\ \delta^l_p & \delta^l_r \end{vmatrix} + \delta^k_s \begin{vmatrix} \delta^i_p & \delta^i_r \\ \delta^l_p & \delta^l_r \end{vmatrix} - \delta^l_s \begin{vmatrix} \delta^i_p & \delta^i_r \\ \delta^k_p & \delta^k_r \end{vmatrix}
\end{align*}
缩并其中的$l$和$s$可得
\begin{align*}
	e^{iklm}e_{prlm} & = -\begin{vmatrix} \delta^k_p & \delta^k_r \\ \delta^i_p & \delta^i_r \end{vmatrix} + \begin{vmatrix} \delta^i_p & \delta^i_r \\ \delta^k_p & \delta^k_r \end{vmatrix} - 4 \begin{vmatrix} \delta^i_p & \delta^i_r \\ \delta^k_p & \delta^k_r \end{vmatrix} = -2 \begin{vmatrix} \delta^i_p & \delta^i_r \\ \delta^k_p & \delta^k_r \end{vmatrix} = -2(\delta^i_p\delta^k_r - \delta^i_r\delta^k_p)
\end{align*}
再缩并其中的$k$和$r$可得
\begin{equation*}
	e^{iklm}e_{pklm} = -2(4\delta^i_p - \delta^i_p) = -6\delta^i_p
\end{equation*}
综上,可有如下关系式
\begin{align}
	e^{iklm}e_{prst} & = -\begin{vmatrix} \delta^i_p & \delta^i_r & \delta^i_s & \delta^i_t \\ \delta^k_p & \delta^k_r & \delta^k_s & \delta^k_t \\ \delta^l_p & \delta^l_r & \delta^l_s & \delta^l_t \\ \delta^m_p & \delta^m_r & \delta^m_s & \delta^m_t \end{vmatrix} \\
	e^{iklm}e_{prsm} & = -\begin{vmatrix} \delta^i_p & \delta^i_r & \delta^i_s \\ \delta^k_p & \delta^k_r & \delta^k_s \\ \delta^l_p & \delta^l_r & \delta^l_s \end{vmatrix} \\
	e^{iklm}e_{prlm} & = -2(\delta^i_p\delta^k_r - \delta^i_r\delta^k_p) \\
	e^{iklm}e_{pklm} & = -6\delta^i_p
\end{align}

如果$A^{ik}$是一个反对称张量,则张量$A^{ik}$和赝张量$A^{*ik} = \dfrac12 e^{iklm}A_{lm}$称为彼此{\bf 对偶}。类似地,$e^{iklm}A_m$是一个与矢量$A^i$对偶的三阶反对称赝张量。对偶张量的乘积$A^{ik}A^*_{ik}$显然是一个赝标量。

联系这里的讨论,三阶全反对称单位赝张量$e_{\alpha\beta\gamma}$是这样一些量的几何,它们在任何一对指标换位时变号。因此在$e_{\alpha\beta\gamma}$的分量中,只有那些具有3个不同指标者才非零。我们令$e_{123}$,于是其他分量等于$1$或$-1$则根据$\alpha,\beta,\gamma$这个序列能经偶数还是奇数次换位排成$x,y,z$的顺序而定(即等于该置换的符号)。乘积$e_{\alpha\beta\gamma}e_{\lambda\mu\nu}$构成一个6阶真三维张量,因而可以表示为单位三维张量$\delta_{\alpha\beta}$分量乘积的组合。对于三阶全反对称单位赝张量也有类似的如下关系式:
\begin{align}
	e_{\alpha\beta\gamma}e_{\lambda\mu\nu} & = \begin{vmatrix} \delta_{\alpha\lambda} & \delta_{\alpha\mu} & \delta_{\alpha\nu} \\ \delta_{\beta\lambda} & \delta_{\beta\mu} & \delta_{\beta\nu} \\ \delta_{\gamma\lambda} & \delta_{\gamma\mu} & \delta_{\gamma\nu} \end{vmatrix} \\
	e_{\alpha\beta\gamma}e_{\lambda\mu\gamma} & = \delta_{\alpha\lambda}\delta_{\beta\mu} - \delta_{\alpha\mu}\delta_{\beta\lambda} \\
	e_{\alpha\beta\gamma}e_{\lambda\beta\gamma} & = 2\delta_{\alpha\lambda} \\
	e_{\alpha\beta\gamma}e_{\alpha\beta\gamma} & = 6
\end{align}

在坐标系的反射(即所有坐标变号)下,一个普通矢量也变号。这样的矢量称为{\bf 极矢量}。一个矢量若能写成两个极矢量的矢积,则其分量在反演下不变号,这样的矢量称为{\bf 轴矢量}。一个极矢量和一个轴矢量的标积并不是一个真标量,而是一个赝标量,它在坐标反演下变号。轴矢量是赝矢量,对偶于某反对称张量。因此,如果$\mbf{C} = \mbf{A} \times \mbf{B}$,那么
\begin{equation*}
	C_\alpha = \frac12 e_{\alpha\beta\gamma}C_{\beta\gamma},\quad \text{其中}\,C_{\beta\gamma} = A_\beta B_\gamma - A_\gamma B_\beta
\end{equation*}

现在回到四维张量上来。反对称张量$A^{ik}$的空间分量($i,k=1,2,3$)对于纯空间变换构成一个三维反对称张量。根据上面的论述,其分量可以用一个三维轴矢量的分量来表示。对于同样的变换,分量$A^{01},A^{02},A^{03}$构成一个三维极矢量。因此一个反对称四维张量可以写成矩阵
\begin{equation}
	\begin{pmatrix} A^{ik} \end{pmatrix} = \begin{pmatrix} 0 & p_x & p_y & p_z \\ -p_x & 0 & -a_z & a_y \\ -p_y & a_z & 0 & -a_x \\ -p_z & -a_y & a_x & 0 \end{pmatrix}
\end{equation}
这里对于空间变换,$\mbf{p}$和$\mbf{a}$分别为极矢量和轴矢量。在列出反对称四维张量的分量时,可以将它们写成形式
\begin{equation*}
	A^{ik} = (\mbf{p},\mbf{a})
\end{equation*}
其协变分量则为
\begin{equation*}
	A_{ik} = (-\mbf{p},\mbf{a})
\end{equation*}

\subsection{四维张量分析}

标量$\phi$的四维梯度是四维矢量
\begin{equation*}
	\frac{\pl \phi}{\pl x^i} = \left(\frac{1}{c}\frac{\pl \phi}{\pl t},\bnb \phi\right)
\end{equation*}
这个导数是该四维梯度的协变分量,此标量的微分
\begin{equation*}
	\mathrm{d}\phi = \frac{\pl \phi}{\pl x^i} \mathrm{d}x^i
\end{equation*}
也是一个标量,其具有两个四维矢量标积的形式,由此也可看出$\dfrac{\pl \phi}{\pl x^i}$是该四维矢量的协变分量。

一般说来,对于坐标$x^i$微分的算符$\dfrac{\pl}{\pl x^i}$应当看成是该算符四维矢量的协变分量。如果我们对于“协变坐标”$x_i$进行微分,则导数
\begin{equation*}
	\frac{\pl \phi}{\pl x_i} = g^{ik} \frac{\pl \phi}{\pl x^k} = \left(\frac{1}{c}\frac{\pl \phi}{\pl t},-\bnb \phi\right)
\end{equation*}
构成一个四维矢量的逆变分量。对于坐目标偏导数可以采用缩写
\begin{equation*}
	\pl_i = \frac{\pl}{\pl x^i},\quad \pl^i = \frac{\pl}{\pl x_i}
\end{equation*}
以这种形式书写微分算符时,由它们构成的量协变或逆变性质是一目了然的。

在三维空间中,可以沿体积、曲面或曲线进行积分。在四维空间中积分有四种类型:
\begin{enumerate}
\item 沿四维空间中的一条曲线的积分。积分元就是线元,即四维矢量$\mathrm{d}x^i$。

\item 沿四维空间中的一个(二维)曲面的积分。在三维空间中,由两个矢量$\mathrm{d}\mbf{r}$和$\mathrm{d}\mbf{r}'$构成的平行四边形的面积在坐标平面$x_\alpha x_\beta$上的投影是$\mathrm{d}x_\alpha \mathrm{d}x'_\beta - \mathrm{d}x_\beta \mathrm{d}x'_\alpha$。类似地,在四维空间中,无限小面元由二阶反对称张量$\mathrm{d}f^{ik}=\mathrm{d}x^i\mathrm{d}x'^k - \mathrm{d}x^k\mathrm{d}x'^i$给定,其分量为该面元在坐标平面上的投影。在三维空间中,我们不用张量$\mathrm{d}f_{\alpha\beta}$而用与之对偶的矢量$\mathrm{d}f_\alpha$来表示该面元$\mathrm{d}f_\alpha = \dfrac12 e_{\alpha\beta\gamma} \mathrm{d}f_{\beta\gamma}$。在几何上,这是一个与面元垂直的矢量,其绝对值等于面元的面积。在四维空间中无法构造这样的矢量,但可以构造与张量$\mathrm{d}f^{ik}$对偶的张量$\mathrm{d}f^{*ik}$,满足
\begin{equation}
	\mathrm{d}f^{*ik} = \frac12 e^{iklm}\mathrm{d}f_{lm}
\end{equation}
在几何上,它描述了一个“垂直”于面元$\mathrm{d}f_{ik}$而且与之大小相等的面元,即有$\mathrm{d}f^{*ik}\mathrm{d}f_{ik} = 0$。

\item 沿一个超曲面,即沿一个三维流形的积分。在三维空间中,由3个矢量张成的平行六面体的体积等于由这些矢量的分量构成的3阶行列式。类似地,可以得到由3个四维矢量$\mathrm{d}x^i,\mathrm{d}x'^i,\mathrm{d}x”^i$张成的平行六面体体积(即该超曲面的“面积”),它们由如下行列式给出
\begin{equation*}
	\mathrm{d}S^{ikl} = \begin{vmatrix} \mathrm{d}x^i & \mathrm{d}x'^i & \mathrm{d}x”^i \\ \mathrm{d}x^k & \mathrm{d}x'^k & \mathrm{d}x”^k \\ \mathrm{d}x^l & \mathrm{d}x'^l & \mathrm{d}x”^l \end{vmatrix}
\end{equation*}
这构成一个三阶张量,对所有3个指标都是反对称的,在此情形下,使用与张量$\mathrm{d}S^{ikl}$对偶的四维矢量$\mathrm{d}S^i$更方便:
\begin{equation}
	\mathrm{d}S^i = -\frac16 e^{iklm} \mathrm{d}S_{klm},\quad \mathrm{d}S_{klm} = e_{nklm}\mathrm{d}S^n
\end{equation}
这里
\begin{equation*}
	\mathrm{d}S^0 = \mathrm{d}S^{123},\quad \mathrm{d}S^1 = \mathrm{d}S^{023},\quad \mathrm{d}S^2 = -\mathrm{d}S^{013},\quad \mathrm{d}S^3 = \mathrm{d}S^{012}
\end{equation*}
在几何上,$\mathrm{d}S^i$是一个四维矢量,数值上等于超曲面元的“面积”,并与该面元垂直。特别是,$\mathrm{d}S^0 = \mathrm{d}x\mathrm{d}y\mathrm{d}z$,这就是三维体积元$\mathrm{d}V$,即超曲面在超平面$x^0=\text{const}$上的投影。

\item 沿一个四维体积的积分,积分元是标量
\begin{equation}
	\mathrm{d}\varOmega = \mathrm{d}x^0\mathrm{d}x^1\mathrm{d}x^2\mathrm{d}x^3 = ct\mathrm{d}t\mathrm{d}V
\end{equation}
\end{enumerate}

类似三维矢量分析中的Gauss定理和Stokes定理,有些定理使我们能做四维积分的变换。

沿一闭合超曲面的积分可以变换到沿其包含的四维体积的积分,用算符
\begin{equation}
	\mathrm{d}S_i \to \mathrm{d}\varOmega \frac{\pl}{\pl x^i} = \mathrm{d}\varOmega \pl_i
\end{equation}
代替积分元$\mathrm{d}S_i$。例如,对于矢量$A_i$的积分,可有
\begin{equation}
	\oint A^i \mathrm{d}S_i = \int \pl_iA^i \mathrm{d}\varOmega
\end{equation}
这个公式是Gauss定理的推广。

沿一个二维曲面的积分可以变换为“包含”它的超曲面的积分,用算符
\begin{equation}
	\mathrm{d}f^*_{ik} \to \mathrm{d}S_i \frac{\pl}{\pl x^k} - \mathrm{d}S_k \frac{\pl}{\pl x^i} = \mathrm{d}S_i \pl_k - \mathrm{d}S_k \pl_i
\end{equation}
代替积分元$\mathrm{d}f^*_{ik}$。例如,对于反对称张量$A^{ik}$的积分,可有
\begin{equation}
	\frac12 \oint A^{ik}\mathrm{d}f^*_{ik} = \frac12 \int \left(\mathrm{d}S_i \pl_kA^{ik} - \mathrm{d}S_k \pl_iA^{ik}\right) = \int \mathrm{d}S_i \pl_kA^{ik}
\end{equation}

沿一条四维闭合曲线的积分可以通过代换
\begin{equation}
	\mathrm{d}x^i \to \mathrm{d}f^{ki}\frac{\pl}{\pl x^k} = \mathrm{d}f^{ki} \pl_k
\end{equation}
变换为“包含”它的曲面的积分。因此,对于一个矢量的积分,可有
\begin{equation}
	\oint A_i \mathrm{d}x^i = \int \mathrm{d}f^{ki}\pl_kA_i = \frac12 \int \mathrm{d}f^{ki} \left(\pl_iA_k - \pl_kA_i\right)
\end{equation}
这是Stokes定理的推广。

\section{四维速度}

由普通的三维速度矢量,我们可以构造一个四维矢量。一个粒子的{\bf 四维速度}(四速度)是矢量
\begin{equation}
	u^i = \frac{\mathrm{d} x^i}{\mathrm{d} s}
\end{equation}
为了求出它的分量,根据式\eqref{chapter8:无穷小固有时}可有
\begin{equation}
	\mathrm{d}s = c\mathrm{d}t\sqrt{1-\frac{v^2}{c^2}}
\end{equation}
其中$v$为粒子的普通三维速度。因此
\begin{equation*}
	u^1 = \frac{\mathrm{d}x^1}{\mathrm{d}s} = \frac{\mathrm{d}x}{c\mathrm{d}t\sqrt{1-\dfrac{v^2}{c^2}}} = \frac{v_x}{c \sqrt{1-\dfrac{v^2}{c^2}}}
\end{equation*}
由此可得
\begin{equation}
	u^i = \left(\frac{1}{\sqrt{1-\dfrac{v^2}{c^2}}},\frac{\mbf{v}}{c\sqrt{1-\dfrac{v^2}{c^2}}}\right)
\end{equation}
注意到,四维速度是一个无量纲量。

四维速度的分量之间并不彼此独立,由于$\mathrm{d}x^i\mathrm{d}x_i = \mathrm{d}s^2$,因此有
\begin{equation}
	u^iu_i = 1
	\label{chapter8:四维速度分量之间的关系}
\end{equation}
因此,四维速度的几何意义是与粒子世界线相切的一个四维单位矢量。

与四维速度的定义类似,二阶导数
\begin{equation*}
	w^i = \frac{\mathrm{d}^2 x^i}{\mathrm{d} s^2} = \frac{\mathrm{d} u^i}{\mathrm{d} s}
\end{equation*}
可以称为{\bf 四维加速度}。微分式\eqref{chapter8:四维速度分量之间的关系}可得
\begin{equation}
	u_i w^i = 0
\end{equation}
即四维速度矢量与四维加速度矢量是相互正交的。

% \begin{example}
% 确定相对论的匀加速运动,即在固有参考系中(每个时刻)加速度$w$保持不变的直线运动。
% \end{example}
% \begin{solution}

% \end{solution}


\chapter{相对论力学}

\section{Hamilton原理}%对应Landau的最小作用量原理

现在我们从Hamilton原理开始,研究实物粒子的运动规律。Hamilton原理指出:对于每一个力学体系,有一个叫做{\bf 作用量}的积分$S$存在,这个积分对于实际运动有最小值\footnote{严格的讲,应该是驻值。},即它的变分$\delta S$为零。

一个自由实物粒子的作用量积分必然与参考系的选择无关,即,它必须对于Lorentz变换保持不变。因此,它必须为一个标量函数。另外显然其被积函数必须是一个微分$1-$形式。对于一个自由粒子,所能构造出的唯一的这种标量,仅仅是间隔$\mathd s$,或者固有时$\mathd \tau$,或者它们乘以一个常数$\alpha\mathd s$。这样一来,对于一个自由粒子,作用量的积分必须取下面的形式:
\begin{equation*}
	S = -\alpha \int_{\tau_1}^{\tau_2} \mathd s
\end{equation*}
其中$\ds \int_a^b$表示沿着粒子在两个特定事件间的世界线的积分,这两个事件就是粒子在$t_1$时刻到达初位置和在$t_2$时刻到达末位置,也就是说$\ds \int_a^b$是沿着两个世界点之间的世界线的积分;而$\alpha$为表征该粒子的一个常数。在第\ref{chapter1:section固有时}节中,积分$\ds\int_a^b \mathd s$沿着一条直的世界线的值最大;沿着一条弯曲的世界线,可以使得积分值为任意小\footnote{因为这些积分的值都是负值。}

作用量可以表示为对时间的积分$\ds S = \int_{t_1}^{t_2} L\mathd t$,其中$L$即为这个力学体系的{\bf Lagrange函数}。利用式\eqref{chapter1:无穷小固有时},可得
\begin{equation*}
	S = -\int_{t_1}^{t_2} \alpha c\sqrt{1-\frac{v^2}{c^2}} \mathd t
\end{equation*}
其中$v$为实物粒子的速度,即实物粒子的Lagrange函数为
\begin{equation}
	L = -\alpha c\sqrt{1-\frac{v^2}{c^2}} 
\end{equation}

上面已经提到,$\alpha$是表征该粒子的一个量。在经典力学中,这个量就是该粒子的质量$m$。当我们取$c\to +\infty$的极限时,$L$的表达式应该过渡到它的经典表达式$L=\dfrac12 mv^2$。将$L$按$\dfrac{v}{c}$展开至$\dfrac{v^2}{c^2}$项可得
\begin{equation*}
	L = -\alpha c\sqrt{1-\dfrac{v^2}{c^2}} = -\alpha c+\frac{\alpha v^2}{2c}
\end{equation*}
Lagrange函数中的常数项对运动方程没有影响,可以略去。略去常数项$-\alpha c$之后,与经典力学中自由粒子的Lagrange函数$L=\dfrac12 mv^2$比较,可得$\alpha=mc$。

所以,自由实物粒子的作用量是
\begin{equation}
	S = -mc\int_a^b \mathd s
	\label{chapter2:自由实物粒子的作用量}
\end{equation}
而Lagrange函数是
\begin{equation}
	L = -mc^2\sqrt{1-\dfrac{v^2}{c^2}}
	\label{chapter2:自由实物粒子的Lagrange函数}
\end{equation}

\section{能量与动量}

类似于经典Hamilton力学中的做法,将矢量$\mbf{p} = \dfrac{\pl L}{\pl \mbf{v}}$\footnote{$\mbf{p} = \dfrac{\pl L}{\pl \mbf{v}}$表示$\ds p^i = \frac{\pl L}{\pl v^i}$。}称为该粒子的{\bf 动量}。利用式\eqref{chapter2:自由实物粒子的Lagrange函数}可得
\begin{equation}
	\mbf{p} = \frac{m\mbf{v}}{\sqrt{1-\dfrac{v^2}{c^2}}}
	\label{chapter2:自由实物粒子的动量}
\end{equation}
对于很小的速度($v\ll c$),或者$c\to +\infty$的极限情形下,式\eqref{chapter2:自由实物粒子的动量}就变为经典的动量公式$\mbf{p} = m\mbf{v}$。而当$v=c$时,实物粒子的动量就变为无穷大。

动量对时间的导数就是作用于粒子的力。将速度矢量记作$\mbf{v}=v\mbf{e}$,其中$\mbf{e}$是与$\mbf{v}$通向的方向矢量,则有
\begin{equation}
	\frac{\mathd \mbf{p}}{\mathd t} = \frac{mv}{\sqrt{1-\dfrac{v^2}{c^2}}} \frac{\mathd \mbf{e}}{\mathd t} + \frac{m\mbf{e}}{\left(1-\dfrac{v^2}{c^2}\right)^{\frac32}} \frac{\mathd v}{\mathd t}
\end{equation}
当粒子的速度大小不变仅有方向改变时,即力与速度方向垂直时,有
\begin{equation*}
	\frac{\mathd \mbf{p}}{\mathd t} = \frac{mv}{\sqrt{1-\dfrac{v^2}{c^2}}} \frac{\mathd \mbf{e}}{\mathd t}
\end{equation*}
当粒子的速度方向不变仅有大小改变时,即力与速度方向平行时,有
\begin{equation*}
	\frac{\mathd \mbf{p}}{\mathd t} = \frac{m\mbf{e}}{\left(1-\dfrac{v^2}{c^2}\right)^{\frac32}} \frac{\mathd v}{\mathd t}
\end{equation*}
即力与加速度之比已经不再是常数,Newton第二定律失效。

粒子的{\bf 能量}$\E$\footnote{此处用$\E$而不用$E$来表示能量是为了避免与后面涉及的电场强度$\mbf{E}$相混淆。}可以定义为
\begin{equation}
	\E = \mbf{p}\cdot \mbf{v} - L
	\label{chapter2:粒子能量的定义}
\end{equation}
将式\eqref{chapter2:自由实物粒子的动量}和式\eqref{chapter2:自由实物粒子的Lagrange函数}代入可得
\begin{equation}
	\E = \frac{mc^2}{\sqrt{1-\dfrac{v^2}{c^2}}}
	\label{chapter2:自由实物粒子的能量}
\end{equation}
相对论力学与经典力学的一个重要差别即为,粒子的能量在$v=0$时并不为零,而是有一个有限值
\begin{equation}
	\E_0 = mc^2
	\label{chapter2:粒子的静能}
\end{equation}
这个量称为粒子的{\bf 静能}。当速度远小于光速时,将式\eqref{chapter2:自由实物粒子的能量}展开为$\dfrac{v^2}{c^2}$的幂级数,可以得到
\begin{equation*}
	\E \approx mc^2 + \frac12mv^2
\end{equation*}
即,扣除静能后,能量表达式\eqref{chapter2:自由实物粒子的能量}退化为经典结果。

虽然我们考虑的是一个自由粒子的情形,但是这些公式同样可以适用于许多粒子构成的复合物体。此时$m$即为该物体的总质量,$\mbf{v}$是指它整体的运动速度。在相对论力学中,任何封闭系统的能量都是一个完全确定的量,它总是正的并与该系统的质量直接相关。而在经典力学中,封闭系统的能量可以有一个任意常数的差别,而且既可以为正值,也可以为负值。

一个静止系统的能量,除了其组成粒子的静能外,还包括它们的动能和它们之间的相互作用能。因此,$mc^2$并不等于$\ds \sum m_\alpha c^2$(其中$m_\alpha$为各个粒子单独存在时的质量),也就是说$m$并不等于$\ds \sum m_\alpha$。在相对论力学中,质量守恒定律不再成立,复合物体的质量不再等于其各部分质量之和,而只有包含了粒子静能在内的能量守恒定律是成立的。

利用式\eqref{chapter2:自由实物粒子的动量}和式\eqref{chapter2:自由实物粒子的能量}消去粒子的速度,可得自由实物粒子能量与动量之间的关系
\begin{equation}
	\frac{\E^2}{c^2} = p^2+m^2c^2
	\label{chapter2:自由实物粒子能量与动量之间的关系}
\end{equation}
进而可获得粒子的Hamilton函数$\mathscr{H}$\footnote{此处用$\mathscr{H}$而不用$H$来表示Hamilton函数是为了避免与后面涉及的磁场强度$\mbf{H}$相混淆。}:
\begin{equation}
	\mathscr{H} = c\sqrt{p^2+m^2c^2}
	\label{chapter2:自由实物粒子的Hamilton函数}
\end{equation}
对于低速情况$\dfrac{p}{mc}\ll 1$,近似地有
\begin{equation*}
	\mathscr{H} \approx mc^2 + \frac{p^2}{2m}
\end{equation*}
扣除静能后,Hamilton函数也退化为经典结果。

利用式\eqref{chapter2:自由实物粒子的动量}和式\eqref{chapter2:自由实物粒子的能量}我们可以得到一个自由粒子的能量、动量和速度之间的关系为
\begin{equation}
	\mbf{p} = \frac{\E \mbf{v}}{c^2}
	\label{chapter2:自由实物粒子能量、动量和速度之间的关系}
\end{equation}

当$v=c$时,粒子的动量和能量都变为无穷大,因此,一个粒子如果其质量不为零,就不可能以光速运动。在相对论力学中,可以存在质量为零同时以光速运动的粒子(例如光子和中微子)。根据式\eqref{chapter2:自由实物粒子能量、动量和速度之间的关系},我们有
\begin{equation}
	p = \frac{\E}{c}
\end{equation}
同样的公式对于非零质量的粒子在{\bf 极端相对论}情况下{\bf 近似成立},此时粒子的能量将远远大于其静能$mc^2$。

下面来推导上面所得到所有关系式的四维形式。根据Hamilton原理
\begin{equation*}
	\delta S = -mc\delta\int_a^b \mathd s
\end{equation*}
由于$\mathd s = \sqrt{\mathd x^i\mathd x_i}$,故有
\begin{align}
	\delta S & = -mc\int_a^b \frac{\mathd x_i\delta \mathd x^i}{\mathd s} = -m\int_a^b u_i\mathd \delta x^i = -mu_i\delta x^i\bigg|_a^b + m\int_a^b \delta x^i \frac{\mathd u_i}{\mathd s}\mathd s \nonumber \\
	& = -mu_i\delta x^i\bigg|_a^b + \frac{m}{c}\int_a^b \delta x^i \frac{\mathd u_i}{\mathd \tau}\mathd s
	\label{chapter2.section2.1:作用量的变分}
\end{align}
由$\delta S = 0$和固定边界条件可得自由实物粒子的运动方程即为$\dfrac{\mathd u_i}{\mathd \tau} = 0$。

但现在我们需要将作用量的变分表示为坐标的函数,此时需将$a$点当做固定的,令粒子按照实际的轨道运动,考察$\delta S$与$(\delta x^i)_b$之间的关系。根据$b$点所取的任意性,将$(\delta x^i)_b$记作$\delta x^i$,则有
\begin{equation}
	\delta S = -mu_i\delta x^i
\end{equation}
由此,称四维矢量
\begin{equation}
	p_i = -\frac{\pl S}{\pl x^i} = mu_i
	\label{chapter2:四维动量矢量}
\end{equation}
为{\bf 四维动量矢量}。在经典力学中,导数$\dfrac{\pl S}{\pl x}, \dfrac{\pl S}{\pl y}, \dfrac{\pl S}{\pl z}$是粒子动量矢量$\mbf{p}$的三个分量,而导数$-\dfrac{\pl S}{\pl t}$是粒子的能量$\E$。因此,即有四维动量的协变分量为
\begin{equation}
	p_i = \left(\frac{\E}{c},-\mbf{p}\right)
	\label{chapter2:四维动量矢量的协变分量}
\end{equation}
而逆变分量则为
\begin{equation}
	p^i = \left(\frac{\E}{c},\mbf{p}\right)
	\label{chapter2:四维动量矢量的逆变分量}
\end{equation}

因此,在相对论力学中,能量与动量是同一个四维矢量的分量。故可直接由此获得能量与动量由一个惯性系变换到另一个惯性系时的变换公式,即
\begin{equation}
\begin{cases}
	\ds p_x = \frac{p'_x+\dfrac{V}{c^2}\E'}{\sqrt{1-\dfrac{V^2}{c^2}}} \\
	p_y = p'_y \\
	p_z = p'_z \\
	\ds \E = \frac{\E'+Vp'_x}{\sqrt{1-\dfrac{V^2}{c^2}}}
\end{cases}
\label{chapter2:惯性系能量动量变换关系式}
\end{equation}
式中$p_x, p_y, p_z$是三维矢量$\mbf{p}$的分量(即四维矢量$p^i$的逆变分量$p^1, p^2$和$p^3$)。

四维动量满足
\begin{equation}
	p^ip_i = m^2u^iu_i = m^2c^2
	\label{chapter2:四维动量的平方}
\end{equation}
将式\eqref{chapter2:四维动量矢量的协变分量}和式\eqref{chapter2:四维动量矢量的逆变分量}代入,即可得到关系式\eqref{chapter2:自由实物粒子能量与动量之间的关系}。

而类比于力的通常定义,四维力矢量可以定义为
\begin{equation}
	f^i = \frac{\mathd p^i}{\mathd \tau} = m\frac{\mathd u^i}{\mathd \tau}
	\label{chapter2:四维力矢量}
\end{equation}
其满足恒等式
\begin{equation}
	f^i u_i = 0
\end{equation}
这个四维矢量可以用通常的三维力矢量$\mbf{F} = \dfrac{\mathd \mbf{p}}{\mathd t}$表示为
\begin{equation}
	f^i = \Bigg(\frac{\mbf{f}\cdot\mbf{v}}{c\sqrt{1-\dfrac{v^2}{c^2}}}, \frac{\mbf{f}}{\sqrt{1-\dfrac{v^2}{c^2}}}\Bigg)
\end{equation}
其时间分量与外力对该粒子的功率相关。

将式\eqref{chapter2:四维动量矢量}代入式\eqref{chapter2:四维动量的平方}就得到相对论的Hamilton-Jacobi方程
\begin{equation}
	\frac{\pl S}{\pl x^i} \frac{\pl S}{\pl x_i} = \eta^{ij} \frac{\pl S}{\pl x^i}\frac{\pl S}{\pl x^j} = m^2c^2
	\label{chapter2:相对论力学的Hamilton-Jacobi方程}
\end{equation}
或者将其展开为
\begin{equation}
	\frac{1}{c^2}\left(\frac{\pl S}{\pl t}\right)^2 - \left(\frac{\pl S}{\pl x}\right)^2 - \left(\frac{\pl S}{\pl y}\right)^2 - \left(\frac{\pl S}{\pl z}\right)^2 = m^2c^2
	\label{chapter2:相对论力学的Hamilton-Jacobi方程2}
\end{equation}
将方程\eqref{chapter2:相对论力学的Hamilton-Jacobi方程2}过渡到经典力学极限需要首先作变换
\begin{equation}
	S = S'-mc^2t
	\label{chapter2:作用量的代换}
\end{equation}
将$S'$作为新的作用量。这是由于相对论力学中一个粒子的能量包括了一个经典力学中没有的$mc^2$项。将式\eqref{chapter2:作用量的代换}代入\eqref{chapter2:相对论力学的Hamilton-Jacobi方程2}中,可得
\begin{equation*}
	\frac{1}{2mc^2} \left(\frac{\pl S'}{\pl t}\right)^2 - \frac{\pl S'}{\pl t} - \frac{1}{2m}\left[\left(\frac{\pl S'}{\pl x}\right)^2 + \left(\frac{\pl S'}{\pl y}\right)^2 + \left(\frac{\pl S'}{\pl z}\right)^2\right] = 0
\end{equation*}
在$c\to +\infty$的极限情况下,这个方程就退化为经典的Hamilton-Jacobi方程。

\section{分布函数的变换}

在许多物理问题中都需要处理粒子动量的分布函数$f(\mbf{p})$,乘积$f(\mbf{p})\mathd p_x\mathd p_y\mathd p_z$表示动量分量在给定间隔$\mathd p_x, \mathd p_y, \mathd p_z$内的粒子数,即在动量空间内体积元$\mathd^3p = \mathd p_x\mathd p_y \mathd p_z$内的粒子数。于是当我们从一个参考系变换到另一个参考系中时,需要同时获得分布函数$f(\mbf{p})$的变换。

因为$f(\mbf{p})\mathd p_x\mathd p_y \mathd p_z$表示粒子数,这显然是一个不变量,所以只需要确定“体积元”$\mathd p_x\mathd p_y \mathd p_z$在Lorentz变换下的性质。引入一个四维坐标系,其坐标为四维动量矢量,则$\mathd p_x\mathd p_y \mathd p_z$可以看作是由方程$p^ip_i=m^2c^2$定义的超曲面元的零分量。这个超曲面元是沿该超曲面法线指向的四维矢量。超曲面$p^ip_i=m^2c^2$的法线方向与四维矢量$p_i$一致。由此可知,比值
\begin{equation}
	\frac{\mathd p_x\mathd p_y\mathd p_z}{\E}
\end{equation}
是一个不变量,因为它是两个平行四维矢量的对应分量的比值。由于$f(\mbf{p})\mathd p_x\mathd p_y \mathd p_z$也是不变量,将其写作
\begin{equation*}
	f(\mbf{p})\E \frac{\mathd p_x\mathd p_y\mathd p_z}{\E}
\end{equation*}
由此可得乘积$f(\mbf{p})\E$是不变量。因此可以得到$K'$系中的分布函数与$K$系中分布函数的关系为
\begin{equation*}
	f'(\mbf{p}')\E' = f(\mbf{p})\E
\end{equation*}
由此即有
\begin{equation}
	f'(\mbf{p}') = \frac{f(\mbf{p})\E}{\E'}
\end{equation}
其中$\mbf{p}$和$\E$需要利用变换公式\eqref{chapter2:惯性系能量动量变换关系式}表示。

在气体动理论中,分布函数则表示为$f(\mbf{r},\mbf{p})$,乘积$f(\mbf{r},\mbf{p})\mathd p_x\mathd p_y\mathd p_z\mathd V$表示体积元$\mathd V$中动量在间隔$\mathd p_x, \mathd p_y, \mathd p_z$内的粒子数。函数$f(\mbf{r},\mbf{p})$称为{\bf 相空间}中的分布函数,微分形式$\mathd \tau = \mathd^3p\mathd V$是这个空间的体积元。

方便起见,再引入一个具有给定动量的粒子在其中处于静止的参考系$K_0$,粒子在该参考系中占有的体积元即为其固有体积$\mathd V_0$。由于$K_0$相对于该粒子是静止的,所以$K$系和$K'$系相对于$K_0$系的速度与这些粒子在$K$和$K'$系中的速度相同,因此有
\begin{equation*}
	\mathd V = \mathd V_0\sqrt{1-\frac{v^2}{c^2}},\quad \mathd V' = \mathd V_0\sqrt{1-\frac{v'^2}{c^2}}
\end{equation*}
由此即有
\begin{equation*}
	\frac{\mathd V}{\mathd V'} = \frac{\sqrt{1-\dfrac{v^2}{c^2}}}{\sqrt{1-\dfrac{v'^2}{c^2}}} = \frac{\E'}{\E}
\end{equation*}
又由于$\dfrac{\mathd^3p}{\E} = \dfrac{\mathd^3p'}{\E'}$,即有
\begin{equation*}
	\mathd \tau = \mathd \tau'
\end{equation*}
即相空间的体积元是不变量\footnote{此处相当于是将经典统计力学中的Liouville定理推广至了相对论情形。}。因为粒子数$f(\mbf{r},\mbf{p})\mathd \tau$根据定义也是不变量,因而可得:相空间的分布函数是不变量,即
\begin{equation}
	f'(\mbf{r}',\mbf{p}') = f(\mbf{r},\mbf{p})
\end{equation}
式中的$\mbf{r}',\mbf{p}'$由Lorentz变换公式与$\mbf{r},\mbf{p}$相联系。

\section{粒子的衰变}

现在来考虑一个质量为$M$的粒子自发衰变为质量分别为$m_1$和$m_2$两部分的情形。将能量守恒定律应用于原粒子处于静止的参考系中,可得
\begin{equation}
	Mc^2 = \E_{10}+\E_{20}
	\label{chapter2:粒子衰变的能量守恒}
\end{equation}
式中$\E_{10}$和$\E_{20}$是出射粒子的能量。因为它们的动量都非零,故$\E_{10} > m_1c^2, \E_{20} > m_2c^2$,因此仅当$M>m_1+m_2$时,式\eqref{chapter2:粒子衰变的能量守恒}才能满足,即一个粒子可以自发衰变为质量之和小于该粒子的两部分。另一方面,如果$M<m_1+m_2$,则粒子对于该类型的衰变是稳定的,不会自发衰变。如果需要在这种情况下人为引起衰变,必须从外界对其注入至少等于其“束缚能”$(m_1+m_2-M)c^2$的能量。

衰变过程中,动量也是守恒的。因为粒子的初始动量是零,故出射粒子的动量之和也是零,即
\begin{equation}
	\mbf{p}_{10}+\mbf{p}_{20} = \mbf{0}
	\label{chapter2:粒子衰变的动量守恒1}
\end{equation}
因而有$p_{10}^2=p_{20}^2$,即有
\begin{equation}
	\E_{10}^2-m_1^2c^4 = \E_{20}^2-m_2^2c^4
	\label{chapter2:粒子衰变的动量守恒2}
\end{equation}
式\eqref{chapter2:粒子衰变的能量守恒}和式\eqref{chapter2:粒子衰变的动量守恒2}唯一决定了出射粒子的能量:
\begin{equation}
\begin{cases}
	\ds \E_{10} = \frac{M^2+m_1^2-m_2^2}{2M}c^2 \\
	\ds \E_{20} = \frac{M^2-m_1^2+m_2^2}{2M}c^2
\end{cases}
\end{equation}

在一定意义上,这个问题的逆问题就是计算两个碰撞粒子在其总动量为零的参考系\footnote{这个参考系简称为{\bf 动量中心系}或“C系”。}中的总能量$Mc^2$。这个量的计算给出了伴随碰撞粒子状态改变或新粒子“产生”的各种非弹性碰撞过程可能存在的判据。仅当“反应产物”的质量之和不超过$M$时,这类过程才能够发生。

假设在初始参考系(即{\bf 实验室系})中,一个质量为$m_1$能量为$\E_1$的粒子同一个质量为$m_2$的静止粒子相撞,这两个粒子的总能量是
\begin{equation*}
	\E = \E_1+\E_2 = \E_1 + m_2c^2
\end{equation*}
它们的总动量则为
\begin{equation*}
	\mbf{p} = \mbf{p}_1+\mbf{p}_2 = \mbf{p}_1
\end{equation*}
将两个粒子看成一个复合系统,由式\eqref{chapter2:自由实物粒子能量、动量和速度之间的关系}可得其整体的运动速度为
\begin{equation}
	\mbf{V} = \frac{\mbf{p}c^2}{\E} = \frac{\mbf{p}_1c^2}{\E+m_2c^2}
\end{equation}
这个速度即为C系相对于实验室系(L系)的运动速度。

当然在测定$M$时,没有必要从一个参考系变换到另一个参考系\footnote{因为质量$M$是四维标量。}。直接利用实物粒子的能量动量关系\eqref{chapter2:自由实物粒子能量与动量之间的关系}即有
\begin{equation*}
	M^2c^4 = \E^2-p^2c^2 = (\E_1+m_2c^2)^2-(\E_1^2-m_1^2c^4)
\end{equation*}
由此可有
\begin{equation}
	M^2 = m_1^2+m_2^2+\frac{2m_2\E_1}{c^2}
\end{equation}

\iffalse
\begin{example}
一个以速度$V$运动的粒子在“飞行”中分解为两个粒子。求这些粒子的出射角同其能量之间的关系。
\end{example}
\begin{solution}
记$\E_0$为衰变粒子之一在C系中的能量,$\E$是这一粒子在L系中的能量
\end{solution}
\fi

\section{不变截面}

碰撞过程由其{\bf 有效截面}(或{\bf 截面})表征,它决定碰撞的粒子束之间发生的(特定类型的)碰撞数。

假设有两个碰撞束,用$n_1$和$n_2$表示其中的粒子数密度(即单位体积内的粒子数),用$\mbf{v}_1$和$\mbf{v}_2$表示粒子的速度。在粒子2处于静止的参考系(或者称粒子2的{\bf 静止系})中,只需考虑粒子束1与静止靶的碰撞。于是按照碰撞截面$\sigma$的通常定义,体积$\mathd V$内时间$\mathd t$中发生的碰撞数是
\begin{equation}
	\mathd \nu = \sigma v_{\text{rel}}n_1n_2 \mathd V \mathd t
	\label{chapter2:不变截面1}
\end{equation}
式中$v_{\text{rel}}$是粒子1在粒子2静止系中的速度。

由于$\mathd \nu$是粒子数,因此$\mathd \nu$应该是不变量。下面我们来将它表示为可适用于任何参考系的形式:
\begin{equation}
	\mathd \nu = An_1n_2\mathd V\mathd t
	\label{chapter2:不变截面2}
\end{equation}
式中$A$是一个待定的数,在粒子之一为静止的参考系中,$A$的值为$\sigma v_{\text{rel}}$。

在式\eqref{chapter2:不变截面2}中,微分形式$\mathd V\mathd t$是不变量,因此乘积$An_1n_2$必然也是不变量。而给定体积元中的粒子数$n\mathd V$也是不变量,由此即可得到粒子数密度的变换规律。记$n\mathd V = n_0\mathd V_0$,其中带有指标$0$的量表示其在静止系中的值,利用体积变换公式\eqref{chapter1:固有体积},可得粒子数密度之间的变换关系为
\begin{equation}
	n = \frac{n_0}{\sqrt{1-\dfrac{v^2}{c^2}}}
	\label{chapter2:不变截面3}
\end{equation}
再由式\eqref{chapter2:自由实物粒子能量与动量之间的关系}和式\eqref{chapter2:自由实物粒子能量、动量和速度之间的关系}可得
\begin{equation*}
	\frac{\E^2}{m^2c^4} = \frac{p^2c^2}{m^2c^4} + 1 = 1 + \frac{\E^2}{m^2c^4} \frac{v^2}{c^2}
\end{equation*}
即有
\begin{equation*}
	\frac{\E}{mc^2} = \frac{1}{\sqrt{1-\dfrac{v^2}{c^2}}}
\end{equation*}
据此以及式\eqref{chapter2:不变截面3}可得
\begin{equation}
	n = \frac{\E}{mc^2}n_0
	\label{chapter2:不变截面4}
\end{equation}

因此,$An_1n_2$是不变量与$A\E_1\E_2$是不变量等价。这个条件也可以表示为如下更方便的形式,即量
\begin{equation}
	A\frac{\E_1\E_2}{p_{1i}p_2^i} = A\frac{\E_1\E_2}{\E_1\E_2-\mbf{p}_1\cdot \mbf{p}_2}
	\label{chapter2:不变截面5}
\end{equation}
也是不变量。在粒子2的静止系中,有$\E_2=m_2c^2, \mbf{p}_2=\mbf{0}$,此时不变量\eqref{chapter2:不变截面5}就等于$A$。另一方面,在该参考系中有$A = \sigma v_{\text{rel}}$。所以再任意参考系中有
\begin{equation}
	A = \sigma v_{\text{rel}} \frac{p_{1i}p_2^i}{\E_1\E_2}
	\label{chapter2:不变截面6}
\end{equation}

下面用两个粒子的速度来改写该式。在粒子2的静止系中,有
\begin{equation*}
	p_{1i}p^{2i} = \frac{\E_1\E_2}{c^2} = \E_1m_2 = \frac{m_1m_2c^2}{\sqrt{1-\dfrac{v_{\text{rel}}^2}{c^2}}}
\end{equation*}
于是有
\begin{equation}
	v_{\text{rel}} = c\sqrt{1-\frac{m_1^2m_2^2c^4}{(p_{1i}p_2^i)^2}}
	\label{chapter2:不变截面7}
\end{equation}
利用式\eqref{chapter2:自由实物粒子的能量}和式\eqref{chapter2:自由实物粒子的动量}可得
\begin{equation*}
	p_{1i}p_2^i = \frac{\E_1\E_2}{c^2} - \mbf{p}_1\cdot\mbf{p}_2 = m_1m_2c^2 \frac{1-\dfrac{\mbf{v}_1}{c}\cdot \dfrac{\mbf{v}_2}{c}}{\sqrt{\left(1-\dfrac{v_1^2}{c^2}\right)\left(1-\dfrac{v_2^2}{c^2}\right)}}
\end{equation*}
据此可有
\begin{equation}
	v_{\text{rel}} = \frac{\sqrt{(\mbf{v}_1-\mbf{v}_2)^2-(\mbf{v}_1 \times \mbf{v}_2)^2}}{1-\dfrac{\mbf{v}_1}{c}\cdot \dfrac{\mbf{v}_2}{c}}
	\label{chapter2:不变截面8-粒子间相对速度的计算}
\end{equation}
注意到式\eqref{chapter2:不变截面8-粒子间相对速度的计算}对$\mbf{v}_1$和$\mbf{v}_2$是对称的,即相对速度的数值与用来定义它的粒子的选择无关。

由此,将式\eqref{chapter2:不变截面7}或式\eqref{chapter2:不变截面8-粒子间相对速度的计算}代入式\eqref{chapter2:不变截面6}和式\eqref{chapter2:不变截面2}中,即得到不变截面的最终形式为
\begin{equation}
	\mathd \nu = \sigma \frac{\sqrt{(p_{1i}p_2^i)^2-m_1^2m_2^2c^4}}{\E_1\E_2}n_1n_2 \mathd V\mathd t
	\label{chapter2:不变截面最终形式1}
\end{equation}
或者
\begin{equation}
	\mathd \nu = \sigma \sqrt{(\mbf{v}_1-\mbf{v}_2)^2-(\mbf{v}_1 \times \mbf{v}_2)^2} n_1n_2 \mathd V\mathd t
	\label{chapter2:不变截面最终形式2}
\end{equation}

如果速度$\mbf{v}_1$和$\mbf{v}_2$共线,此时$\mbf{v}_1\times \mbf{v}_2 = \mbf{0}$,于是式\eqref{chapter2:不变截面最终形式2}化为如下形式:
\begin{equation}
	\mathd \nu = \sigma |\mbf{v}_1-\mbf{v}_2|n_1n_2 \mathd V\mathd t
\end{equation}

本节的中间结果可以导出如下一个重要结论。
\begin{example}
求相对论速度空间中的“线元”,即速度为$\mbf{v}$和$\mbf{v}+\mathd \mbf{v}$两点间的相对速度。
\end{example}
\begin{solution}
根据式\eqref{chapter2:不变截面8-粒子间相对速度的计算}可得
\begin{equation*}
	\mathd l_{\mbf{v}}^2 = \frac{(\mathd \mbf{v})^2 - (\mbf{v}\times \mathd \mbf{v})^2}{\left(1-\dfrac{v^2}{c^2}\right)^2}
\end{equation*}
如果在速度空间中引入球坐标系,则有
\begin{equation*}
	\mathd l_{\mbf{v}}^2 = \frac{\mathd v^2}{\left(1-\dfrac{v^2}{c^2}\right)^2} + \frac{v^2}{1-\dfrac{v^2}{c^2}}(\mathd \theta^2 + \sin^2 \theta \mathd \phi^2)
\end{equation*}
如果通过方程$v = \tanh \chi$引进新变量$\chi$来代替$v$,则线元表示为
\begin{equation}
	\mathd l_{\mbf{v}}^2 = \mathd \chi^2 + \sinh^2\chi (\mathd \theta^2 + \sin^2 \theta\mathd \phi^2)
\end{equation}

从几何的观点来看,这就是三维Lobachevsky空间\footnote{即罗巴切夫斯基空间。}(负常曲率空间)的线元。
\end{solution}

\section{粒子的弹性碰撞}

现在我们从相对论力学的观点来考虑粒子的{\bf 弹性碰撞}。将两个碰撞粒子的质量、动量和能量分别记作$m_1, \mbf{p}_1, \E_1$和$m_2, \mbf{p}_2, \E_2$,用带撇号的相应量表示碰撞后的量。

碰撞过程中的动量个能量守恒定律可以一并写为四维动量守恒方程
\begin{equation}
	p_1^i + p_2^i = p_1'^i + p_2'^i
	\label{chapter2:粒子的弹性碰撞1}
\end{equation}
从这个四维矢量方程可以构造出有助于进一步计算的不变关系式。将式\eqref{chapter2:粒子的弹性碰撞1}
改写为形式
\begin{equation}
	p_1^i+p_2^i - p_1'^i = p_2'^i
	\label{chapter2:粒子的弹性碰撞2}
\end{equation}
将其两边平方,即有
\begin{equation*}
	(p_1^i+p_2^i - p_1'^i)(p_{1i}+p_{2i} - p'_{1i}) = p_2'^ip'_{2i}
\end{equation*}
注意到其中$p_1^ip_{1i} = p_1'^ip'_{1i} = m_1^2c^2, p_2^ip_{2i} = p_2'^ip'_{2i} = m_2^2c^2$,可得
\begin{equation}
	m_1^2c^2 + p_{1i}p_2^i - p_{1i}p_1'^i - p_{2i}p_1'^i = 0
	\label{chapter2:粒子的弹性碰撞3}
\end{equation}
类似的将式$p_1^i+p_2^i - p_2'^i = p_1'^i$两边平方,可得
\begin{equation}
	m_2^2c^2 + p_{1i}p_2^i - p_{2i}p_2'^i - p_{1i}p_2'^i = 0
	\label{chapter2:粒子的弹性碰撞4}
\end{equation}

现在来考虑实验室系(L系)中的碰撞,该系中的粒子$m_2$碰撞前处于静止,即$\mbf{p}_2=\mbf{0}, \E_2 = m_2c^2$,于是有
\begin{equation}
\begin{cases}
	p_{1i}p_2^i = \E_1m_2 \\
	p_{2i}p_1'^i = m_2\E'_1 \\
	p_{1i}p_1'^i = \dfrac{\E_1\E'_1}{c^2} - \mbf{p}_1 \cdot \mbf{p}'_1 = \dfrac{\E_1\E'_1}{c^2} - p_1p'_1\cos \theta_1
\end{cases}
\label{chapter2:粒子的弹性碰撞5}
\end{equation}
式中$\theta_1$是入射粒子$m_1$的散射角。将式\eqref{chapter2:粒子的弹性碰撞5}代入式\eqref{chapter2:粒子的弹性碰撞3}中可得
\begin{equation}
	\cos \theta_1 = \frac{\E'_1(\E_1+m_2c^2) - \E_1m_2c^2-m_1^2c^4}{c^2p_1p'_1}
	\label{chapter2:粒子弹性碰撞后粒子1的散射角}
\end{equation}
类似地,由式\eqref{chapter2:粒子的弹性碰撞4}可得
\begin{equation}
	\cos \theta_2 = \frac{(\E_1+m_2c^2)(\E'_2-m_2c^2)}{c^2p_1p'_2}
	\label{chapter2:粒子弹性碰撞后粒子2的散射角}
\end{equation}
式中$\theta_2$是入射粒子的动量$\mbf{p}_1$与变换后动量$\mbf{p}'_2$之间的夹角。

式\eqref{chapter2:粒子弹性碰撞后粒子1的散射角}和式\eqref{chapter2:粒子弹性碰撞后粒子2的散射角}将L系中两个粒子的散射角与它们在碰撞中的能量变化联系了起来。反演这些公式,就可以用$\theta_1$和$\theta_2$来表示能量$\E'_1$和$\E'_2$。将式\eqref{chapter2:粒子弹性碰撞后粒子2的散射角}两边平方,并将关系
\begin{equation*}
	p_1^2c^2 = \E_1^2 - m_1^2c^4,\quad p_2'^2c^2 = \E_2'^2-m_2^2c^4
\end{equation*}
代入,可得
\begin{equation}
	\E_2' = m_2c^2 \frac{(\E_1+m_2c^2)^2 + (\E_1^2-m_1^2c^4)\cos^2 \theta_2}{(\E_1+m_2c^2)^2 - (\E_1^2-m_1^2c^4)\cos^2 \theta_2}
	\label{chapter2:粒子的弹性碰撞6}
\end{equation}
反演公式\eqref{chapter2:粒子弹性碰撞后粒子1的散射角}可得出在一般情形下用$\theta_1$表示$\E'$的非常复杂的公式。但当入射粒子质量为零,即$m_1=0$时,有$p_1c = \E_1, p_1'c = \E'$。此时入射粒子碰撞后的能量可以用其偏转角度表示为
\begin{equation}
	\E_1' = \frac{m_2c^2}{1-\cos\theta_1+\dfrac{m_2c^2}{\E_1}}
\end{equation}

注意到,如果$m_1>m_2$,即入射粒子重于靶粒子,则散射角$\theta_1$不能超过某一最大值,即
\begin{equation}
	\sin \theta_{1\max} = \frac{m_2}{m_1}
	\label{chapter2:粒子的弹性碰撞7}
\end{equation}
这与经典结果相一致。

一般来讲,在C系中讨论碰撞最为简单。我们用带有下标$0$的量来代表这个参考系中的量。因此有$\mbf{p}_{10} = -\mbf{p}_{20} =:\mbf{p}_0$。根据动量守恒,碰撞过程中,两个粒子的动量只有转动,保持数值相等且方向相反。再根据能量守恒,每个动量的数值保持不变。

设$\chi$为C系中的散射角,即动量$\mbf{p}_{10}$和$\mbf{p}_{20}$由于碰撞而转过的角。这个量完全决定了C系中,因而也是任何其他参考系中的散射过程。在L系中描述碰撞时它也是方便的,是应用动量和能量守恒以后仍然不定的单一参量。

我们借助这个参量来表示两个粒子在L系中的终态能量。重新计算式\eqref{chapter2:粒子的弹性碰撞3}中的各项,在C系中写出乘积$p_{1i}p_1'^i$如下
\begin{equation}
	p_{1i}p_1'^i = \frac{\E_{10}\E'_{10}}{c^2} - \mbf{p}_{10}\cdot \mbf{p}'_{10} = \frac{\E_{10}^2}{c^2} - p_0^2\cos \chi = p_0^2(1-\cos\chi)+m_1^2c^2
	\label{chapter2:粒子的弹性碰撞8}
\end{equation}
由于$p_{1i}p_1'^i$是四维标量,故在L系中,$p_{1i}p_1'^i$也为式\eqref{chapter2:粒子的弹性碰撞8}之取值。将式\eqref{chapter2:粒子的弹性碰撞8}和式\eqref{chapter2:粒子的弹性碰撞5}代入式\eqref{chapter2:粒子的弹性碰撞3}中,可得
\begin{equation}
	\E_1'-\E_1 = -\frac{p_0^2}{m_2}(1-\cos\chi)
	\label{chapter2:粒子的弹性碰撞9}
\end{equation}
最后还需要用L系中的诸量来表示$p_0^2$,为此,只需让不变量$p_{1i}p_2^i$在L系和C系中的值相等即可,即
\begin{equation*}
	\frac{\E_{10}\E_{20}}{c^2} - \mbf{p}_{10}\cdot \mbf{p}_{20} = \E_1m_2
\end{equation*}
即有
\begin{equation*}
	\sqrt{(p_0^2+m_1^2c^2)(p_0^2+m_2^2c^2)} = \E_1m_2 - p_0^2
\end{equation*}
从上式中解出$p_0^2$可有
\begin{equation}
	p_0^2 = \frac{m_2^2(\E_1^2-m_1^2c^4)}{m_1^2c^2+m_2^2c^2+2\E_1m_2}
	\label{chapter2:粒子的弹性碰撞10}
\end{equation}
因此最后可得
\begin{equation}
	\E_1' = \E_1 - \frac{m_2(\E_1^2-m_1^2c^4)}{m_1^2c^2+m_2^2c^2+2\E_1m_2}(1-\cos \chi)
	\label{chapter2:粒子的弹性碰撞11}
\end{equation}
第二个粒子的能量可以通过能量守恒定律$\E_1+m_2c^2=\E_1'+\E_2'$得到,即
\begin{equation}
	\E_2' = m_2c^2 + \frac{m_2(\E_1^2-m_1^2c^4)}{m_1^2c^2+m_2^2c^2+2\E_1m_2}(1-\cos \chi)
	\label{chapter2:粒子的弹性碰撞12}
\end{equation}

在式\eqref{chapter2:粒子的弹性碰撞11}和式\eqref{chapter2:粒子的弹性碰撞12}中,第二项表示第一个粒子失去并转移给第二个粒子的能量,当$\chi = \pi$时,能量转移最大为
\begin{equation}
	\E'_{2\max}-m_2c^2 = \E_1-\E_{1\min} = \frac{2m_2(\E_1^2-m_1^2c^4)}{m_1^2c^2+m_2^2c^2+2\E_1m_2}
	\label{chapter2:粒子的弹性碰撞13}
\end{equation}
碰撞后入射粒子的最小动能语气初始能量的比值为
\begin{equation}
	\frac{\E'_{1\min}-m_1c^2}{\E_1-m_1c^2} = \frac{(m_1-m_2)^2c^2}{m_1^2c^2+m_2^2c^2+2\E_1m_2}
	\label{chapter2:粒子的弹性碰撞14}
\end{equation}
在低速极限情况下(即$c\to +\infty$时),这个关系趋于常数
\begin{equation*}
	\left(\frac{m_1-m_2}{m_1+m_2}\right)^2
\end{equation*}
在能量$\E_1$很大的相反极限下,关系\eqref{chapter2:粒子的弹性碰撞14}趋于$0$,而量$\E'_{1\min}$则趋于常数
\begin{equation*}
	\E'_{1\min} = \frac{m_1^2c^2+m_2^2c^2}{2m_2}
\end{equation*}

\section{角动量}

在经典力学中,一个封闭系统除动量和能量守恒外,还有角动量守恒,即矢量
\begin{equation*}
	\mbf{L} = \sum \mbf{r}\times \mbf{p}
\end{equation*}
守恒。其中的$\mbf{r}$和$\mbf{p}$是粒子的矢径和动量,求和遍历系统内的所有粒子。由于空间的各向同性,因而有封闭系统的Lagrange函数在系统整体转动下不变,因而导出了角动量守恒。

下面推导角动量守恒的四维形式。设$x^i$是该系统中一个粒子的坐标,在该四维空间中作一个无限小转动\footnote{这样的变换一定是线性变换。},设新旧坐标满足如下关系
\begin{equation}
	x'^i-x^i = x_j\delta \varOmega^{ij}
	\label{chapter2:角动量1}
\end{equation}
其中$\delta \varOmega_{ij}$为该无穷小线性变换的系数,其所有分量构成一个四维张量。四维张量$\delta\varOmega_{ij}$的各个分量的取值需满足:矢径的长度在该转动下不变,即$x'^ix'_i = x^ix_i$,将式\eqref{chapter2:角动量1}的结果代入并略去二阶小量,可得
\begin{equation*}
	x_ix_j\delta \varOmega^{ij}+x^ix^k\delta \varOmega_{ik} = 0
\end{equation*}
或者
\begin{equation}
	x^ix^j \delta\varOmega_{ij} = 0
	\label{chapter2:角动量2}
\end{equation}
这个方程对于任意的$x^i$都要满足。注意到$x^ix^j$是一个对称张量,所以$\delta\varOmega_{ij}$必为反对称张量\footnote{也可将式\eqref{chapter2:角动量2}中的傀指标$i, j$交换,再与自身相加得到。},即
\begin{equation}
	\delta \varOmega_{ij} = -\delta \varOmega_{ji}
	\label{chapter2:角动量3}
\end{equation}

对于起点为$a$终点为$b$的轨道的无穷小坐标变换,根据式\eqref{chapter2.section2.1:作用量的变分}可得
\begin{equation*}
	\delta S = -mu_i\delta x^i \big|_a^b = -p^i\delta x_i\big|_a^b
\end{equation*}
而对系统中所有粒子,系统的作用量的改变则具有形式
\begin{equation*}
	\delta S = -\sum p^i\delta x_i\Big|_a^b
\end{equation*}
式中的求和遍历系统中的所有粒子。在无穷小转动下,有$\delta x_i = x^j\delta \varOmega_{ij}$,于是有
\begin{equation*}
	\delta S = -\delta \varOmega_{ij} \sum p^ix^j\Big|_a^b
\end{equation*}
将张量$\ds\sum p^ix^j$分解为对称和反对称两部分之和,对称部分与反对称张量$\delta \varOmega_{ij}$的乘积恒为零,因此只余下张量$\ds\sum p^ix^j$的反对称部分,即
\begin{equation}
	\delta S = -\delta \varOmega_{ij} \frac12 \sum (p^ix^j-p^jx^i)\Big|_a^b
	\label{chapter2:角动量4}
\end{equation}

对于封闭系统,Lagrange函数是一个不变量,它不因四维空间中的转动而改变,因此对任意的转动,即任意的四维反对称张量$\delta\varOmega_{ij}$都有$\delta S=0$,这说明式\eqref{chapter2:角动量4}中$\delta\varOmega_{ij}$的系数必须为零,即
\begin{equation}
	\sum (p^ix^j-p^jx^i)_b = \sum (p^ix^j-p^jx^i)_a
\end{equation}
这说明在封闭系统中,四维张量
\begin{equation}
	L^{ij} = \sum (x^ip^j-x^jp^i)
	\label{chapter2:四维角动量张量的定义}
\end{equation}
是守恒的。这个反对称张量称为{\bf 四维角动量张量}。这个张量的空间分量是三维角动量矢量$\ds \mbf{L} = \sum\mbf{r}\times \mbf{p}$的分量:
\begin{equation*}
	L^{23} = L_x,\quad L^{31} = L_y,\quad L^{12} = L_z
\end{equation*}
而分量$L^{01}, L^{02}, L^{03}$构成一个矢量$\ds c\sum \left(t\mbf{p}-\dfrac{\E\mbf{r}}{c^2}\right)$,于是,对比式\eqref{chapter1:四维反对称张量的表示}和式\eqref{chapter1:四维反对称张量的逆变分量}的形式,可以把张量$L^{ij}$的分量写成形式
\begin{equation}
	L^{ij} = \left(c\sum \left(t\mbf{p}-\dfrac{\E\mbf{r}}{c^2}\right), -\mbf{L}\right)
	\label{chapter2:四维角动量张量的逆变分量表示}
\end{equation}

特别地,由于四维角动量张量$L^{ij}$对封闭系统守恒,所以有
\begin{equation*}
	\sum \left(t\mbf{p}-\frac{\E\mbf{r}}{c^2}\right) = \const
\end{equation*}
另外由于总能量$\ds\sum \E$也守恒,因此可以将其改写为
\begin{equation}
	\frac{\ds \sum\E\mbf{r}}{\ds \sum \E} - t\,\frac{\ds c^2\sum\mbf{p}}{\ds \sum\E} = \const
	\label{chapter2:角动量5}
\end{equation}
从式\eqref{chapter2:角动量5}中可以看出,矢径
\begin{equation}
	\mbf{R} = \frac{\ds\sum \E\mbf{r}}{\ds\sum\E}
	\label{chapter2:角动量-惯性中心的坐标}
\end{equation}
在以速度
\begin{equation}
	\mbf{v} = \frac{\ds c^2\sum\mbf{p}}{\ds\sum\E}
	\label{chapter2:角动量-惯性中心的速度}
\end{equation}
作匀速运动,这正是系统整体的运动速度。式\eqref{chapter2:角动量-惯性中心的坐标}给出了系统{\bf 惯性中心}坐标的相对论定义。在低速近似下,可近似地令$\E\approx mc^2$,则式\eqref{chapter2:角动量-惯性中心的坐标}化为经典形式
\begin{equation*}
	\mbf{R} = \frac{\ds \sum m\mbf{r}}{\ds \sum m}
\end{equation*}

需要注意的是,矢量\eqref{chapter2:角动量-惯性中心的坐标}并不能被构造为任何一个四维矢量的空间分量,故在参考系的变换下它的分量并不像一个点的坐标那样变换,因此同一粒子系统的惯性中心对不同的参考系来讲是不同的点。

\begin{example}
一物体(即粒子系统)在其以速度$\mbf{V}$运动的参考系$K$中的角动量为$\mbf{L}$,在该物体总体静止的参考系$K_0$中的角动量为$\mbf{L}_{(0)}$,试求$\mbf{L}$和$\mbf{L}_{(0)}$之间的关系。此处两种情形下的角动量都是相对于物体在$K_0$系的惯性中心定义的。
\end{example}
\begin{solution}
将$K$系中的四维坐标记作$x^i$,$K'$系中的四维坐标则记作$x^i_{(0)}$,则根据Lorentz变换可得
\begin{equation*}
\begin{cases}
	\ds x^0 = \frac{x^0_{(0)}+\dfrac{V}{c}x^1_{(0)}}{\sqrt{1-\dfrac{V^2}{c^2}}} \\
	\ds x^1 = \frac{x^1_{(0)}+\dfrac{V}{c}x^0_{(0)}}{\sqrt{1-\dfrac{V^2}{c^2}}} \\
	x^2 = x^2_{(0)} \\
	x^3 = x^3_{(0)}
\end{cases}
\end{equation*}
根据四维张量的变换法则可得
\begin{equation*}
	L^{ij} = \frac{\pl x^i}{\pl x^k_{(0)}} \frac{\pl x^j}{\pl x^l_{(0)}} L^{kl}_{(0)}
\end{equation*}
由此可有
\begin{align*}
	L^{23} & = L^{23}_{(0)} \\
	L^{31} & = \frac{\pl x^1}{\pl x^0_{(0)}} L^{30}_{(0)} + \frac{\pl x^1}{\pl x^1_{(0)}} L^{31}_{(0)} = \frac{-\dfrac{V}{c}}{\sqrt{1-\dfrac{V^2}{c^2}}} L^{03}_{(0)} + \frac{1}{\sqrt{1-\dfrac{V^2}{c^2}}} L^{31}_{(0)} \\
	L^{12} & = \frac{\pl x^1}{\pl x^0_{(0)}} L^{02}_{(0)} + \frac{\pl x^1}{\pl x^1_{(0)}} L^{12}_{(0)} = \frac{\dfrac{V}{c}}{\sqrt{1-\dfrac{V^2}{c^2}}} L^{02}_{(0)} + \frac{1}{\sqrt{1-\dfrac{V^2}{c^2}}} L^{12}_{(0)}
\end{align*}
由于$K_0$系是惯性中心系,因此在$K_0$系中有
\begin{equation*}
	\sum \mbf{p}_{(0)} = \mbf{0},\quad \sum \E_{(0)}\mbf{r}_{(0)} = \mbf{0}
\end{equation*}
即$L^{01}_{(0)} = L^{02}_{(0)} = L^{03}_{(0)} = 0$,故有
\begin{equation*}
\begin{cases}
	\ds L^{23} = L^{23}_{(0)} \\
	\ds L^{31} = \frac{1}{\sqrt{1-\dfrac{V^2}{c^2}}} L^{31}_{(0)} \\
	\ds L^{12} = \frac{1}{\sqrt{1-\dfrac{V^2}{c^2}}} L^{12}_{(0)}
\end{cases}
\end{equation*}
即
\begin{equation*}
	L_x = L_{(0)x},\quad L_y = \frac{L_{(0)y}}{\sqrt{1-\dfrac{V^2}{c^2}}} ,\quad L_z = \frac{L_{(0)z}}{\sqrt{1-\dfrac{V^2}{c^2}}}
\end{equation*}
\end{solution}
\chapter{电磁场中的电荷}

\section{相对论中的基本粒子}

粒子间的相互作用可以用{\bf 场}的概念来描述。在这种观点下,我们不认为是一个粒子直接作用于另一个粒子,而是认为一个粒子在其周围建立起场,这个场内的任何其他粒子都受到一定的力的作用。在经典力学中,场仅仅是用来描述粒子相互作用这一现象的方法。但在相对论中,由于相互作用是以有限速度传播的,因此,在某一时刻,作用在一个粒子上的力并不是由其他粒子在该时刻的位置决定。某一个粒子改变了位置,需要经过一段时间后才能影响到其他粒子。这说明了场本身具有物理上的真实性。相互作用具有传播速度表明粒子之间不能直接发生相互作用,而仅能影响其周围空间中的邻近区域。因此,应当认为,一个粒子与场发生相互作用,然后场与粒子发生相互作用。

在经典力学中,我们可以引入刚体的概念。刚体指在任何情况下都不会发生形变的物体。类似地,如果希望在相对论中引入刚体的概念,则这种刚体应该在它们处于静止的参考系中其所有尺寸都保持不变。但是,相对论使得一般情形下的刚体不可能存在。

例如,考虑一个绕自身轴转动的圆盘,并假设它是刚体。固连于圆盘的参考系显然不是惯性系。但是对于圆盘的每一个无限小单元,可以引入一个惯性参考系,其中该单元在某一时刻处于静止。而对圆盘上的不同单元,这些惯性系之间必然是有相对运动的。现在考察圆盘的一条半径上的所有无限小单元,由于它们之间的相对运动与其自身取向相垂直,因此一个静止的观察者在该半径扫过他时测量出旋转圆盘的半径应与圆盘静止时相同。而另一方面,在给定时刻圆盘圆周上从静止观察者身旁经过的每一单元的长度由于其取向与运动方向垂直,会发生Lorentz收缩而导致静止观察者测量出的旋转圆盘整个圆周的长度将小于静止圆盘的周长。于是我们发现,由于圆盘的转动静止观察者测得旋转圆盘的圆周与半径的比值将不再是$2\pi$(事实上这个数值将小于$2\pi$)。这个结论表明,圆盘实际上不再是刚体,它在转动时必然发生了某种复杂的形变,这种形变则显然与其组成物质的弹性性质有关。

还可以用另外一种方法来证明刚体是不可能存在的。假设某一外力作用在刚体的某一点上,使这个物体发生了运动。如果这个物体是刚体,那么它上面的任何一点都必须在相同时刻与受到外力的点同时开始运动,否则物体就要发生变形了。然而相对论表明相互作用具有传播速度,力是以有限速度从其作用点传到其余的点,它们不可能同时开始运动。

刚体的不存在性使得相对论力学中的基本粒子必须以{\bf 几何点}的形式存在而不能有有限的尺寸。即在经典的(非量子的)相对论力学中,基本粒子不能被赋予有限的尺寸,必须将其当做几何点来看待。

\section{场的四维势}

一个在给定电磁场中运动的粒子的作用量将由两部分组成\footnote{类比经典力学对势能的引入方式。}:一项即为自由粒子的作用量\eqref{chapter2:自由实物粒子的作用量},而另一项则描述粒子与场的相互作用,其中必然包括表征粒子本身性质的量和表征场的性质的量。

实验表明,粒子同电磁场相互作用的性质由一个量所决定,这个量称为粒子的{\bf 电荷}$e$\footnote{此处需要指出的是,这里的$e$只是指任意的电荷而非特指元电荷。现在我们还没有建立任何将电磁学量同任何已知的量联系起来的关系,因此这些新引入的量的单位可以任意选取。}。电荷可以为正,也可以为负,也可以为零。而场的性质则由一个四维矢量$A^i$,称为{\bf 四维势}表征,其分量是四维坐标的函数。这些量以形式
\begin{equation*}
	-\frac{e}{c}\int_a^b A_i\mathd x^i
\end{equation*}
出现在作用量里。因此,电磁场中带电粒子的作用量将具有如下形式:
\begin{equation}
	S = \int_a^b \left(-mc\mathd s - \frac{e}{c}A_i\mathd x^i\right)
	\label{chapter3:电磁场中带电粒子的作用量1}
\end{equation}

四维势矢量$A^i$的三个空间分量构成一个三维空间矢量$\mbf{A}$,称为电磁场的{\bf 矢势},时间分量称为电磁场的{\bf 标势},记作$A^0 = \phi$。即有
\begin{equation}
	A^i = \left(\phi,\mbf{A}\right),\quad A_i = \left(\phi,-\mbf{A}\right)
	\label{chapter3:电磁场的标势和矢势}
\end{equation}
所以作用量的积分可以写作
\begin{equation*}
	S = \int_a^b \left(-mc\mathd s + \frac{e}{c}\mbf{A}\cdot \mathd \mbf{r} - e\phi \mathd t\right)
\end{equation*}
引入$\mbf{v} = \dfrac{\mathd \mbf{r}}{\mathd t}$可得
\begin{equation}
	S = \int_{t_1}^{t_2} \left(-mc^2\sqrt{1-\frac{v^2}{c^2}} + \frac{e}{c}\mbf{A}\cdot \mbf{v} - e\phi\right) \mathd t
	\label{chapter3:电磁场中带电粒子的作用量2}
\end{equation}
由此可得电磁场中带电粒子的Lagrange函数
\begin{equation}
	L = -mc^2\sqrt{1-\frac{v^2}{c^2}} + \frac{e}{c}\mbf{A}\cdot \mbf{v} - e\phi
	\label{chapter3:电磁场中带电粒子的Lagrange函数}
\end{equation}
这个Lagrange函数与自由粒子Lagrange函数\eqref{chapter2:自由实物粒子的Lagrange函数}相差了
\begin{equation*}
	\frac{e}{c}\mbf{A}\cdot \mbf{v} - e\phi
\end{equation*}
该项描述了带电粒子与电磁场的相互作用。

粒子的广义动量为
\begin{equation}
	\mbf{P} = \frac{\pl L}{\pl \mbf{v}} = \frac{m\mbf{v}}{\sqrt{1-\dfrac{v^2}{c^2}}} + \frac{e}{c}\mbf{A} = \mbf{p}+\frac{e}{c}\mbf{A}
	\label{chapter3:电磁场中带电粒子的广义动量}
\end{equation}
此处$\mbf{p}$表示该带电粒子的{\bf 机械动量},以后简称为动量。

而电磁场中带电粒子的Hamilton量则由式
\begin{equation*}
	\mathscr{H} = \mbf{v}\cdot \frac{\pl L}{\pl \mbf{v}} - L
\end{equation*}
决定。将式\eqref{chapter3:电磁场中带电粒子的Lagrange函数}代入,可得
\begin{equation}
	\mathscr{H} = \frac{mc^2}{\sqrt{1-\dfrac{v^2}{c^2}}}+e\phi
	\label{chapter3:电磁场中带电粒子的Hamilton量-用速度表示的Hamilton量}
\end{equation}
但粒子的Hamilton量需用粒子的广义动量表示而非运动速度,由式\eqref{chapter3:电磁场中带电粒子的广义动量}和式\eqref{chapter3:电磁场中带电粒子的Hamilton量-初步}可以看出,$\mathscr{H}-e\phi$与$\mbf{P}-\dfrac{e}{c}\mbf{A}$之间应该满足
\begin{equation}
	\left(\frac{\mathscr{H}-e\phi}{c}\right)^2 = m^2c^2+\left(\mbf{P}-\dfrac{e}{c}\mbf{A}\right)^2
	\label{chapter3:电磁场中带电粒子的Hamilton量-初步}
\end{equation}
即有
\begin{equation}
	\mathscr{H} = \sqrt{m^2c^4+c^2\left(\mbf{P}-\dfrac{e}{c}\mbf{A}\right)^2}+e\phi
	\label{chapter3:电磁场中带电粒子的Hamilton量}
\end{equation}

在低速近似下,电磁场中带电粒子的Lagrange函数化为
\begin{equation}
	L = \frac12mv^2+\dfrac{e}{c}\mbf{A}\cdot \mbf{v}-e\phi
	\label{chapter3:低速近似下,电磁场中带电粒子的Lagrange函数}
\end{equation}
此时
\begin{equation*}
	\mbf{p} = m\mbf{v} = \mbf{P}-\dfrac{e}{c}\mbf{A}
\end{equation*}
Hamilton量的表达式为
\begin{equation}
	\mathscr{H} = \frac{1}{2m}\left(\mbf{P}-\dfrac{e}{c}\mbf{A}\right)^2 + e\phi
	\label{chapter3:低速近似下,电磁场中带电粒子的Hamilton量}
\end{equation}

最后来推导出电磁场中带电粒子的Hamilton-Jacobi方程,只需在Hamilton量\eqref{chapter3:电磁场中带电粒子的Hamilton量}中,用$\bnb S$代替广义动量$\mbf{P}$,用$-\dfrac{\pl S}{\pl t}$代替$\mathscr{H}$即可。由此,根据式\eqref{chapter3:电磁场中带电粒子的Hamilton量-初步}可得
\begin{equation}
	\left(\bnb S-\dfrac{e}{c}\mbf{A}\right)^2-\frac{1}{c^2}\left(\frac{\pl S}{\pl t} + e\phi\right)^2 + m^2c^2 = 0
	\label{chapter3:电磁场中带电粒子的Hamilton-Jacobi方程}
\end{equation}

\section{场中带电粒子的运动方程}\label{chapter3:section:场中带电粒子的运动方程}

场内的带电粒子不仅会受到场的作用力,还会反过来对场起作用,改变场的分布。但是,当电荷量很小的时候,电荷对于场的作用就可以忽略不计。在这种情况下,当我们只考虑电荷在给定的外电磁场中的运动时,可以假设场本身与电荷的坐标或速度无关。

现在我们在这种假设下,推导出带电粒子在给定电磁场内的运动方程。通过对作用量进行变分,可得运动方程就是Lagrange方程
\begin{equation}
	\frac{\mathd}{\mathd t} \frac{\pl L}{\pl \mbf{v}} - \frac{\pl L}{\pl \mbf{r}} = \mbf{0}
	\label{chapter3:电磁场中带电粒子的Lagrange方程}
\end{equation}
其中Lagrange函数由式\eqref{chapter3:电磁场中带电粒子的Lagrange函数}决定。导数$\dfrac{\pl L}{\pl \mbf{v}}$就是粒子的广义动量\eqref{chapter3:电磁场中带电粒子的广义动量},然后可以计算
\begin{equation*}
	\frac{\pl L}{\pl \mbf{r}} = \dfrac{e}{c}\bnb (\mbf{A}\cdot \mbf{v}) - e\bnb \phi
\end{equation*}
根据矢量恒等式
\begin{equation*}
	\bnb (\mbf{a}\cdot \mbf{b}) = (\mbf{a}\cdot \bnb)\mbf{b} + (\mbf{b}\cdot \bnb)\mbf{a} + \mbf{b}\times (\bnb \times \mbf{a}) + \mbf{a}\times (\bnb \times \mbf{b})
\end{equation*}
可得
\begin{equation*}
	\frac{\pl L}{\pl \mbf{r}} = \dfrac{e}{c}(\mbf{v}\cdot \bnb)\mbf{A} + \dfrac{e}{c}\mbf{v}\times (\bnb \times \mbf{A}) - e\bnb \phi
\end{equation*}
因此,Lagrange方程\eqref{chapter3:电磁场中带电粒子的Lagrange方程}变为
\begin{equation*}
	\frac{\mathd}{\mathd t}\left(\mbf{p}+\dfrac{e}{c}\mbf{A}\right) = \dfrac{e}{c}(\mbf{v}\cdot \bnb)\mbf{A} + \dfrac{e}{c}\mbf{v}\times (\bnb \times \mbf{A}) - e\bnb \phi
\end{equation*}
为了计算上式左端,考虑到矢势$\mbf{A}$是空间坐标和时间的函数,因此其全微分为
\begin{equation*}
	\mathd \mbf{A} = \frac{\pl \mbf{A}}{\pl t}\mathd t + (\mathd \mbf{r}\cdot \bnb) \mbf{A}
\end{equation*}
据此可有
\begin{equation*}
	\frac{\mathd \mbf{A}}{\mathd t} = \frac{\pl \mbf{A}}{\pl t} + (\mbf{v}\cdot \bnb)\mbf{A}
\end{equation*}
由此可得带电粒子在给定电磁场中运动的方程为
\begin{equation}
	\frac{\mathd \mbf{p}}{\mathd t} = -\dfrac{e}{c}\frac{\pl \mbf{A}}{\pl t} - e\bnb \phi + \dfrac{e}{c}\mbf{v}\times (\bnb \times \mbf{A})
	\label{chapter3:带电粒子在给定电磁场中的运动方程}
\end{equation}

式\eqref{chapter3:带电粒子在给定电磁场中的运动方程}左端即为粒子的动量对时间的导数,因此其右端就是电磁场作用在带电粒子上的力。这个力可以分为两部分,式\eqref{chapter3:带电粒子在给定电磁场中的运动方程}右端第一、第二项即为第一部分,这一部分的力与粒子的速度无关。第三项为第二部分,这部分的力与粒子的速度有关,它与速度成正比,而且垂直于速度。

我们将作用于单位电荷上的第一部分的力,称为{\bf 电场强度},记作$\mbf{E}$,于是有
\begin{equation}
	\mbf{E} = - \frac1c\frac{\pl \mbf{A}}{\pl t} - \bnb \phi
	\label{chapter3:电场强度的定义}
\end{equation}
电场强度$\mbf{E}$是极矢量。作用于单位电荷上的第二部分的力中的速度因子,称为{\bf 磁场强度}\footnote{如此定义的一个好处就是,电场强度与磁场强度将有相同的单位。},记作$\mbf{H}$,于是有
\begin{equation}
	\mbf{H} = \bnb \times \mbf{A}
	\label{chapter3:磁场强度的定义}
\end{equation}
磁场强度$\mbf{H}$是轴矢量。

如果在一电磁场中$\mbf{E}\neq \mbf{0}$,但$\mbf{H}=\mbf{0}$,我们就称它为{\bf 电场};如果$\mbf{E}= \mbf{0}$,但$\mbf{H}\neq \mbf{0}$,我们就称它为{\bf 磁场}。在一般情形下,电磁场是电场和磁场的叠加。

由此,一个带电粒子在电磁场中的运动方程可以写作
\begin{equation}
	\frac{\mathd \mbf{p}}{\mathd t} = e\mbf{E}+\dfrac{e}{c}\mbf{v}\times \mbf{H}
	\label{chapter3:带电粒子在给定电磁场中的运动方程-Lorentz力}
\end{equation}
等式右端的式子称为{\bf Lorentz力}。其第一部分(电场作用于电荷上的力)与电荷速度无关,并沿着$\mbf{E}$的方向。第二部分(磁场作用于电荷上的力)与电荷速度成正比,而其方向既垂直于速度又垂直于磁场$\mbf{H}$。

粒子在电磁场中的机械能由式\eqref{chapter2:自由实物粒子的能量}决定,即
\begin{equation*}
	\E_{\text{m}} = \frac{mc^2}{\sqrt{1-\dfrac{v^2}{c^2}}}
\end{equation*}
将式\eqref{chapter2:自由实物粒子能量与动量之间的关系}两端对时间求导数,即可得
\begin{equation*}
	\frac{\mathd \E_{\text{m}}}{\mathd t} = \mbf{v}\cdot \frac{\mathd \mbf{p}}{\mathd t}
\end{equation*}
将式\eqref{chapter3:带电粒子在给定电磁场中的运动方程-Lorentz力}中的$\dfrac{\mathd \mbf{p}}{\mathd t}$代入,可得
\begin{equation}
	\frac{\mathd \E_{\text{m}}}{\mathd t} = e\mbf{E}\cdot \mbf{v}
	\label{chapter3:电磁场中带电粒子能量的变化}
\end{equation}

带电粒子的机械能随时间的变化率就是场对粒子做功的功率。对电荷做功的仅仅是电场,磁场不能对在其中运动的粒子做功。

\section{规范不变性}

现在来研究场的势可唯一地确定到什么程度。首先,需要强调的是,场是由它对其内电荷的运动所产生的影响来刻画的。但是在运动方程\eqref{chapter3:带电粒子在给定电磁场中的运动方程-Lorentz力},而只出现了场强$\mbf{E}$和$\mbf{H}$。所以两个场如果用两个矢量$\mbf{E}$和$\mbf{H}$来描述,在物理上也是完全等同的。

假如给定了四维势$A^i$,则根据式\eqref{chapter3:电场强度的定义}和\eqref{chapter3:磁场强度的定义},$\mbf{E}$和$\mbf{H}$就由它们完全唯一地确定了。但是同一个场可以对应于不同的势。对四维势做变换
\begin{equation}
	A'_i = A_i-\frac{\pl f}{\pl x^i}
	\label{chapter3:四维势的一个变换}
\end{equation}
其中$f$是四维坐标的任意函数。经过这样的改变,在作用量积分\eqref{chapter3:电磁场中带电粒子的作用量1}中将出现附加项
\begin{equation}
	\dfrac{e}{c}\frac{\pl f}{\pl x^i}\mathd x^i = \mathd \left(\dfrac{e}{c}f\right)
\end{equation}
然而将一个全微分加在作用量积分的被积函数中,运动方程不会受到影响。

这个变换反映在矢势和标势上可以写作
\begin{equation}
	\mbf{A}'=\mbf{A} + \bnb f,\quad \phi' = \phi-\frac1c\frac{\pl f}{\pl t}
\end{equation}
很容易验证,在此变换下,由式\eqref{chapter3:电场强度的定义}和\eqref{chapter3:磁场强度的定义}定义的电场强度和磁场强度并不发生改变。因此,势的变换\eqref{chapter3:四维势的一个变换}并不改变场,所以势没有被唯一地确定,确定矢势仅仅精确到一个任意函数的梯度,而确定标势则仅仅精确到同一个任意函数的时间导数。

只有那些对于四维势变换\eqref{chapter3:四维势的一个变换}为不变的量才有物理意义,特别地,所有方程在这个变换下必须是不变的。这种不变性称为{\bf 规范不变性(Gauge invariance)}。

势缺乏唯一性,使得我们有可能去选择它们,使他们满足我们所选择的附加条件。由于势仅能精确到相差一个任意函数的四维梯度,因此我们能够令四维势的各个分量之间满足一个额外条件,以确定变换\eqref{chapter3:四维势的一个变换}中的任意函数$f$。特别而言,我们总是能够选择势,使得标势$\phi$为零。

\section{恒定电磁场}

{\bf 恒定}电磁场指与时间无关的电磁场。显然,恒定电磁场的势可以选择为与时间无关而仅与坐标有关的函数,恒定电场和恒定磁场即为
\begin{equation*}
	\mbf{E} = -\bnb \phi,\quad \mbf{H} = \bnb \times \mbf{A}
\end{equation*}

由于势没有被唯一确定,我们可以在标势上加一个任意常数而不改变场。通常要给$\phi$加上一个附加条件,即在空间内某一特定点有一个特定的值。通常,我们规定$\phi$在无穷远处的值为零。这样恒定场的标势就被唯一地确定了。

同样地,矢势也没有被唯一确定,我们可以把任意一个坐标的函数的梯度加到矢势上。

对于恒定电磁场,带电粒子的Lagrange函数中不显含时间。在这种情况下,能量是守恒的,即为其Hamilton量。根据式\eqref{chapter3:电磁场中带电粒子的Hamilton量-用速度表示的Hamilton量},可有
\begin{equation}
	\E = \frac{mc^2}{\sqrt{1-\dfrac{v^2}{c^2}}}+e\phi
	\label{chapter3:恒定电磁场中带电粒子的总能量}
\end{equation}
由于场的存在,粒子的能量加上了一项$e\phi$,它就是带电粒子在电磁场内的势能。此处可以注意到,总能量仅与标势有关,而与矢势无关。这说明磁场不影响带电粒子的能量,只有电场才能改变带电粒子的能量。

假如场强在空间所有点上都一样,这样的场称为{\bf 均匀场}。场强为$\mbf{E}$的均匀电场,其标势应为
\begin{equation}
	\phi = -\mbf{E}\cdot \mbf{r}
	\label{chapter3:恒定均匀电场的标势}
\end{equation}
均匀磁场的矢势可以用场强$\mbf{H}$表示为
\begin{equation}
	\mbf{H} = \frac12 \mbf{H}\times \mbf{r}
	\label{chapter3:均匀磁场的矢势1}
\end{equation}
如果选取$z$轴沿着$\mbf{H}$的方向,均匀磁场的矢势也可以选择为以下形式:
\begin{equation}
	A_x = -Hy,\quad A_y = A_z = 0
	\label{chapter3:均匀磁场的矢势2}
\end{equation}
很容易发现,势\eqref{chapter3:均匀磁场的矢势1}和\eqref{chapter3:均匀磁场的矢势2}彼此之间的差为函数$f = -\dfrac12 xyH$的梯度。

\section{在恒定均匀电场中的运动}\label{chapter3:section:在恒定均匀电场中的运动}

现在研究电荷量为$e$的带电粒子在均匀的恒定电场$\mbf{E}$中的运动。取电场方向为$x$轴,运动显然在一个平面内进行,将其取为$xy$平面。这时,运动方程\eqref{chapter3:带电粒子在给定电磁场中的运动方程-Lorentz力}变为
\begin{equation*}
\begin{cases}
	\dot{p}_x = eE \\
	\dot{p}_y = 0
\end{cases}
\end{equation*}
所以可得
\begin{equation}
	p_x = eEt,\quad p_y = p_0
	\label{chapter3:恒定电场中带电粒子的动量}
\end{equation}
将时间参考点选择在$p_x=0$的时刻,$p_0$表示粒子在该时刻的动量。据此,可求得粒子的机械能为
\begin{equation}
	\E_m = \sqrt{m^2c^4+p^2} = \sqrt{m^2c^4+c^2p_0^2+(ceEt)^2} = \sqrt{\E_0^2+(ceEt)^2}
	\label{chapter3:恒定电场中带电粒子的机械能}
\end{equation}
其中$\E_0$是$t=0$时的机械能。根据式\eqref{chapter2:自由实物粒子能量、动量和速度之间的关系}可得
\begin{equation*}
	\frac{\mathd x}{\mathd t} = \frac{p_xc^2}{\E_m} = \frac{c^2eEt}{\sqrt{\E_0^2+(ceEt)^2}}
\end{equation*}
积分可得
\begin{equation}
	x = \frac{1}{eE}\sqrt{\E_0^2+(ceEt)^2}
	\label{chapter3:均匀恒定电场中带电粒子的x坐标随时间关系}
\end{equation}
其中我们已经令积分常数等于零。为了求得$y$,考虑
\begin{equation*}
	\frac{\mathd y}{\mathd t} = \frac{p_yc^2}{\E_m} = \frac{p_0c^2}{\sqrt{\E_0^2+(ceEt)^2}}
\end{equation*}
积分可得
\begin{equation}
	y = \frac{p_0c}{eE}\arsinh \frac{ceEt}{\E_0}
	\label{chapter3:均匀恒定电场中带电粒子的y坐标随时间关系}
\end{equation}

联立式\eqref{chapter3:均匀恒定电场中带电粒子的x坐标随时间关系}和\eqref{chapter3:均匀恒定电场中带电粒子的y坐标随时间关系},即可得到轨道方程为
\begin{equation}
	x = \frac{\E_0}{eE}\cosh \frac{eEy}{p_0c}
	\label{chapter3:均匀恒定电场中带电粒子的轨迹方程}
\end{equation}
由此可见,均匀电场中的一个带电粒子沿着悬链线运动。

在低速近似下,可令$p_0=mv_0, \E_0 = mc^2$,并将式\eqref{chapter3:均匀恒定电场中带电粒子的轨迹方程}展开为$\dfrac1c$的幂级数并略去高阶项,可得
\begin{equation*}
	x = \frac{eE}{2mv_0^2}y^2+\text{const}
\end{equation*}
即电荷在低速近似下沿抛物线运动,与经典力学中的结果一致。

\section{在恒定均匀磁场中的运动}

本节研究电荷量为$e$的带电粒子在均匀的恒定磁场$\mbf{H}$中的运动。取磁场的方向为$z$轴方向,则有
\begin{equation*}
	\frac{\mathd \mbf{p}}{\mathd t} = \dfrac{e}{c}\mbf{v} \times \mbf{H}
\end{equation*}
将速度和动量的关系$\mbf{v} = \dfrac{c^2\mbf{p}}{\E}$代入,由于磁场不改变粒子的能量,可得
\begin{equation}
	\frac{\mathd \mbf{v}}{\mathd t} = \frac{ce}{\E}\mbf{v}\times \mbf{H}
	\label{chapter3:带电粒子在恒定均匀磁场中的运动方程}
\end{equation}
记$\omega = \dfrac{ceH}{\E}$,将方程\eqref{chapter3:带电粒子在恒定均匀磁场中的运动方程}写成分量形式,则有
\begin{subnumcases}{}
	\ds \dot{v}_x = \omega v_y \label{chapter3:带电粒子在恒定均匀磁场中的分量方程1} \\
	\ds \dot{v}_y = -\omega v_x \label{chapter3:带电粒子在恒定均匀磁场中的分量方程2} \\
	\ds \dot{v}_z = 0 \label{chapter3:带电粒子在恒定均匀磁场中的分量方程3}
\end{subnumcases}
由式\eqref{chapter3:带电粒子在恒定均匀磁场中的分量方程3}可以解得
\begin{equation}
	v_z = v_{0z}\quad \text{(常数)}
	\label{chapter3:带电粒子在恒定均匀磁场中的z速度}
\end{equation}
再由式\eqref{chapter3:带电粒子在恒定均匀磁场中的分量方程1}和式\eqref{chapter3:带电粒子在恒定均匀磁场中的分量方程2}消去$v_y$可得方程
\begin{equation*}
	\ddot{v}_x + \omega^2 v_x = 0
\end{equation*}
这个方程的解为
\begin{equation}
	v_x = v_{0t}\cos(\omega t+\alpha)
	\label{chapter3:带电粒子在恒定均匀磁场中的x速度}
\end{equation}
同时可得
\begin{equation}
	v_y = -v_{0t}\sin(\omega+\alpha)
	\label{chapter3:带电粒子在恒定均匀磁场中的y速度}
\end{equation}
常数$v_{0t}$和初始相位$\alpha$都是由初始条件来决定的。由于
\begin{equation*}
	v_{0t} = \sqrt{v_x^2+v_y^2}
\end{equation*}
即,$v_{0t}$是粒子在$xy$平面内速度的分量大小。由于运动过程中粒子的能量是不变的,故这个分量也是不变的。

再次积分式\eqref{chapter3:带电粒子在恒定均匀磁场中的x速度}、式\eqref{chapter3:带电粒子在恒定均匀磁场中的y速度}和式\eqref{chapter3:带电粒子在恒定均匀磁场中的z速度},可得
\begin{equation}
\begin{cases}
	x = x_0 + r \sin(\omega t+\alpha) \\
	y = y_0 + r \cos(\omega t+\alpha) \\
	z = z_0 + v_{0z}t
\end{cases}
\label{chapter3:带电粒子在恒定均匀磁场中运动的轨迹方程}
\end{equation}
其中
\begin{equation}
	r = \dfrac{v_{0t}}{\omega} = \dfrac{v_{0t}\E}{ceH} = \frac{cp_t}{eH}
	\label{chapter3:带电粒子在恒定均匀磁场中运动的轨道半径}
\end{equation}
式\eqref{chapter3:带电粒子在恒定均匀磁场中运动的轨迹方程}表明,带电粒子在恒定均匀磁场中沿螺旋线运动,螺旋线的轴沿磁场方向,螺旋线的半径由式\eqref{chapter3:带电粒子在恒定均匀磁场中运动的轨道半径}决定,在运动过程中,粒子的速度保持不变,其旋转的角速度为$\omega$。

当$v_{0z}=0$,即粒子沿着磁场方向的速度分量为零的特殊情况下,粒子将在与磁场垂直的平面内作匀速圆周运动。此时$\omega$即为粒子圆周运动的角速度。

\begin{example}
将一个带电的空间谐振子置于均匀磁场内,这个振子的本征频率为$\omega_0$。求它在磁场中的振动频率。
\end{example}
\begin{solution}
设磁场方向为$z$方向,因此振子的受迫振动方程为\footnote{此处应用的是经典结论。}
\begin{subnumcases}{}
	\ds \ddot{x} + \omega_0^2 x = \frac{eH}{mc}\dot{y} \label{chapter3:恒定均匀磁场中的带电谐振子方程1} \\
	\ds \ddot{y} + \omega_0^2 y = -\frac{eH}{mc}\dot{x} \label{chapter3:恒定均匀磁场中的带电谐振子方程2} \\
	\ds \ddot{z} + \omega_0^2 z = 0 \label{chapter3:恒定均匀磁场中的带电谐振子方程3}
\end{subnumcases}
将式\eqref{chapter3:恒定均匀磁场中的带电谐振子方程2}乘以虚数单位$\mathi$并与式\eqref{chapter3:恒定均匀磁场中的带电谐振子方程1}相加,令$\xi = x+\i y$,则可得
\begin{equation*}
	\ddot{\xi} + \mathi \frac{eH}{mc} \dot{\xi} + \omega_0^2\xi = 0
\end{equation*}
这个微分方程的通解为
\begin{equation*}
	\xi = A\mathe^{\mathi \left(-\frac{eH}{2mc}+\sqrt{\omega_0^2+\frac14\left(\frac{eH}{mc}\right)^2}\right) t} + B\mathe^{\mathi \left(-\frac{eH}{2mc}-\sqrt{\omega_0^2+\frac14\left(\frac{eH}{mc}\right)^2}\right) t}
\end{equation*}
其中$A, B$是任意的复常数。于是有\small
\begin{align*}
	\ds x & = \Re \xi \\
	& = A_0\cos\left[\left(\sqrt{\omega_0^2 + \frac14\left(\frac{eH}{mc}\right)^2} - \frac{eH}{2mc}\right) t + \phi_1\right] + B_0\cos\left[\left(\sqrt{\omega_0^2 + \frac14\left(\frac{eH}{mc}\right)^2} + \frac{eH}{2mc}\right) t + \phi_2\right] \\
	\ds y & = \Im \xi \\
	& = A_0\sin\left[\left(\sqrt{\omega_0^2 + \frac14\left(\frac{eH}{mc}\right)^2} - \frac{eH}{2mc}\right) t + \phi_1\right] + B_0\sin\left[\left(\sqrt{\omega_0^2 + \frac14\left(\frac{eH}{mc}\right)^2} + \frac{eH}{2mc}\right) t + \phi_2\right]
\end{align*}
\normalsize 由此可得振子在与磁场垂直的平面内的振动频率即为
\begin{equation*}
	\omega = \sqrt{\omega_0^2 + \frac14\left(\frac{eH}{mc}\right)^2} \pm \frac{eH}{2m}
\end{equation*}
磁场很弱的时候,这个结果变为
\begin{equation*}
	\omega \approx \omega_0 \pm \frac{eH}{2mc}
\end{equation*}
\end{solution}

\section{带电粒子在均匀恒定的电场和磁场中的运动}

本节研究在电场和磁场都存在并且是均匀恒定的情况下,一个带电粒子如何运动。这些讨论仅局限于粒子的速度远小于光速,即$v\ll c$的情形,因此质点的动量为$\mbf{p}=m\mbf{v}$。

选择磁场$\mbf{H}$的方向为$z$轴方向,矢量$\mbf{H}$和$\mbf{E}$决定的平面为$yz$平面。这时,带电粒子的运动方程为
\begin{equation}
	m\dot{\mbf{v}} = e\mbf{E} + \frac{e}{c}\mbf{v}\times \mbf{H}
	\label{chapter3:均匀恒定电场和磁场中带电粒子的运动方程}
\end{equation}
其分量形式可以写作
\begin{subnumcases}{}
	m\ddot{x} = \dfrac{e}{c}H\dot{y} \label{chapter3:均匀恒定电场和磁场中带电粒子的运动方程分量形式1} \\
	m\ddot{y} = eE_y - \dfrac{e}{c}H\dot{x} \label{chapter3:均匀恒定电场和磁场中带电粒子的运动方程分量形式2} \\
	m\ddot{z} = eE_z \label{chapter3:均匀恒定电场和磁场中带电粒子的运动方程分量形式3}
\end{subnumcases}
由式\eqref{chapter3:均匀恒定电场和磁场中带电粒子的运动方程分量形式3}可得,带电粒子以恒定加速度沿着$z$轴方向运动,即
\begin{equation}
	z = \frac{eE_z}{2m}t^2+v_{0z}t
\end{equation}
此处已经令积分常数为零。由此,后面可以不考虑这个方向的运动,即考虑带电粒子在正交的电场和磁场中的运动。

用$\mathi$乘以式\eqref{chapter3:均匀恒定电场和磁场中带电粒子的运动方程分量形式2}然后与式\eqref{chapter3:均匀恒定电场和磁场中带电粒子的运动方程分量形式1}相加,并令$\xi = x+\mathi y$可得
\begin{equation}
	\ddot{\xi} + \mathi \omega\dot{\xi} = \mathi \frac{eE_y}{m}
\end{equation}
式中$\omega = \dfrac{eH}{mc}$,其通解为
\begin{equation}
	\dot{\xi} = a\mathe^{-\mathi\omega t}+\frac{cE_y}{H}
\end{equation}
其中$a$是任意复常数。将$a$写成$a=b\mathe^{\mathi \alpha}$的形式,其中$b$和$\alpha$为实数,这说明,适当选择时间原点可以使得复常数$a$的虚部为零。我们这样选取时间原点,使得$a$为实数,由此即有
\begin{equation}
\begin{cases}
	\dot{x} = a\cos\omega t+\dfrac{cE_y}{H} \\
	\dot{y} = -a\sin\omega t
\end{cases}
\label{chapter3:均匀恒定电场和磁场中带电粒子的速度}
\end{equation}
当$t=0$时,速度沿着$x$轴方向。

由此可以看出,粒子的速度是时间的周期函数,它们的平均值为
\begin{equation*}
	\bar{\dot{x}} = \frac{cE_y}{H},\quad \bar{\dot{y}} = 0
\end{equation*}
带电粒子在正交的电场和磁场中运动的这个平均速度称为{\bf 电漂移速度},它的方向与两个场都垂直且与电荷的正负无关,它可以改写为矢量形式:
\begin{equation}
	\bar{\mbf{v}} = \frac{c\mbf{E}\times \mbf{H}}{H^2}
	\label{chapter3:电漂移速度}
\end{equation}

这一节的所有公式都假设了粒子的速度远小于光速,即电场和磁场必须满足
\begin{equation}
	\frac{E_y}{H} \ll 1
\end{equation}
即电场强度远小于磁场强度,而$E_y$和$H$的绝对大小是任意的。

将方程\eqref{chapter3:均匀恒定电场和磁场中带电粒子的速度}再积分一次,并使得当$t=0$时,$x=y=0$,则有
\begin{equation}
\begin{cases}
	\ds x = \frac{a}{\omega}\sin\omega t+\frac{cE_y}{H}t \\
	\ds y = \frac{a}{\omega}(\cos \omega t - 1)
\end{cases}
\label{chapter3:均匀恒定电场和磁场中带电粒子的运动轨迹}
\end{equation}

将式\eqref{chapter3:均匀恒定电场和磁场中带电粒子的运动轨迹}看作一条曲线的参数方程,则它们定义了一条次摆线,如图\ref{chapter3:外摆线的情形}、\ref{chapter3:内摆线的情形}和\ref{chapter3:摆线的情形}所示,其具体的形状取决于$|a|$和$\dfrac{cE_y}{H}$之间的大小关系。

\begin{figure}[htb]
\centering
\begin{minipage}[t]{0.48\textwidth}
\begin{asy}
	texpreamble("\usepackage{xeCJK}");
	texpreamble("\setCJKmainfont{SimSun}");
	usepackage("amsmath");
	import graph;
	size(180);
	//
	real x1,x2,y;
	real a,b,omega;
	pair f(real t){
		return (a/omega*sin(omega*t)+b*t,a/omega*(cos(omega*t)-1));
	}
	x1 = -1;
	x2 = 6.5;
	y = 2.5;
	draw(Label("$x$",EndPoint),(x1,0)--(x2,0),Arrow);
	draw(Label("$y$",EndPoint),(0,0)--(0,y),Arrow);
	a = -1;
	b = 0.7;
	omega = 1;
	path p;
	p = graph(f,-2.5,2*pi+2.5);
	draw(p,linewidth(0.8bp));
	add(arrow(p,invisible,FillDraw(black),Relative(0.1)));
	add(arrow(p,invisible,FillDraw(black),Relative(0.93)));
	add(arrow(p,invisible,FillDraw(black),Relative(0.53)));
\end{asy}
\caption{外摆线:$|a|>\dfrac{cE_y}{H}$的情形}
\label{chapter3:外摆线的情形}
\end{minipage}
\hspace{0.1cm}
\begin{minipage}[t]{0.48\textwidth}
\begin{asy}
	texpreamble("\usepackage{xeCJK}");
	texpreamble("\setCJKmainfont{SimSun}");
	usepackage("amsmath");
	import graph;
	size(180);
	//
	real x1,x2,y;
	real a,b,omega;
	pair f(real t){
		return (a/omega*sin(omega*t)+b*t,a/omega*(cos(omega*t)-1));
	}
	x1 = -1;
	x2 = 6.5;
	y = 2.5;
	draw(Label("$x$",EndPoint),(x1,0)--(x2,0),Arrow);
	draw(Label("$y$",EndPoint),(0,0)--(0,y),Arrow);
	a = -0.5;
	b = 0.8;
	omega = 1;
	path p;
	p = graph(f,-2,2*pi+2);
	draw(p,linewidth(0.8bp));
	add(arrow(p,invisible,FillDraw(black),Relative(0.1)));
	add(arrow(p,invisible,FillDraw(black),Relative(0.93)));
	add(arrow(p,invisible,FillDraw(black),Relative(0.53)));
\end{asy}
\caption{内摆线:$|a|<\dfrac{cE_y}{H}$的情形}
\label{chapter3:内摆线的情形}
\end{minipage}\\

\begin{minipage}[t]{0.48\textwidth}
\begin{asy}
	texpreamble("\usepackage{xeCJK}");
	texpreamble("\setCJKmainfont{SimSun}");
	usepackage("amsmath");
	import graph;
	size(180);
	//
	real x1,x2,y;
	real a,b,omega;
	pair f(real t){
		return (a/omega*sin(omega*t)+b*t,a/omega*(cos(omega*t)-1));
	}
	x1 = -1;
	x2 = 6.5;
	y = 2.5;
	draw(Label("$x$",EndPoint),(x1,0)--(x2,0),Arrow);
	draw(Label("$y$",EndPoint),(0,0)--(0,y),Arrow);
	a = -0.8;
	b = 0.8;
	omega = 1;
	path p;
	p = graph(f,-2,2*pi+2);
	draw(p,linewidth(0.8bp));
	add(arrow(p,invisible,FillDraw(black),Relative(0.1)));
	add(arrow(p,invisible,FillDraw(black),Relative(0.93)));
	add(arrow(p,invisible,FillDraw(black),Relative(0.53)));
\end{asy}
\caption{摆线:$|a|=\dfrac{cE_y}{H}$的情形}
\label{chapter3:摆线的情形}
\end{minipage}
\end{figure}

如果$a=-\dfrac{cE_y}{H}$,那么式\eqref{chapter3:均匀恒定电场和磁场中带电粒子的运动轨迹}变为
\begin{equation}
\begin{cases}
	x = \dfrac{cE_y}{\omega H}(\omega t-\sin \omega t) \\[1.5ex]
	y = \dfrac{cE_y}{\omega H}(1-\cos\omega t)
\end{cases}
\end{equation}
此时,轨道在$xy$平面上的投影是一条摆线,如图\ref{chapter3:摆线的情形}所示。

\begin{example}
求带电粒子在平行均匀电场和磁场中的相对论运动。
\end{example}
\begin{solution}
取磁场$\mbf{H}$的方向为$z$轴方向,由于磁场对$z$轴方向上的运动没有影响,因而$z$方向上的运动只在电场的作用下发生,于是根据第\ref{chapter3:section:在恒定均匀电场中的运动}节的结果,可得
\begin{equation}
	z = \frac{\E_m}{eE} = \frac{1}{eE}\sqrt{\E_0^2+(ceEt)^2}
	\label{chapter3:例3.2在z方向上的运动}
\end{equation}

对于在$xy$平面内的运动,运动方程为
\begin{equation*}
\begin{cases}
	\dot{p}_x = \dfrac{e}{c}Hv_y \\[1.5ex]
	\dot{p}_y = -\dfrac{e}{c}Hv_x
\end{cases}
\end{equation*}
将其转化为复变数的方程可得
\begin{equation}
	\frac{\mathd}{\mathd t}(p_x+\mathi p_y) = -\mathi \dfrac{eH}{c}(v_x+\mathi v_y)
	\label{chapter3:例3.2动量满足的方程}
\end{equation}
由式\eqref{chapter2:自由实物粒子能量、动量和速度之间的关系}可得
\begin{equation*}
	v_x = \frac{c^2p_x}{\E_m},\quad v_y = \frac{c^2p_y}{\E_m}
\end{equation*}
其中$\E_m = \sqrt{\E_0^2+(ceEt)^2}$。如果记$p=p_x+\mathi p_y$,则方程\eqref{chapter3:例3.2动量满足的方程}化为
\begin{equation}
	\frac{\mathd p}{\mathd t} = -\mathi \frac{ceH}{\E_m}p = -\mathi \frac{ceH}{\sqrt{\E_0^2+(ceEt)^2}}p
	\label{chapter3:例3.2复动量满足的方程}
\end{equation}
将方程\eqref{chapter3:例3.2复动量满足的方程}分离变量并积分,即有
\begin{equation*}
	\int_{p_t}^p \frac{\mathd p}{p} = -\int_0^t \mathi \frac{ceH}{\sqrt{\E_0^2+(ceEt)^2}} \mathd t
\end{equation*}
其中$p_t$为动量在$xy$平面上投影的恒定值。由此则有
\begin{equation*}
	p = p_t\mathe^{-\mathi \frac{H}{E}\arsinh \frac{ceEt}{\E_0}}
\end{equation*}
方便起见,引入一个辅助量$\psi$,它满足
\begin{equation*}
	\psi = \frac{H}{E}\arsinh \frac{ceEt}{\E_0}
\end{equation*}
或者
\begin{equation}
	ct = \frac{\E_0}{eE}\sinh \frac{E}{H}\psi
	\label{chapter3:例3.2辅助量psi与时间的关系}
\end{equation}
它的微分满足关系
\begin{equation*}
	\mathd \psi = \frac{ceH}{\sqrt{\E_0^2+(ceEt)^2}} \mathd t
\end{equation*}
由此即有
\begin{equation*}
	p = p_x+\mathi p_y = p_t\mathe^{-\mathi \psi}
\end{equation*}
再利用式\eqref{chapter2:自由实物粒子能量、动量和速度之间的关系}可得
\begin{equation*}
	p_x+\mathi p_y = \frac{\E_m\dot{x}}{c^2} + \mathi \frac{\E_m\dot{y}}{c^2} = \frac{\E_m}{c^2} \left(\frac{\mathd x}{\mathd \psi} \frac{\mathd \psi}{\mathd t} + \mathi \frac{\mathd y}{\mathd \psi} \frac{\mathd \psi}{\mathd t}\right) = \dfrac{eH}{c}\left(\frac{\mathd x}{\mathd \psi} + \mathi \frac{\mathd y}{\mathd \psi}\right)
\end{equation*}
即有
\begin{equation*}
	\dfrac{eH}{c}\left(\frac{\mathd x}{\mathd \psi} + \mathi \frac{\mathd y}{\mathd \psi}\right) = p_t\mathe^{-\mathi \psi}
\end{equation*}
从中可以解得
\begin{equation}
	x = \frac{cp_t}{eH}\sin \psi,\quad y = \frac{cp_t}{eH}\cos\psi
	\label{chapter3:例3.2在xy平面内的运动}
\end{equation}
由式\eqref{chapter3:例3.2在z方向上的运动}和式\eqref{chapter3:例3.2辅助量psi与时间的关系}可得
\begin{equation}
	z = \frac{\E_0}{eE}\cosh\frac{E}{H}\psi
	\label{chapter3:例3.2在z方向上的运动-辅助量表示}
\end{equation}
式\eqref{chapter3:例3.2在xy平面内的运动}和式\eqref{chapter3:例3.2在z方向上的运动-辅助量表示}一起以参数形式决定了粒子的运动,其轨道为螺旋线,半径为$\dfrac{cp_t}{eH}$,步长单调增加,粒子沿着螺旋线以减小的角速度$\omega = \dfrac{ceH}{\E_m}$运动,其沿$z$轴的速度趋向于$c$。
\end{solution}

\begin{example}
求带电粒子在互相垂直且数值相等的电场和磁场中的相对论运动。\footnote{互相垂直而数值不等的$\mbf{E}$和$\mbf{H}$场中的运动问题,可以通过适当的参考系变换,化为纯电场或纯磁场中运动的问题,见第\ref{chapter3:section:场的不变量}节。}
\end{example}
\begin{solution}
取$\mbf{H}$的方向为$z$方向,$\mbf{E}$的方向为$y$方向,并考虑到$E=H$可有运动方程为
\begin{subnumcases}{}
	\dfrac{\mathd p_x}{\mathd t} = \frac{e}{c}Ev_y \label{chapter3:例3.3运动方程1} \\
	\dfrac{\mathd p_y}{\mathd t} = eE-\frac{e}{c}Ev_x \label{chapter3:例3.3运动方程2} \\
	\dfrac{\mathd p_z}{\mathd t} = 0 \label{chapter3:例3.3运动方程3} 
\end{subnumcases}
由方程\eqref{chapter3:例3.3运动方程3}可得
\begin{equation}
	p_z = p_{0z}\quad \text{(常数)}
	\label{chapter3:例3.3-z方向动量}
\end{equation}
再由式\eqref{chapter3:电磁场中带电粒子能量的变化}可得
\begin{equation*}
	\frac{\mathd \E_m}{\mathd t} = eEv_y
\end{equation*}
即有
\begin{equation}
	\E_m-cp_x = \alpha\quad \text{(常数)}
	\label{chapter3:例3.3守恒定律1}
\end{equation}
由式\eqref{chapter2:自由实物粒子能量与动量之间的关系}可得
\begin{equation*}
	\E_m^2 = c^2p_x^2+c^2p_y^2+c^2p_z^2+m^2c^4
\end{equation*}
令$\eps^2 = m^2c^4+c^2p_z^2$(常数),可有
\begin{equation*}
	\E_m^2-c^2p_x^2 = (\E_m+cp_x)(\E_m-cp_x) = c^2p_y^2+\eps^2
\end{equation*}
所以可得
\begin{equation}
	\E_m+cp_x = \frac{1}{\alpha}(c^2p_y^2+\eps^2)
	\label{chapter3:例3.3能量动量关系1}
\end{equation}
联立式\eqref{chapter3:例3.3守恒定律1}和式\eqref{chapter3:例3.3能量动量关系1}可得
\begin{subnumcases}{}
	\ds \E_m = \frac{\alpha}{2}+\frac{c^2p_y^2+\eps^2}{2\alpha} \label{chapter3:例3.3机械能和y方向动量的关系} \\
	\ds p_x = -\frac{\alpha}{2c}+\frac{c^2p_y^2+\eps^2}{2\alpha c} \label{chapter3:例3.3x方向动量和y方向动量的关系}
\end{subnumcases}
进一步可有
\begin{equation*}
	\E_m \frac{\mathd p_y}{\mathd t} = eE\left(\E_m - \frac{\E_mv_x}{c}\right) = eE(\E_m-cp_x) = eE\alpha
\end{equation*}
结合式\eqref{chapter3:例3.3机械能和y方向动量的关系}可得
\begin{equation}
	2eEt = \left(1+\frac{\eps^2}{\alpha^2}\right)p_y + \frac{c^2}{3\alpha^2} p_y^3
	\label{chapter3:例3.3时间与y方向动量的关系}
\end{equation}

下面,为了确定轨道方程,首先在方程
\begin{equation}
	\frac{\mathd x}{\mathd t} = \frac{c^2p_x}{\E_m},\quad \frac{\mathd y}{\mathd t} = \frac{c^2p_y}{\E_m},\quad \frac{\mathd z}{\mathd t} = \frac{c^2p_z}{\E_m}
\end{equation}
中,作变量代换
\begin{equation*}
	\mathd t = \frac{\E_m}{eE\alpha}\mathd p_y
\end{equation*}
可得以$p_y$为参数的轨道方程
\begin{equation}
\begin{cases}
	\ds x = \frac{c}{2eE}\left(-1+\frac{\eps^2}{\alpha^2}\right)p_y + \frac{c^3}{6\alpha^2eE}p_y^3 \\[1.5ex]
	\ds y = \frac{c^2}{2\alpha eE}p_y^2 \\[1.5ex]
	\ds z = \frac{p_zc^2}{eE\alpha}p_y
\end{cases}
\end{equation}
\end{solution}

\section{电磁场张量}

在第\ref{chapter3:section:场中带电粒子的运动方程}节中,我们从三维形式的Lagrange函数\eqref{chapter3:电磁场中带电粒子的Lagrange函数}导出了带电粒子在场内的运动方程。本节将直接从四维形式的作用量\eqref{chapter3:电磁场中带电粒子的作用量1}中导出同样的方程。

Hamilton原理给出:
\begin{equation}
	\delta S = \delta \int_a^b \left(-mc\mathd s-\frac{e}{c}A_i\mathd x^i\right)
	\label{chapter3:电磁场张量-电磁场的Hamilton原理1}
\end{equation}
注意到$\mathd s^2 = \mathd x_i\mathd x^i$,可求得
\begin{equation*}
	\delta S = -\int_a^b \left(mc\frac{\mathd x_i\mathd \delta x^i}{\mathd s} + \frac{e}{c}A_i\mathd \delta x^i + \frac{e}{c}\delta A_i\mathd x^i\right)
\end{equation*}
分部积分可得
\begin{align}
	\delta S & = -\int_a^b \left(mu_i\mathd \delta x^i + \frac{e}{c}A_i\mathd \delta x^i + \frac{e}{c}\delta A_i\mathd x^i\right) \nonumber \\
	& = \int_a^b \left(m\mathd u_i \delta x^i + \frac{e}{c}\frac{\pl A_i}{\pl x^k}\mathd x^k\delta x^i - \frac{e}{c} \frac{\pl A_i}{\pl x^k}\mathd x^i \delta x^k\right) - \left(mu_i+\frac{e}{c}A_i\right)\delta x^i\bigg|_a^b \nonumber \\
	& = \frac1c \int_a^b \left[m\frac{\mathd u_i}{\mathd \tau} - \frac{e}{c}\left(\frac{\pl A_k}{\pl x^i} - \frac{\pl A_i}{\pl x^k}\right) u^k\right] \delta x^i\mathd s - \left(mu_i+\frac{e}{c}A_i\right)\delta x^i\bigg|_a^b
	\label{chapter3:电磁场张量-电磁场的Hamilton原理2}
\end{align}

在固定边界条件下,式\eqref{chapter3:电磁场张量-电磁场的Hamilton原理2}的第二项为零,由于$\delta x^i$的任意性,可得式\eqref{chapter3:电磁场张量-电磁场的Hamilton原理2}的第一项中被积函数必须为零,即
\begin{equation*}
	m\frac{\mathd u_i}{\mathd \tau} - \frac{e}{c}\left(\frac{\pl A_k}{\pl x^i} - \frac{\pl A_i}{\pl x^k}\right) u^k = 0
\end{equation*}

现在我们引入下面的符号:
\begin{equation}
	F_{ik} = \frac{\pl A_k}{\pl x^i} - \frac{\pl A_i}{\pl x^k}
	\label{chapter3:电磁场张量的定义}
\end{equation}
其中反对称张量$F_{ik}$称为{\bf 电磁场张量},由此,运动方程具有下面的形式
\begin{equation}
	m\frac{\mathd u_i}{\mathd \tau} = \frac{e}{c}F_{ik}u^k
	\label{chapter3:带电粒子在电磁场中运动方程的四维形式}
\end{equation}

将$A_i=(\phi,-\mbf{A})$代入式\eqref{chapter3:电磁场张量的定义},可以得出张量$F_{ik}$各分量与场量之间的联系。其结果可以写成一个矩阵,其中指标$i=0,1,2,3$表示行,$k=0,1,2,3$表示列,即
\begin{equation}
	F_{ik} = \begin{pmatrix} 0 & E_x & E_y & E_z \\
							-E_x & 0 & -H_z & H_y \\
							-E_y & H_z & 0 & -H_x \\
							-E_z & -H_y & H_x & 0 \end{pmatrix},\quad
	F^{ik} = \begin{pmatrix} 0 & -E_x & -E_y & -E_z \\
							E_x & 0 & -H_z & H_y \\
							E_y & H_z & 0 & -H_x \\
							E_z & -H_y & H_x & 0 \end{pmatrix}
	\label{chapter3:电磁场张量的分量形式}
\end{equation}
这个结果也可以简单地写作
\begin{equation}
	F_{ik} = (\mbf{E},\mbf{H}),\quad F^{ik} = (-\mbf{E},\mbf{H})
\end{equation}
由此可见,电场强度与磁场强度的分量是同一四维电磁场张量的分量。

容易验证,运动方程\eqref{chapter3:带电粒子在电磁场中运动方程的四维形式}的三个空间分量($i=1,2,3$)与矢量运动方程\eqref{chapter3:带电粒子在给定电磁场中的运动方程-Lorentz力}相同,而时间分量($i=0$)则给出功方程\eqref{chapter3:电磁场中带电粒子能量的变化}。后者是运动方程的推论。

假如在变分$\delta S$中,只考虑真实的轨道,那么式\eqref{chapter3:电磁场张量-电磁场的Hamilton原理2}的第一项就恒等于零。此时积分上限是变化的,由此可得作用量作为坐标函数的微分为
\begin{equation}
	\delta S = -\left(mu_i+\frac{e}{c}A_i\right)\delta x^i
	\label{chapter3:作用量对坐标的微分}
\end{equation}
由此可得
\begin{equation}
	-\frac{\pl S}{\pl x^i} = mu_i + \frac{e}{c}A_i = p_i+\frac{e}{c}A_i
	\label{chapter3:电磁场中带电粒子的四维动量}
\end{equation}
四维矢量$-\dfrac{\pl S}{\pl x^i}$即为粒子的四维动量矢量$P_i$。将四维机械动量$p_i$和四维势$A_i$代入,可得
\begin{equation}
	P^i = \left(\frac{\E_m+e\phi}{c},\mbf{p} +\frac{e}{c}\mbf{A}\right)
	\label{chapter3:电磁场中带电粒子的四维动量矢量}
\end{equation}
这个四维矢量$P^i$的三个空间分量构成三维广义动量矢量\eqref{chapter3:电磁场中带电粒子的广义动量},其时间分量则为$\dfrac{\E}{c}$,其中$\E$是带电粒子在场中的总能量。

\section{场的Lorentz变换}

四维势的变换公式可以根据式\eqref{chapter1:四维矢量协变分量的变换关系式}得出,结合$A^i = (\phi,\mbf{A})$可得
\begin{equation}
	\phi = \frac{\phi'+\dfrac{V}{c}A'_x}{\sqrt{1-\dfrac{V^2}{c^2}}},\quad
	A_x = \frac{A'_x+\dfrac{V}{c}\phi'}{\sqrt{1-\dfrac{V^2}{c^2}}},\quad
	A_y = A'_y,\quad A_z = A'_z
\label{chapter3:四维势的变换}
\end{equation}

电磁场张量的变换公式可以根据式\eqref{chapter1:四维张量的变换规则}得出,由此可以得出电场的变换公式:
\begin{equation}
	E_x = E'_x,\quad E_y = \frac{E'_y+\dfrac{V}{c}H'_z}{\sqrt{1-\dfrac{V^2}{c^2}}},\quad E_z = \frac{E'_z-\dfrac{V}{c}H'_y}{\sqrt{1-\dfrac{V^2}{c^2}}}
	\label{chapter3:电场的变换}
\end{equation}
以及磁场的变换公式
\begin{equation}
	H_x = H'_x,\quad H_y = \frac{H'_y-\dfrac{V}{c}E'_z}{\sqrt{1-\dfrac{V^2}{c^2}}},\quad H_z = \frac{H'_z+\dfrac{V}{c}E'_y}{\sqrt{1-\dfrac{V^2}{c^2}}}
	\label{chapter3:磁场的变换}
\end{equation}

因此,电场和磁场与很多其他物理量一样,是相对的。它们在不同的参考系中有不同的特性。特别地讲,电场或磁场可以在某一个参考系中为零,而在另外的参考系中又存在。

变换式\eqref{chapter3:电场的变换}和\eqref{chapter3:磁场的变换}在$V\ll c$的情形下将大大简化,精确至$\dfrac{V}{c}$数量级,可得
\begin{align*}
	& E_x = E'_x,\quad E_y = E'_y+\frac{V}{c}H'_z,\quad E_z = E'_z-\frac{V}{c}H'_y \\
	& H_x = H'_x,\quad H_y = H'_y-\frac{V}{c}E'_z,\quad H_z = H'_z+\frac{V}{c}E'_y
\end{align*}
这些变换式可以用矢量形式表示为:
\begin{equation}
	\mbf{E} = \mbf{E}'+\frac1c \mbf{H}'\times \mbf{V},\quad \mbf{H} = \mbf{H}'-\frac1c \mbf{E}'\times \mbf{V}
	\label{chapter3:低速近似下的电磁场变换公式}
\end{equation}

以上公式相应的逆变换公式可以通过直接改变$V$的符号来获得。

如果在$K'$系中磁场$\mbf{H}'=\mbf{0}$,那么,根据式\eqref{chapter3:电场的变换}和\eqref{chapter3:磁场的变换}可以验证,在$K$系中,电场和磁场之间存在如下关系:
\begin{equation}
	\mbf{H} = \frac1c\mbf{V}\times \mbf{E}
\end{equation}
假如在$K'$系中,$\mbf{E}'=\mbf{0}$,那么在$K$系中,电场和磁场之间存在如下关系:
\begin{equation}
	\mbf{E} = -\frac1c\mbf{V}\times \mbf{H}
\end{equation}
在上面的两种情况下,电场与磁场在$K$系中都是互相垂直的。

这个结论是很有意义的,如果场$\mbf{E}$和$\mbf{H}$在某个参考系$K$中互相垂直(但数值不等),那么必然存在一个参考系$K'$,其中的场是纯电场或纯磁场。这个参考系相对于$K$的速度$\mbf{V}$垂直于$\mbf{E}$和$\mbf{H}$,当$H<E$时,其大小为$\dfrac{cH}{E}$;当$E<H$时,其大小为$\dfrac{cE}{H}$。

\section{场的不变量}\label{chapter3:section:场的不变量}

从电磁场张量可以构造出一些不变量,这些不变量在从一个惯性系变换到另一个惯性系时保持不变。

从场的四维表示出发,用反对称四维张量$F^{ik}$可以得出这些不变量的形式。由这个张量的分量可以构造如下不变量:
\begin{align}
	F_{ik}F^{ik} & = \text{不变量} \label{chapter3:场的不变量1-四维形式} \\
	e^{iklm}F_{ik}F_{lm} & = \text{不变量} \label{chapter3:场的不变量2-四维形式}
\end{align}
不变量\eqref{chapter3:场的不变量1-四维形式}是一个标量,而不变量\eqref{chapter3:场的不变量2-四维形式}则是赝标量(张量$F^{ik}$同其对偶张量的乘积\footnote{赝标量\eqref{chapter3:场的不变量2-四维形式}也可以表示为一个四维散度
\begin{equation*}
	e^{iklm}F_{ik}F_{lm} = 4\frac{\pl}{\pl x^i}\left(e^{iklm}A_k\frac{\pl A_m}{\pl x^l}\right)
\end{equation*}
})。

利用式\eqref{chapter3:电磁场张量的分量形式}可以将不变量用场变量$\mbf{E}$和$\mbf{H}$表示出来,在三维形式下可有
\begin{align}
	H^2-E^2 & = \text{不变量} \label{chapter3:场的不变量1} \\
	\mbf{E} \cdot \mbf{H} & = \text{不变量} \label{chapter3:场的不变量2}
\end{align}
其中第二个不变量是极矢量$\mbf{E}$和轴矢量$\mbf{H}$的乘积,因而是一个赝标量。

由不变量\eqref{chapter3:场的不变量2}可知,如果电场和磁场在某一惯性系中互相垂直,即$\mbf{E}\cdot\mbf{H}=0$,那么它们在其他所有惯性系中也垂直。由不变量\eqref{chapter3:场的不变量1}可知,如果电场和磁场在某一惯性系中相等,即$H=E$,那么它们在任何其他所有惯性系中也相等。

如果在某一参考系中$E>H$(或$H>E$),那么在其它所有惯性系中也有$E>H$(或$H>E$)。如果在某个参考系中矢量$\mbf{E}$和$\mbf{H}$成锐角(或钝角),那么它们在每个其它参考系也成锐角(或钝角)。

借助Lorentz变换,我们总可以给予$\mbf{E}$和$\mbf{H}$任何给定的值,唯一的附加条件是$E^2-H^2$和$\mbf{E}\cdot\mbf{H}$具有固定值。特别地,总可以找到一个惯性系,其中电场和磁场在给定点彼此平行。在这个参考系中$\mbf{E}\cdot\mbf{H}=EH$,并且由
\begin{equation*}
	E^2-H^2=E_0^2-H_0^2,\quad EH = \mbf{E}_0\cdot\mbf{H}_0
\end{equation*}
可以求出$\mbf{E}$和$\mbf{H}$在这个参考系中的值($\mbf{E}_0$和$\mbf{H}_0$是在原来参考系中的电场和磁场)。

在两个不变量都为零的情形下,$\mbf{E}$和$\mbf{H}$在所有的参考系中都相等并且互相垂直。

如果$\mbf{E}\cdot\mbf{H}=0$,我们总可以找到一个参考系,其中$\mbf{E}=\mbf{0}$或者$\mbf{H}=\mbf{0}$(依$E^2-H^2$的正负而定),即为纯电场或纯磁场。反之,如果在某一坐标系中$\mbf{E}=\mbf{0}$或者$\mbf{H}=\mbf{0}$,则它们在每个其它参考系中都互相垂直。
\chapter{电磁场方程}

\section{第一对Maxwell方程}

从电场$\mbf{E}$和磁场$\mbf{H}$的定义式
\begin{equation*}
	\mbf{E} = -\frac1c\frac{\pl \mbf{A}}{\pl t}-\bnb \phi,\quad \mbf{H} = \bnb \times \mbf{A}
\end{equation*}
很容易得到仅含有场变量的方程。先求
\begin{equation*}
	\bnb \times \mbf{E} = -\frac1c \frac{\pl}{\pl t}(\bnb \times \mbf{A}) - \bnb \times \bnb \phi
\end{equation*}
由于$\bnb \times \bnb \phi=\mbf{0}$,所以有
\begin{equation}
	\bnb \times \mbf{E} = -\frac1c \frac{\pl \mbf{H}}{\pl t}
	\label{chapter4:电磁场方程-第一个Maxwell方程}
\end{equation}
再求$\bnb \cdot\mbf{H}$可有
\begin{equation}
	\bnb \cdot \mbf{H} = \mbf{0}
	\label{chapter4:电磁场方程-第二个Maxwell方程}
\end{equation}
方程\eqref{chapter4:电磁场方程-第一个Maxwell方程}和\eqref{chapter4:电磁场方程-第二个Maxwell方程}称为第一对Maxwell方程。

方程\eqref{chapter4:电磁场方程-第一个Maxwell方程}和\eqref{chapter4:电磁场方程-第二个Maxwell方程}可以写成积分形式。根据Gauss定理,可有
\begin{equation*}
	\int \bnb \cdot \mbf{H}\mathd V = \oint \mbf{H} \cdot \mathd \mbf{f}
\end{equation*}
由方程\eqref{chapter4:电磁场方程-第二个Maxwell方程}可得
\begin{equation}
	\oint \mbf{H}\cdot \mathd \mbf{f} = 0
	\label{chapter4:电磁场方程-第二个Maxwell方程的积分形式}
\end{equation}
即磁场通过每个封闭曲面的通量为零。

根据Stokes定理,可有
\begin{equation*}
	\int \bnb \times \mbf{E}\cdot \mathd \mbf{f} = \oint \mbf{E}\cdot \mathd \mbf{l}
\end{equation*}
由方程\eqref{chapter4:电磁场方程-第一个Maxwell方程}可得
\begin{equation}
	\oint \mbf{E}\cdot \mathd \mbf{l} = -\frac1c \frac{\pl}{\pl t}\int \mbf{H}\cdot \mathd \mbf{f}
	\label{chapter4:电磁场方程-第一个Maxwell方程的积分形式}
\end{equation}
电场强度的环流也称为该回路内的{\bf 电动势}。即任何回路内的电动势,等于穿过由该回路所包围曲面的磁场强度通量的时间导数的负值。

Maxwell方程\eqref{chapter4:电磁场方程-第一个Maxwell方程}和\eqref{chapter4:电磁场方程-第二个Maxwell方程}可以写成四维形式,利用电磁场张量的定义式\eqref{chapter3:电磁场张量的定义},很容易验证出
\begin{equation}
	\frac{\pl F_{ik}}{\pl x^l} + \frac{\pl F_{kl}}{\pl x^i} + \frac{\pl F_{li}}{\pl x^k} = 0
	\label{chapter4:第一对Maxwell方程的四维形式1}
\end{equation}
方程\eqref{chapter4:第一对Maxwell方程的四维形式1}左边的表达式是一个三阶张量,它对所有三个指标都是反对称的。只有那些$i\neq k\neq l$的分量才是非零的。将式\eqref{chapter3:电磁场张量的分量形式}代入,可以验证这四个方程正好就是方程\eqref{chapter4:电磁场方程-第一个Maxwell方程}和\eqref{chapter4:电磁场方程-第二个Maxwell方程}。

将这个三阶反对称四维张量乘以$e^{iklm}$并对四对指标缩并,我们可以构造出与其对偶的四维矢量,因此方程\eqref{chapter4:第一对Maxwell方程的四维形式1}可以写成形式
\begin{equation}
	e^{iklm}\frac{\pl F_{lm}}{\pl x^k} = 0
	\label{chapter4:第一对Maxwell方程的四维形式2}
\end{equation}
这说明第一对Maxwell方程中,独立的方程只有四个。

\section{电磁场的作用量}

由电磁场和场内的粒子构成的整个体系的作用量$S$,应当包含三个部分:
\begin{equation}
	S = S_m+S_f+S_{mf}
\end{equation}
其中$S_m$是作用量中仅与粒子性质有关的部分,即自由粒子的作用量\eqref{chapter2:自由实物粒子的作用量}。如果有多个粒子,那么它们的总作用量就是单个粒子的作用量之和。因此有
\begin{equation}
	S_m = -\sum mc\int \mathd s
	\label{chapter4:电磁场作用量中粒子的部分}
\end{equation}

\section{四维电流矢量}

\section{连续性方程}

\section{第二对Maxwell方程}

\section{能量密度和能流}

\section{能量动量张量}

\section{电磁场的能量动量张量}

\section{位力定理}

\section{宏观物体的能量动量张量}

\backmatter

\addcontentsline{toc}{part}{\listfigurename}{%
\let\oldnumberline\numberline%
\renewcommand{\numberline}{\figurename~\oldnumberline}%
\listoffigures
}

%\listoffigures

%参考文献
\begin{thebibliography}{9}
\addcontentsline{toc}{part}{\bibname}

\bibitem{Landau-力学-2013}
{郎道, 栗弗席兹.}\, 力学(第五版). 北京: 高等教育出版社, 2013.

\bibitem{Landau-场论-2012}
{郎道, 栗弗席兹.}\, 场论(第八版). 北京: 高等教育出版社, 2012.

\end{thebibliography}


\end{document}
