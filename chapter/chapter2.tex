\chapter{相对论力学}

\section{Hamilton原理}%对应Landau的最小作用量原理

现在我们从Hamilton原理开始,研究实物粒子的运动规律。Hamilton原理指出:对于每一个力学体系,有一个叫做{\bf 作用量}的积分$S$存在,这个积分对于实际运动有最小值\footnote{严格的讲,应该是驻值。},即它的变分$\delta S$为零。

一个自由实物粒子的作用量积分必然与参考系的选择无关,即,它必须对于Lorentz变换保持不变。因此,它必须为一个标量函数。另外显然其被积函数必须是一个微分$1-$形式。对于一个自由粒子,所能构造出的唯一的这种标量,仅仅是间隔$\mathd s$,或者固有时$\mathd \tau$,或者它们乘以一个常数$\alpha\mathd s$。这样一来,对于一个自由粒子,作用量的积分必须取下面的形式:
\begin{equation*}
	S = -\alpha \int_{\tau_1}^{\tau_2} \mathd s
\end{equation*}
其中$\ds \int_a^b$表示沿着粒子在两个特定事件间的世界线的积分,这两个事件就是粒子在$t_1$时刻到达初位置和在$t_2$时刻到达末位置,也就是说$\ds \int_a^b$是沿着两个世界点之间的世界线的积分;而$\alpha$为表征该粒子的一个常数。在第\ref{chapter1:section固有时}节中,积分$\ds\int_a^b \mathd s$沿着一条直的世界线的值最大;沿着一条弯曲的世界线,可以使得积分值为任意小\footnote{因为这些积分的值都是负值。}

作用量可以表示为对时间的积分$\ds S = \int_{t_1}^{t_2} L\mathd t$,其中$L$即为这个力学体系的{\bf Lagrange函数}。利用式\eqref{chapter1:无穷小固有时},可得
\begin{equation*}
	S = -\int_{t_1}^{t_2} \alpha c\sqrt{1-\frac{v^2}{c^2}} \mathd t
\end{equation*}
其中$v$为实物粒子的速度,即实物粒子的Lagrange函数为
\begin{equation}
	L = -\alpha c\sqrt{1-\frac{v^2}{c^2}} 
\end{equation}

上面已经提到,$\alpha$是表征该粒子的一个量。在经典力学中,这个量就是该粒子的质量$m$。当我们取$c\to +\infty$的极限时,$L$的表达式应该过渡到它的经典表达式$L=\dfrac12 mv^2$。将$L$按$\dfrac{v}{c}$展开至$\dfrac{v^2}{c^2}$项可得
\begin{equation*}
	L = -\alpha c\sqrt{1-\dfrac{v^2}{c^2}} = -\alpha c+\frac{\alpha v^2}{2c}
\end{equation*}
Lagrange函数中的常数项对运动方程没有影响,可以略去。略去常数项$-\alpha c$之后,与经典力学中自由粒子的Lagrange函数$L=\dfrac12 mv^2$比较,可得$\alpha=mc$。

所以,自由实物粒子的作用量是
\begin{equation}
	S = -mc\int_a^b \mathd s
	\label{chapter2:自由实物粒子的作用量}
\end{equation}
而Lagrange函数是
\begin{equation}
	L = -mc^2\sqrt{1-\dfrac{v^2}{c^2}}
	\label{chapter2:自由实物粒子的Lagrange函数}
\end{equation}

\section{能量与动量}

类似于经典Hamilton力学中的做法,将矢量$\mbf{p} = \dfrac{\pl L}{\pl \mbf{v}}$\footnote{$\mbf{p} = \dfrac{\pl L}{\pl \mbf{v}}$表示$\ds p^i = \frac{\pl L}{\pl v^i}$。}称为该粒子的{\bf 动量}。利用式\eqref{chapter2:自由实物粒子的Lagrange函数}可得
\begin{equation}
	\mbf{p} = \frac{m\mbf{v}}{\sqrt{1-\dfrac{v^2}{c^2}}}
	\label{chapter2:自由实物粒子的动量}
\end{equation}
对于很小的速度($v\ll c$),或者$c\to +\infty$的极限情形下,式\eqref{chapter2:自由实物粒子的动量}就变为经典的动量公式$\mbf{p} = m\mbf{v}$。而当$v=c$时,实物粒子的动量就变为无穷大。

动量对时间的导数就是作用于粒子的力。将速度矢量记作$\mbf{v}=v\mbf{e}$,其中$\mbf{e}$是与$\mbf{v}$通向的方向矢量,则有
\begin{equation}
	\frac{\mathd \mbf{p}}{\mathd t} = \frac{mv}{\sqrt{1-\dfrac{v^2}{c^2}}} \frac{\mathd \mbf{e}}{\mathd t} + \frac{m\mbf{e}}{\left(1-\dfrac{v^2}{c^2}\right)^{\frac32}} \frac{\mathd v}{\mathd t}
\end{equation}
当粒子的速度大小不变仅有方向改变时,即力与速度方向垂直时,有
\begin{equation*}
	\frac{\mathd \mbf{p}}{\mathd t} = \frac{mv}{\sqrt{1-\dfrac{v^2}{c^2}}} \frac{\mathd \mbf{e}}{\mathd t}
\end{equation*}
当粒子的速度方向不变仅有大小改变时,即力与速度方向平行时,有
\begin{equation*}
	\frac{\mathd \mbf{p}}{\mathd t} = \frac{m\mbf{e}}{\left(1-\dfrac{v^2}{c^2}\right)^{\frac32}} \frac{\mathd v}{\mathd t}
\end{equation*}
即力与加速度之比已经不再是常数,Newton第二定律失效。

粒子的{\bf 能量}$\E$\footnote{此处用$\E$而不用$E$来表示能量是为了避免与后面涉及的电场强度$\mbf{E}$相混淆。}可以定义为
\begin{equation}
	\E = \mbf{p}\cdot \mbf{v} - L
	\label{chapter2:粒子能量的定义}
\end{equation}
将式\eqref{chapter2:自由实物粒子的动量}和式\eqref{chapter2:自由实物粒子的Lagrange函数}代入可得
\begin{equation}
	\E = \frac{mc^2}{\sqrt{1-\dfrac{v^2}{c^2}}}
	\label{chapter2:自由实物粒子的能量}
\end{equation}
相对论力学与经典力学的一个重要差别即为,粒子的能量在$v=0$时并不为零,而是有一个有限值
\begin{equation}
	\E_0 = mc^2
	\label{chapter2:粒子的静能}
\end{equation}
这个量称为粒子的{\bf 静能}。当速度远小于光速时,将式\eqref{chapter2:自由实物粒子的能量}展开为$\dfrac{v^2}{c^2}$的幂级数,可以得到
\begin{equation*}
	\E \approx mc^2 + \frac12mv^2
\end{equation*}
即,扣除静能后,能量表达式\eqref{chapter2:自由实物粒子的能量}退化为经典结果。

虽然我们考虑的是一个自由粒子的情形,但是这些公式同样可以适用于许多粒子构成的复合物体。此时$m$即为该物体的总质量,$\mbf{v}$是指它整体的运动速度。在相对论力学中,任何封闭系统的能量都是一个完全确定的量,它总是正的并与该系统的质量直接相关。而在经典力学中,封闭系统的能量可以有一个任意常数的差别,而且既可以为正值,也可以为负值。

一个静止系统的能量,除了其组成粒子的静能外,还包括它们的动能和它们之间的相互作用能。因此,$mc^2$并不等于$\ds \sum m_\alpha c^2$(其中$m_\alpha$为各个粒子单独存在时的质量),也就是说$m$并不等于$\ds \sum m_\alpha$。在相对论力学中,质量守恒定律不再成立,复合物体的质量不再等于其各部分质量之和,而只有包含了粒子静能在内的能量守恒定律是成立的。

利用式\eqref{chapter2:自由实物粒子的动量}和式\eqref{chapter2:自由实物粒子的能量}消去粒子的速度,可得自由实物粒子能量与动量之间的关系
\begin{equation}
	\frac{\E^2}{c^2} = p^2+m^2c^2
	\label{chapter2:自由实物粒子能量与动量之间的关系}
\end{equation}
进而可获得粒子的Hamilton函数$\mathscr{H}$\footnote{此处用$\mathscr{H}$而不用$H$来表示Hamilton函数是为了避免与后面涉及的磁场强度$\mbf{H}$相混淆。}:
\begin{equation}
	\mathscr{H} = c\sqrt{p^2+m^2c^2}
	\label{chapter2:自由实物粒子的Hamilton函数}
\end{equation}
对于低速情况$\dfrac{p}{mc}\ll 1$,近似地有
\begin{equation*}
	\mathscr{H} \approx mc^2 + \frac{p^2}{2m}
\end{equation*}
扣除静能后,Hamilton函数也退化为经典结果。

利用式\eqref{chapter2:自由实物粒子的动量}和式\eqref{chapter2:自由实物粒子的能量}我们可以得到一个自有例子的能量、动量和速度之间的关系为
\begin{equation}
	\mbf{p} = \frac{\E \mbf{v}}{c^2}
	\label{chapter2:自由实物粒子能量、动量和速度之间的关系}
\end{equation}

当$v=c$时,粒子的动量和能量都变为无穷大,因此,一个粒子如果其质量不为零,就不可能以光速运动。在相对论力学中,可以存在质量为零同时以光速运动的粒子(例如光子和中微子)。根据式\eqref{chapter2:自由实物粒子能量、动量和速度之间的关系},我们有
\begin{equation}
	p = \frac{\E}{c}
\end{equation}
同样的公式对于非零质量的粒子在{\bf 极端相对论}情况下{\bf 近似成立},此时粒子的能量将远远大于其静能$mc^2$。

下面来推导上面所得到所有关系式的四维形式。根据最小作用量原理
\begin{equation*}
	\delta S = -mc\delta\int_a^b \mathd s
\end{equation*}
由于$\mathd s = \sqrt{\mathd x^i\mathd x_i}$,故有
\begin{align*}
	\delta S & = -mc\int_a^b \frac{\mathd x_i\delta \mathd x^i}{\mathd s} = -m\int_a^b u_i\mathd \delta x^i = -mu_i\delta x^i\bigg|_a^b + m\int_a^b \delta x^i \frac{\mathd u_i}{\mathd s}\mathd s \\
	& = -mu_i\delta x^i\bigg|_a^b + \frac{m}{c}\int_a^b \delta x^i \frac{\mathd u_i}{\mathd \tau}\mathd s
\end{align*}
由$\delta S = 0$和固定边界条件可得自由实物粒子的运动方程即为$\dfrac{\mathd u_i}{\mathd \tau} = 0$。

但现在我们需要将作用量的变分表示为坐标的函数,此时需将$a$点当做固定的,令粒子按照实际的轨道运动,考察$\delta S$与$(\delta x^i)_b$之间的关系。根据$b$点所取的任意性,将$(\delta x^i)_b$记作$\delta x^i$,则有
\begin{equation}
	\delta S = -mu_i\delta x^i
\end{equation}
由此,称四维矢量
\begin{equation}
	p_i = -\frac{\pl S}{\pl x^i} = mu_i
	\label{chapter2:四维动量矢量}
\end{equation}
为{\bf 四维动量矢量}。在经典力学中,导数$\dfrac{\pl S}{\pl x}, \dfrac{\pl S}{\pl y}, \dfrac{\pl S}{\pl z}$是粒子动量矢量$\mbf{p}$的三个分量,而导数$-\dfrac{\pl S}{\pl t}$是粒子的能量$\E$。因此,即有四维动量的协变分量为
\begin{equation}
	p_i = \left(\frac{\E}{c},-\mbf{p}\right)
	\label{chapter2:四维动量矢量的协变分量}
\end{equation}
而逆变分量则为
\begin{equation}
	p^i = \left(\frac{\E}{c},\mbf{p}\right)
	\label{chapter2:四维动量矢量的逆变分量}
\end{equation}

因此,在相对论力学中,能量与动量是同一个四维矢量的分量。故可直接由此获得能量与动量由一个惯性系变换到另一个惯性系时的变换公式,即
\begin{equation}
\begin{cases}
	\ds p_x = \frac{p'_x+\dfrac{V}{c^2}\E'}{\sqrt{1-\dfrac{V^2}{c^2}}} \\
	p_y = p'_y \\
	p_z = p'_z \\
	\ds \E = \frac{\E'+Vp'_x}{\sqrt{1-\dfrac{V^2}{c^2}}}
\end{cases}
\label{chapter2:惯性系能量动量变换关系式}
\end{equation}
式中$p_x, p_y, p_z$是三维矢量$\mbf{p}$的分量(即四维矢量$p^i$的逆变分量$p^1, p^2$和$p^3$)。

四维动量满足
\begin{equation}
	p^ip_i = m^2u^iu_i = m^2c^2
	\label{chapter2:四维动量的平方}
\end{equation}
将式\eqref{chapter2:四维动量矢量的协变分量}和式\eqref{chapter2:四维动量矢量的逆变分量}代入,即可得到关系式\eqref{chapter2:自由实物粒子能量与动量之间的关系}。

而类比于力的通常定义,四维力矢量可以定义为
\begin{equation}
	f^i = \frac{\mathd p^i}{\mathd \tau} = m\frac{\mathd u^i}{\mathd \tau}
	\label{chapter2:四维力矢量}
\end{equation}
其满足恒等式
\begin{equation}
	f^i u_i = 0
\end{equation}
这个四维矢量可以用通常的三维力矢量$\mbf{F} = \dfrac{\mathd \mbf{p}}{\mathd t}$表示为
\begin{equation}
	f^i = \Bigg(\frac{\mbf{f}\cdot\mbf{v}}{c\sqrt{1-\dfrac{v^2}{c^2}}}, \frac{\mbf{f}}{\sqrt{1-\dfrac{v^2}{c^2}}}\Bigg)
\end{equation}
其时间分量与外力对该粒子的功率相关。

将式\eqref{chapter2:四维动量矢量}代入式\eqref{chapter2:四维动量的平方}就得到相对论的Hamilton-Jacobi方程
\begin{equation}
	\frac{\pl S}{\pl x^i} \frac{\pl S}{\pl x_i} = \eta^{ij} \frac{\pl S}{\pl x^i}\frac{\pl S}{\pl x^j} = m^2c^2
	\label{chapter2:相对论力学的Hamilton-Jacobi方程}
\end{equation}
或者将其展开为
\begin{equation}
	\frac{1}{c^2}\left(\frac{\pl S}{\pl t}\right)^2 - \left(\frac{\pl S}{\pl x}\right)^2 - \left(\frac{\pl S}{\pl y}\right)^2 - \left(\frac{\pl S}{\pl z}\right)^2 = m^2c^2
	\label{chapter2:相对论力学的Hamilton-Jacobi方程2}
\end{equation}
将方程\eqref{chapter2:相对论力学的Hamilton-Jacobi方程2}过渡到经典力学极限需要首先作变换
\begin{equation}
	S = S'-mc^2t
	\label{chapter2:作用量的代换}
\end{equation}
将$S'$作为新的作用量。这是由于相对论力学中一个粒子的能量包括了一个经典力学中没有的$mc^2$项。将式\eqref{chapter2:作用量的代换}代入\eqref{chapter2:相对论力学的Hamilton-Jacobi方程2}中,可得
\begin{equation*}
	\frac{1}{2mc^2} \left(\frac{\pl S'}{\pl t}\right)^2 - \frac{\pl S'}{\pl t} - \frac{1}{2m}\left[\left(\frac{\pl S'}{\pl x}\right)^2 + \left(\frac{\pl S'}{\pl y}\right)^2 + \left(\frac{\pl S'}{\pl z}\right)^2\right] = 0
\end{equation*}
在$c\to +\infty$的极限情况下,这个方程就退化为经典的Hamilton-Jacobi方程。

\section{分布函数的变换}

在许多物理问题中都需要处理粒子动量的分布函数$f(\mbf{p})$,乘积$f(\mbf{p})\mathd p_x\mathd p_y\mathd p_z$表示动量分量在给定间隔$\mathd p_x, \mathd p_y, \mathd p_z$内的粒子数,即在动量空间内体积元$\mathd^3p = \mathd p_x\mathd p_y \mathd p_z$内的粒子数。于是当我们从一个参考系变换到另一个参考系中时,需要同时获得分布函数$f(\mbf{p})$的变换。

因为$f(\mbf{p})\mathd p_x\mathd p_y \mathd p_z$表示粒子数,这显然是一个不变量,所以只需要确定“体积元”$\mathd p_x\mathd p_y \mathd p_z$在Lorentz变换下的性质。引入一个四维坐标系,其坐标为四维动量矢量,则$\mathd p_x\mathd p_y \mathd p_z$可以看作是由方程$p^ip_i=m^2c^2$定义的超曲面元的零分量。这个超曲面元是沿该超曲面法线指向的四维矢量。超曲面$p^ip_i=m^2c^2$的法线方向与四维矢量$p_i$一致。由此可知,比值
\begin{equation}
	\frac{\mathd p_x\mathd p_y\mathd p_z}{\E}
\end{equation}
是一个不变量,因为它是两个平行四维矢量的对应分量的比值。由于$f(\mbf{p})\mathd p_x\mathd p_y \mathd p_z$也是不变量,将其写作
\begin{equation*}
	f(\mbf{p})\E \frac{\mathd p_x\mathd p_y\mathd p_z}{\E}
\end{equation*}
由此可得乘积$f(\mbf{p})\E$是不变量。因此可以得到$K'$系中的分布函数与$K$系中分布函数的关系为
\begin{equation*}
	f'(\mbf{p}')\E' = f(\mbf{p})\E
\end{equation*}
由此即有
\begin{equation}
	f'(\mbf{p}') = \frac{f(\mbf{p})\E}{\E'}
\end{equation}
其中$\mbf{p}$和$\E$需要利用变换公式\eqref{chapter2:惯性系能量动量变换关系式}表示。

在气体动理论中,分布函数则表示为$f(\mbf{r},\mbf{p})$,乘积$f(\mbf{r},\mbf{p})\mathd p_x\mathd p_y\mathd p_z\mathd V$表示体积元$\mathd V$中动量在间隔$\mathd p_x, \mathd p_y, \mathd p_z$内的粒子数。函数$f(\mbf{r},\mbf{p})$称为{\bf 相空间}中的分布函数,微分形式$\mathd \tau = \mathd^3p\mathd V$是这个空间的体积元。

方便起见,再引入一个具有给定动量的粒子在其中处于静止的参考系$K_0$,粒子在该参考系中占有的体积元即为其固有体积$\mathd V_0$。由于$K_0$相对于该粒子是静止的,所以$K$系和$K'$系相对于$K_0$系的速度与这些粒子在$K$和$K'$系中的速度相同,因此有
\begin{equation*}
	\mathd V = \mathd V_0\sqrt{1-\frac{v^2}{c^2}},\quad \mathd V' = \mathd V_0\sqrt{1-\frac{v'^2}{c^2}}
\end{equation*}
由此即有
\begin{equation*}
	\frac{\mathd V}{\mathd V'} = \frac{\sqrt{1-\dfrac{v^2}{c^2}}}{\sqrt{1-\dfrac{v'^2}{c^2}}} = \frac{\E'}{\E}
\end{equation*}
又由于$\dfrac{\mathd^3p}{\E} = \dfrac{\mathd^3p'}{\E'}$,即有
\begin{equation*}
	\mathd \tau = \mathd \tau'
\end{equation*}
即相空间的体积元是不变量\footnote{此处相当于是将经典统计力学中的Liouville定理推广至了相对论情形。}。因为粒子数$f(\mbf{r},\mbf{p})\mathd \tau$根据定义也是不变量,因而可得:相空间的分布函数是不变量,即
\begin{equation}
	f'(\mbf{r}',\mbf{p}') = f(\mbf{r},\mbf{p})
\end{equation}
式中的$\mbf{r}',\mbf{p}'$由Lorentz变换公式与$\mbf{r},\mbf{p}$相联系。

\section{粒子的衰变}

现在来考虑一个质量为$M$的粒子自发衰变为质量分别为$m_1$和$m_2$两部分的情形。将能量守恒定律应用于原粒子处于静止的参考系中,可得
\begin{equation}
	Mc^2 = \E_{10}+\E_{20}
	\label{chapter2:粒子衰变的能量守恒}
\end{equation}
式中$\E_{10}$和$\E_{20}$是出射粒子的能量。因为它们的动量都非零,故$\E_{10} > m_1c^2, \E_{20} > m_2c^2$,因此仅当$M>m_1+m_2$时,式\eqref{chapter2:粒子衰变的能量守恒}才能满足,即一个粒子可以自发衰变为质量之和小于该粒子的两部分。另一方面,如果$M<m_1+m_2$,则粒子对于该类型的衰变是稳定的,不会自发衰变。如果需要在这种情况下人为引起衰变,必须从外界对其注入至少等于其“束缚能”$(m_1+m_2-M)c^2$的能量。

衰变过程中,动量也是守恒的。因为粒子的初始动量是零,故出射粒子的动量之和也是零,即
\begin{equation}
	\mbf{p}_{10}+\mbf{p}_{20} = \mbf{0}
	\label{chapter2:粒子衰变的动量守恒1}
\end{equation}
因而有$p_{10}^2=p_{20}^2$,即有
\begin{equation}
	\E_{10}^2-m_1^2c^4 = \E_{20}^2-m_2^2c^4
	\label{chapter2:粒子衰变的动量守恒2}
\end{equation}
式\eqref{chapter2:粒子衰变的能量守恒}和式\eqref{chapter2:粒子衰变的动量守恒2}唯一决定了出射粒子的能量:
\begin{equation}
\begin{cases}
	\ds \E_{10} = \frac{M^2+m_1^2-m_2^2}{2M}c^2 \\
	\ds \E_{20} = \frac{M^2-m_1^2+m_2^2}{2M}c^2
\end{cases}
\end{equation}

在一定意义上,这个问题的逆问题就是计算两个碰撞粒子在其总动量为零的参考系\footnote{这个参考系简称为{\bf 动量中心系}或“C系”。}中的总能量$Mc^2$。这个量的计算给出了伴随碰撞粒子状态改变或新粒子“产生”的各种非弹性碰撞过程可能存在的判据。仅当“反应产物”的质量之和不超过$M$时,这类过程才能够发生。

假设在初始参考系(即{\bf 实验室系})中,一个质量为$m_1$能量为$\E_1$的粒子同一个质量为$m_2$的静止粒子相撞,这两个粒子的总能量是
\begin{equation*}
	\E = \E_1+\E_2 = \E_1 + m_2c^2
\end{equation*}
它们的总动量则为
\begin{equation*}
	\mbf{p} = \mbf{p}_1+\mbf{p}_2 = \mbf{p}_1
\end{equation*}
将两个粒子看成一个复合系统,由式\eqref{chapter2:自由实物粒子能量、动量和速度之间的关系}可得其整体的运动速度为
\begin{equation}
	\mbf{V} = \frac{\mbf{p}c^2}{\E} = \frac{\mbf{p}_1c^2}{\E+m_2c^2}
\end{equation}
这个速度即为C系相对于实验室系(L系)的运动速度。

当然在测定$M$时,没有必要从一个参考系变换到另一个参考系\footnote{因为质量$M$是四维标量。}。直接利用实物粒子的能量动量关系\eqref{chapter2:自由实物粒子能量与动量之间的关系}即有
\begin{equation*}
	M^2c^4 = \E^2-p^2c^2 = (\E_1+m_2c^2)^2-(\E_1^2-m_1^2c^4)
\end{equation*}
由此可有
\begin{equation}
	M^2 = m_1^2+m_2^2+\frac{2m_2\E_1}{c^2}
\end{equation}

\iffalse
\begin{example}
一个以速度$V$运动的粒子在“飞行”中分解为两个粒子。求这些粒子的出射角同其能量之间的关系。
\end{example}
\begin{solution}
记$\E_0$为衰变粒子之一在C系中的能量,$\E$是这一粒子在L系中的能量
\end{solution}
\fi

\section{不变截面}

碰撞过程由其{\bf 有效截面}(或{\bf 截面})表征,它决定碰撞的粒子束之间发生的(特定类型的)碰撞数。

假设有两个碰撞束,用$n_1$和$n_2$表示其中的粒子数密度(即单位体积内的粒子数),用$\mbf{v}_1$和$\mbf{v}_2$表示粒子的速度。在粒子2处于静止的参考系(或者称粒子2的{\bf 静止系})中,只需考虑粒子束1与静止靶的碰撞。于是按照碰撞截面$\sigma$的通常定义,体积$\mathd V$内时间$\mathd t$中发生的碰撞数是
\begin{equation}
	\mathd \nu = \sigma v_{\text{rel}}n_1n_2 \mathd V \mathd t
	\label{chapter2:不变截面1}
\end{equation}
式中$v_{\text{rel}}$是粒子1在粒子2静止系中的速度。

由于$\mathd \nu$是粒子数,因此$\mathd \nu$应该是不变量。下面我们来将它表示为可适用于任何参考系的形式:
\begin{equation}
	\mathd \nu = An_1n_2\mathd V\mathd t
	\label{chapter2:不变截面2}
\end{equation}
式中$A$是一个待定的数,在粒子之一为静止的参考系中,$A$的值为$\sigma v_{\text{rel}}$。

在式\eqref{chapter2:不变截面2}中,微分形式$\mathd V\mathd t$是不变量,因此乘积$An_1n_2$必然也是不变量。而给定体积元中的粒子数$n\mathd V$也是不变量,由此即可得到粒子数密度的变换规律。记$n\mathd V = n_0\mathd V_0$,其中带有指标$0$的量表示其在静止系中的值,利用体积变换公式\eqref{chapter1:固有体积},可得粒子数密度之间的变换关系为
\begin{equation}
	n = \frac{n_0}{\sqrt{1-\dfrac{v^2}{c^2}}}
	\label{chapter2:不变截面3}
\end{equation}
再由式\eqref{chapter2:自由实物粒子能量与动量之间的关系}和式\eqref{chapter2:自由实物粒子能量、动量和速度之间的关系}可得
\begin{equation*}
	\frac{\E^2}{m^2c^4} = \frac{p^2c^2}{m^2c^4} + 1 = 1 + \frac{\E^2}{m^2c^4} \frac{v^2}{c^2}
\end{equation*}
即有
\begin{equation*}
	\frac{\E}{mc^2} = \frac{1}{\sqrt{1-\dfrac{v^2}{c^2}}}
\end{equation*}
据此以及式\eqref{chapter2:不变截面3}可得
\begin{equation}
	n = \frac{\E}{mc^2}n_0
	\label{chapter2:不变截面4}
\end{equation}

因此,$An_1n_2$是不变量与$A\E_1\E_2$是不变量等价。这个条件也可以表示为如下更方便的形式,即量
\begin{equation}
	A\frac{\E_1\E_2}{p_{1i}p_2^i} = A\frac{\E_1\E_2}{\E_1\E_2-\mbf{p}_1\cdot \mbf{p}_2}
	\label{chapter2:不变截面5}
\end{equation}
也是不变量。在粒子2的静止系中,有$\E_2=m_2c^2, \mbf{p}_2=\mbf{0}$,此时不变量\eqref{chapter2:不变截面5}就等于$A$。另一方面,在该参考系中有$A = \sigma v_{\text{rel}}$。所以再任意参考系中有
\begin{equation}
	A = \sigma v_{\text{rel}} \frac{p_{1i}p_2^i}{\E_1\E_2}
	\label{chapter2:不变截面6}
\end{equation}

下面用两个粒子的速度来改写该式。在粒子2的静止系中,有
\begin{equation*}
	p_{1i}p^{2i} = \frac{\E_1\E_2}{c^2} = \E_1m_2 = \frac{m_1m_2c^2}{\sqrt{1-\dfrac{v_{\text{rel}}^2}{c^2}}}
\end{equation*}
于是有
\begin{equation}
	v_{\text{rel}} = c\sqrt{1-\frac{m_1^2m_2^2c^4}{(p_{1i}p_2^i)^2}}
	\label{chapter2:不变截面7}
\end{equation}
利用式\eqref{chapter2:自由实物粒子的能量}和式\eqref{chapter2:自由实物粒子的动量}可得
\begin{equation*}
	p_{1i}p_2^i = \frac{\E_1\E_2}{c^2} - \mbf{p}_1\cdot\mbf{p}_2 = m_1m_2c^2 \frac{1-\dfrac{\mbf{v}_1}{c}\cdot \dfrac{\mbf{v}_2}{c}}{\sqrt{\left(1-\dfrac{v_1^2}{c^2}\right)\left(1-\dfrac{v_2^2}{c^2}\right)}}
\end{equation*}
据此可有
\begin{equation}
	v_{\text{rel}} = \frac{\sqrt{(\mbf{v}_1-\mbf{v}_2)^2-(\mbf{v}_1 \times \mbf{v}_2)^2}}{1-\dfrac{\mbf{v}_1}{c}\cdot \dfrac{\mbf{v}_2}{c}}
	\label{chapter2:不变截面8-粒子间相对速度的计算}
\end{equation}
注意到式\eqref{chapter2:不变截面8-粒子间相对速度的计算}对$\mbf{v}_1$和$\mbf{v}_2$是对称的,即相对速度的数值与用来定义它的粒子的选择无关。

由此,将式\eqref{chapter2:不变截面7}或式\eqref{chapter2:不变截面8-粒子间相对速度的计算}代入式\eqref{chapter2:不变截面6}和式\eqref{chapter2:不变截面2}中,即得到不变截面的最终形式为
\begin{equation}
	\mathd \nu = \sigma \frac{\sqrt{(p_{1i}p_2^i)^2-m_1^2m_2^2c^4}}{\E_1\E_2}n_1n_2 \mathd V\mathd t
	\label{chapter2:不变截面最终形式1}
\end{equation}
或者
\begin{equation}
	\mathd \nu = \sigma \sqrt{(\mbf{v}_1-\mbf{v}_2)^2-(\mbf{v}_1 \times \mbf{v}_2)^2} n_1n_2 \mathd V\mathd t
	\label{chapter2:不变截面最终形式2}
\end{equation}

如果速度$\mbf{v}_1$和$\mbf{v}_2$共线,此时$\mbf{v}_1\times \mbf{v}_2 = \mbf{0}$,于是式\eqref{chapter2:不变截面最终形式2}化为如下形式:
\begin{equation}
	\mathd \nu = \sigma |\mbf{v}_1-\mbf{v}_2|n_1n_2 \mathd V\mathd t
\end{equation}

本节的中间结果可以导出如下一个重要结论。
\begin{example}
求相对论速度空间中的“线元”,即速度为$\mbf{v}$和$\mbf{v}+\mathd \mbf{v}$两点间的相对速度。
\end{example}
\begin{solution}
根据式\eqref{chapter2:不变截面8-粒子间相对速度的计算}可得
\begin{equation*}
	\mathd l_{\mbf{v}}^2 = \frac{(\mathd \mbf{v})^2 - (\mbf{v}\times \mathd \mbf{v})^2}{\left(1-\dfrac{v^2}{c^2}\right)^2}
\end{equation*}
如果在速度空间中引入球坐标系,则有
\begin{equation*}
	\mathd l_{\mbf{v}}^2 = \frac{\mathd v^2}{\left(1-\dfrac{v^2}{c^2}\right)^2} + \frac{v^2}{1-\dfrac{v^2}{c^2}}(\mathd \theta^2 + \sin^2 \theta \mathd \phi^2)
\end{equation*}
如果通过方程$v = \tanh \chi$引进新变量$\chi$来代替$v$,则线元表示为
\begin{equation}
	\mathd l_{\mbf{v}}^2 = \mathd \chi^2 + \sinh^2\chi (\mathd \theta^2 + \sin^2 \theta\mathd \phi^2)
\end{equation}

从几何的观点来看,这就是三维Lobachevsky空间\footnote{即罗巴切夫斯基空间。}(负常曲率空间)的线元。
\end{solution}

\section{粒子的弹性碰撞}

现在我们从相对论力学的观点来考虑粒子的{\bf 弹性碰撞}。将两个碰撞粒子的质量、动量和能量分别记作$m_1, \mbf{p}_1, \E_1$和$m_2, \mbf{p}_2, \E_2$,用带撇号的相应量表示碰撞后的量。

碰撞过程中的动量个能量守恒定律可以一并写为四维动量守恒方程
\begin{equation}
	p_1^i + p_2^i = p_1'^i + p_2'^i
	\label{chapter2:粒子的弹性碰撞1}
\end{equation}
从这个四维矢量方程可以构造出有助于进一步计算的不变关系式。将式\eqref{chapter2:粒子的弹性碰撞1}
改写为形式
\begin{equation}
	p_1^i+p_2^i - p_1'^i = p_2'^i
	\label{chapter2:粒子的弹性碰撞2}
\end{equation}
将其两边平方,即有
\begin{equation*}
	(p_1^i+p_2^i - p_1'^i)(p_{1i}+p_{2i} - p'_{1i}) = p_2'^ip'_{2i}
\end{equation*}
注意到其中$p_1^ip_{1i} = p_1'^ip'_{1i} = m_1^2c^2, p_2^ip_{2i} = p_2'^ip'_{2i} = m_2^2c^2$,可得
\begin{equation}
	m_1^2c^2 + p_{1i}p_2^i - p_{1i}p_1'^i - p_{2i}p_1'^i = 0
	\label{chapter2:粒子的弹性碰撞3}
\end{equation}
类似的将式$p_1^i+p_2^i - p_2'^i = p_1'^i$两边平方,可得
\begin{equation}
	m_2^2c^2 + p_{1i}p_2^i - p_{2i}p_2'^i - p_{1i}p_2'^i = 0
	\label{chapter2:粒子的弹性碰撞4}
\end{equation}

现在来考虑实验室系(L系)中的碰撞,该系中的粒子$m_2$碰撞前处于静止,即$\mbf{p}_2=\mbf{0}, \E_2 = m_2c^2$,于是有
\begin{equation}
\begin{cases}
	p_{1i}p_2^i = \E_1m_2 \\
	p_{2i}p_1'^i = m_2\E'_1 \\
	p_{1i}p_1'^i = \dfrac{\E_1\E'_1}{c^2} - \mbf{p}_1 \cdot \mbf{p}'_1 = \dfrac{\E_1\E'_1}{c^2} - p_1p'_1\cos \theta_1
\end{cases}
\label{chapter2:粒子的弹性碰撞5}
\end{equation}
式中$\theta_1$是入射粒子$m_1$的散射角。将式\eqref{chapter2:粒子的弹性碰撞5}代入式\eqref{chapter2:粒子的弹性碰撞3}中可得
\begin{equation}
	\cos \theta_1 = \frac{\E'_1(\E_1+m_2c^2) - \E_1m_2c^2-m_1^2c^4}{c^2p_1p'_1}
	\label{chapter2:粒子弹性碰撞后粒子1的散射角}
\end{equation}
类似地,由式\eqref{chapter2:粒子的弹性碰撞4}可得
\begin{equation}
	\cos \theta_2 = \frac{(\E_1+m_2c^2)(\E'_2-m_2c^2)}{c^2p_1p'_2}
	\label{chapter2:粒子弹性碰撞后粒子2的散射角}
\end{equation}
式中$\theta_2$是入射粒子的动量$\mbf{p}_1$与变换后动量$\mbf{p}'_2$之间的夹角。

式\eqref{chapter2:粒子弹性碰撞后粒子1的散射角}和式\eqref{chapter2:粒子弹性碰撞后粒子2的散射角}将L系中两个粒子的散射角与它们在碰撞中的能量变化联系了起来。反演这些公式,就可以用$\theta_1$和$\theta_2$来表示能量$\E'_1$和$\E'_2$。将式\eqref{chapter2:粒子弹性碰撞后粒子2的散射角}两边平方,并将关系
\begin{equation*}
	p_1^2c^2 = \E_1^2 - m_1^2c^4,\quad p_2'^2c^2 = \E_2'^2-m_2^2c^4
\end{equation*}
代入,可得
\begin{equation}
	\E_2' = m_2c^2 \frac{(\E_1+m_2c^2)^2 + (\E_1^2-m_1^2c^4)\cos^2 \theta_2}{(\E_1+m_2c^2)^2 - (\E_1^2-m_1^2c^4)\cos^2 \theta_2}
	\label{chapter2:粒子的弹性碰撞6}
\end{equation}
反演公式\eqref{chapter2:粒子弹性碰撞后粒子1的散射角}可得出在一般情形下用$\theta_1$表示$\E'$的非常复杂的公式。但当入射粒子质量为零,即$m_1=0$时,有$p_1c = \E_1, p_1'c = \E'$。此时入射粒子碰撞后的能量可以用其偏转角度表示为
\begin{equation}
	\E_1' = \frac{m_2c^2}{1-\cos\theta_1+\dfrac{m_2c^2}{\E_1}}
\end{equation}

注意到,如果$m_1>m_2$,即入射粒子重于靶粒子,则散射角$\theta_1$不能超过某一最大值,即
\begin{equation}
	\sin \theta_{1\max} = \frac{m_2}{m_1}
	\label{chapter2:粒子的弹性碰撞7}
\end{equation}
这与经典结果相一致。

一般来讲,在C系中讨论碰撞最为简单。我们用带有下标$0$的量来代表这个参考系中的量。因此有$\mbf{p}_{10} = -\mbf{p}_{20} =:\mbf{p}_0$。根据动量守恒,碰撞过程中,两个粒子的动量只有转动,保持数值相等且方向相反。再根据能量守恒,每个动量的数值保持不变。

设$\chi$为C系中的散射角,即动量$\mbf{p}_{10}$和$\mbf{p}_{20}$由于碰撞而转过的角。这个量完全决定了C系中,因而也是任何其他参考系中的散射过程。在L系中描述碰撞时它也是方便的,是应用动量和能量守恒以后仍然不定的单一参量。

我们借助这个参量来表示两个粒子在L系中的终态能量。重新计算式\eqref{chapter2:粒子的弹性碰撞3}中的各项,在C系中写出乘积$p_{1i}p_1'^i$如下
\begin{equation}
	p_{1i}p_1'^i = \frac{\E_{10}\E'_{10}}{c^2} - \mbf{p}_{10}\cdot \mbf{p}'_{10} = \frac{\E_{10}^2}{c^2} - p_0^2\cos \chi = p_0^2(1-\cos\chi)+m_1^2c^2
	\label{chapter2:粒子的弹性碰撞8}
\end{equation}
由于$p_{1i}p_1'^i$是四维标量,故在L系中,$p_{1i}p_1'^i$也为式\eqref{chapter2:粒子的弹性碰撞8}之取值。将式\eqref{chapter2:粒子的弹性碰撞8}和式\eqref{chapter2:粒子的弹性碰撞5}代入式\eqref{chapter2:粒子的弹性碰撞3}中,可得
\begin{equation}
	\E_1'-\E_1 = -\frac{p_0^2}{m_2}(1-\cos\chi)
	\label{chapter2:粒子的弹性碰撞9}
\end{equation}
最后还需要用L系中的诸量来表示$p_0^2$,为此,只需让不变量$p_{1i}p_2^i$在L系和C系中的值相等即可,即
\begin{equation*}
	\frac{\E_{10}\E_{20}}{c^2} - \mbf{p}_{10}\cdot \mbf{p}_{20} = \E_1m_2
\end{equation*}
即有
\begin{equation*}
	\sqrt{(p_0^2+m_1^2c^2)(p_0^2+m_2^2c^2)} = \E_1m_2 - p_0^2
\end{equation*}
从上式中解出$p_0^2$可有
\begin{equation}
	p_0^2 = \frac{m_2^2(\E_1^2-m_1^2c^4)}{m_1^2c^2+m_2^2c^2+2\E_1m_2}
	\label{chapter2:粒子的弹性碰撞10}
\end{equation}
因此最后可得
\begin{equation}
	\E_1' = \E_1 - \frac{m_2(\E_1^2-m_1^2c^4)}{m_1^2c^2+m_2^2c^2+2\E_1m_2}(1-\cos \chi)
	\label{chapter2:粒子的弹性碰撞11}
\end{equation}
第二个粒子的能量可以通过能量守恒定律$\E_1+m_2c^2=\E_1'+\E_2'$得到,即
\begin{equation}
	\E_2' = m_2c^2 + \frac{m_2(\E_1^2-m_1^2c^4)}{m_1^2c^2+m_2^2c^2+2\E_1m_2}(1-\cos \chi)
	\label{chapter2:粒子的弹性碰撞12}
\end{equation}

在式\eqref{chapter2:粒子的弹性碰撞11}和式\eqref{chapter2:粒子的弹性碰撞12}中,第二项表示第一个粒子失去并转移给第二个粒子的能量,当$\chi = \pi$时,能量转移最大为
\begin{equation}
	\E'_{2\max}-m_2c^2 = \E_1-\E_{1\min} = \frac{2m_2(\E_1^2-m_1^2c^4)}{m_1^2c^2+m_2^2c^2+2\E_1m_2}
	\label{chapter2:粒子的弹性碰撞13}
\end{equation}
碰撞后入射粒子的最小动能语气初始能量的比值为
\begin{equation}
	\frac{\E'_{1\min}-m_1c^2}{\E_1-m_1c^2} = \frac{(m_1-m_2)^2c^2}{m_1^2c^2+m_2^2c^2+2\E_1m_2}
	\label{chapter2:粒子的弹性碰撞14}
\end{equation}
在低速极限情况下(即$c\to +\infty$时),这个关系趋于常数
\begin{equation*}
	\left(\frac{m_1-m_2}{m_1+m_2}\right)^2
\end{equation*}
在能量$\E_1$很大的相反极限下,关系\eqref{chapter2:粒子的弹性碰撞14}则趋于$0$,而量$\E'_{1\min}$趋于常数
\begin{equation*}
	\E'_{1\min} = \frac{m_1^2c^2+m_2^2c^2}{2m_2}
\end{equation*}

\section{角动量}

