\chapter{电磁场中的电荷}

\section{相对论中的基本粒子}

粒子间的相互作用可以用{\bf 场}的概念来描述。在这种观点下,我们不认为是一个粒子直接作用于另一个粒子,而是认为一个粒子在其周围建立起场,这个场内的任何其他粒子都受到一定的力的作用。在经典力学中,场仅仅是用来描述粒子相互作用这一现象的方法。但在相对论中,由于相互作用是以有限速度传播的,因此,在某一时刻,作用在一个粒子上的力并不是由其他粒子在该时刻的位置决定。某一个粒子改变了位置,需要经过一段时间后才能影响到其他粒子。这说明了场本身具有物理上的真实性。相互作用具有传播速度表明粒子之间不能直接发生相互作用,而仅能影响其周围空间中的邻近区域。因此,应当认为,一个粒子与场发生相互作用,然后场与粒子发生相互作用。

在经典力学中,我们可以引入刚体的概念。刚体指在任何情况下都不会发生形变的物体。类似地,如果希望在相对论中引入刚体的概念,则这种刚体应该在它们处于静止的参考系中其所有尺寸都保持不变。但是,相对论使得一般情形下的刚体不可能存在。

例如,考虑一个绕自身轴转动的圆盘,并假设它是刚体。固连于圆盘的参考系显然不是惯性系。但是对于圆盘的每一个无限小单元,可以引入一个惯性参考系,其中该单元在某一时刻处于静止。而对圆盘上的不同单元,这些惯性系之间必然是有相对运动的。现在考察圆盘的一条半径上的所有无限小单元,由于它们之间的相对运动与其自身取向相垂直,因此一个静止的观察者在该半径扫过他时测量出旋转圆盘的半径应与圆盘静止时相同。而另一方面,在给定时刻圆盘圆周上从静止观察者身旁经过的每一单元的长度由于其取向与运动方向垂直,会发生Lorentz收缩而导致静止观察者测量出的旋转圆盘整个圆周的长度将小于静止圆盘的周长。于是我们发现,由于圆盘的转动静止观察者测得旋转圆盘的圆周与半径的比值将不再是$2\pi$(事实上这个数值将小于$2\pi$)。这个结论表明,圆盘实际上不再是刚体,它在转动时必然发生了某种复杂的形变,这种形变则显然与其组成物质的弹性性质有关。

还可以用另外一种方法来证明刚体是不可能存在的。假设某一外力作用在刚体的某一点上,使这个物体发生了运动。如果这个物体是刚体,那么它上面的任何一点都必须在相同时刻与受到外力的点同时开始运动,否则物体就要发生变形了。然而相对论表明相互作用具有传播速度,力是以有限速度从其作用点传到其余的点,它们不可能同时开始运动。

刚体的不存在性使得相对论力学中的基本粒子必须以{\bf 几何点}的形式存在而不能有有限的尺寸。即在经典的(非量子的)相对论力学中,基本粒子不能被赋予有限的尺寸,必须将其当做几何点来看待。

\section{场的四维势}

一个在给定电磁场中运动的粒子的作用量将由两部分组成\footnote{类比经典力学对势能的引入方式。}:一项即为自由粒子的作用量\eqref{chapter2:自由实物粒子的作用量},而另一项则描述粒子与场的相互作用,其中必然包括表征粒子本身性质的量和表征场的性质的量。

实验表明,粒子同电磁场相互作用的性质由一个量所决定,这个量称为粒子的{\bf 电荷}$e$\footnote{此处需要指出的是,这里的$e$只是指任意的电荷而非特指元电荷。现在我们还没有建立任何将电磁学量同任何已知的量联系起来的关系,因此这些新引入的量的单位可以任意选取。}。电荷可以为正,也可以为负,也可以为零。而场的性质则由一个四维矢量$A^i$,称为{\bf 四维势}表征,其分量是四维坐标的函数。这些量以形式
\begin{equation*}
	-\frac{e}{c}\int_a^b A_i\mathd x^i
\end{equation*}
出现在作用量里。因此,电磁场中电荷的作用量将具有如下形式:
\begin{equation}
	S = \int_a^b \left(-mc\mathd s - \frac{e}{c}A_i\mathd x^i\right)
	\label{chapter3:电磁场中电荷的作用量1}
\end{equation}

四维势矢量$A^i$的三个空间分量构成一个三维空间矢量$\mbf{A}$,称为电磁场的{\bf 矢势},时间分量称为电磁场的{\bf 标势},记作$A^0 = \phi$。即有
\begin{equation}
	A^i = \left(\phi,\mbf{A}\right),\quad A_i = \left(\phi,-\mbf{A}\right)
	\label{chapter3:电磁场的标势和矢势}
\end{equation}
所以作用量的积分可以写作
\begin{equation*}
	S = \int_a^b \left(-mc\mathd s + \frac{e}{c}\mbf{A}\cdot \mathd \mbf{r} - e\phi \mathd t\right)
\end{equation*}
引入$\mbf{v} = \dfrac{\mathd \mbf{r}}{\mathd t}$可得
\begin{equation}
	S = \int_{t_1}^{t_2} \left(-mc^2\sqrt{1-\frac{v^2}{c^2}} + \frac{e}{c}\mbf{A}\cdot \mbf{v} - e\phi\right) \mathd t
	\label{chapter3:电磁场中电荷的作用量2}
\end{equation}
由此可得电磁场中电荷的Lagrange函数
\begin{equation}
	L = -mc^2\sqrt{1-\frac{v^2}{c^2}} + \frac{e}{c}\mbf{A}\cdot \mbf{v} - e\phi
	\label{chapter3:电磁场中电荷的Lagrange函数}
\end{equation}
这个Lagrange函数与自由粒子Lagrange函数\eqref{chapter2:自由实物粒子的Lagrange函数}相差了
\begin{equation*}
	\frac{e}{c}\mbf{A}\cdot \mbf{v} - e\phi
\end{equation*}
该项描述了电荷与电磁场的相互作用。

粒子的广义动量为
\begin{equation}
	\mbf{P} = \frac{\pl L}{\pl \mbf{v}} = \frac{m\mbf{v}}{\sqrt{1-\dfrac{v^2}{c^2}}} + \frac{e}{c}\mbf{A} = \mbf{p}+\frac{e}{c}\mbf{A}
	\label{chapter3:电磁场中电荷的广义动量}
\end{equation}
此处$\mbf{p}$表示该电荷的{\bf 机械动量},以后简称为动量。

而电磁场中电荷的Hamilton量则由式
\begin{equation*}
	\mathscr{H} = \mbf{v}\cdot \frac{\pl L}{\pl \mbf{v}} - L
\end{equation*}
决定。将式\eqref{chapter3:电磁场中电荷的Lagrange函数}代入,可得
\begin{equation}
	\mathscr{H} = \frac{mc^2}{\sqrt{1-\dfrac{v^2}{c^2}}}+e\phi
	\label{chapter3:电磁场中电荷的Hamilton量-用速度表示的Hamilton量}
\end{equation}
但粒子的Hamilton量需用粒子的广义动量表示而非运动速度,由式\eqref{chapter3:电磁场中电荷的广义动量}和式\eqref{chapter3:电磁场中电荷的Hamilton量-初步}可以看出,$\mathscr{H}-e\phi$与$\mbf{P}-\dfrac{e}{c}\mbf{A}$之间应该满足
\begin{equation}
	\left(\frac{\mathscr{H}-e\phi}{c}\right)^2 = m^2c^2+\left(\mbf{P}-\dfrac{e}{c}\mbf{A}\right)^2
	\label{chapter3:电磁场中电荷的Hamilton量-初步}
\end{equation}
即有
\begin{equation}
	\mathscr{H} = \sqrt{m^2c^4+c^2\left(\mbf{P}-\dfrac{e}{c}\mbf{A}\right)^2}+e\phi
	\label{chapter3:电磁场中电荷的Hamilton量}
\end{equation}

在低速近似下,电磁场中电荷的Lagrange函数化为
\begin{equation}
	L = \frac12mv^2+\dfrac{e}{c}\mbf{A}\cdot \mbf{v}-e\phi
	\label{chapter3:低速近似下,电磁场中电荷的Lagrange函数}
\end{equation}
此时
\begin{equation*}
	\mbf{p} = m\mbf{v} = \mbf{P}-\dfrac{e}{c}\mbf{A}
\end{equation*}
Hamilton量的表达式为
\begin{equation}
	\mathscr{H} = \frac{1}{2m}\left(\mbf{P}-\dfrac{e}{c}\mbf{A}\right)^2 + e\phi
	\label{chapter3:低速近似下,电磁场中电荷的Hamilton量}
\end{equation}

最后来推导出电磁场中电荷的Hamilton-Jacobi方程,只需在Hamilton量\eqref{chapter3:电磁场中电荷的Hamilton量}中,用$\bnb S$代替广义动量$\mbf{P}$,用$-\dfrac{\pl S}{\pl t}$代替$\mathscr{H}$即可。由此,根据式\eqref{chapter3:电磁场中电荷的Hamilton量-初步}可得
\begin{equation}
	\left(\bnb S-\dfrac{e}{c}\mbf{A}\right)^2-\frac{1}{c^2}\left(\frac{\pl S}{\pl t} + e\phi\right)^2 + m^2c^2 = 0
	\label{chapter3:电磁场中电荷的Hamilton-Jacobi方程}
\end{equation}

\section{场中电荷的运动方程}\label{chapter3:section:场中电荷的运动方程}

场内的电荷不仅会受到场的作用力,还会反过来对场起作用,改变场的分布。但是,当电荷量很小的时候,电荷对于场的作用就可以忽略不计。在这种情况下,当我们只考虑电荷在给定的外电磁场中的运动时,可以假设场本身与电荷的坐标或速度无关。

现在我们在这种假设下,推导出电荷在给定电磁场内的运动方程。通过对作用量进行变分,可得运动方程就是Lagrange方程
\begin{equation}
	\frac{\mathd}{\mathd t} \frac{\pl L}{\pl \mbf{v}} - \frac{\pl L}{\pl \mbf{r}} = \mbf{0}
	\label{chapter3:电磁场中电荷的Lagrange方程}
\end{equation}
其中Lagrange函数由式\eqref{chapter3:电磁场中电荷的Lagrange函数}决定。导数$\dfrac{\pl L}{\pl \mbf{v}}$就是粒子的广义动量\eqref{chapter3:电磁场中电荷的广义动量},然后可以计算
\begin{equation*}
	\frac{\pl L}{\pl \mbf{r}} = \dfrac{e}{c}\bnb (\mbf{A}\cdot \mbf{v}) - e\bnb \phi
\end{equation*}
根据矢量恒等式
\begin{equation*}
	\bnb (\mbf{a}\cdot \mbf{b}) = (\mbf{a}\cdot \bnb)\mbf{b} + (\mbf{b}\cdot \bnb)\mbf{a} + \mbf{b}\times (\bnb \times \mbf{a}) + \mbf{a}\times (\bnb \times \mbf{b})
\end{equation*}
可得
\begin{equation*}
	\frac{\pl L}{\pl \mbf{r}} = \dfrac{e}{c}(\mbf{v}\cdot \bnb)\mbf{A} + \dfrac{e}{c}\mbf{v}\times (\bnb \times \mbf{A}) - e\bnb \phi
\end{equation*}
因此,Lagrange方程\eqref{chapter3:电磁场中电荷的Lagrange方程}变为
\begin{equation*}
	\frac{\mathd}{\mathd t}\left(\mbf{p}+\dfrac{e}{c}\mbf{A}\right) = \dfrac{e}{c}(\mbf{v}\cdot \bnb)\mbf{A} + \dfrac{e}{c}\mbf{v}\times (\bnb \times \mbf{A}) - e\bnb \phi
\end{equation*}
为了计算上式左端,考虑到矢势$\mbf{A}$是空间坐标和时间的函数,因此其全微分为
\begin{equation*}
	\mathd \mbf{A} = \frac{\pl \mbf{A}}{\pl t}\mathd t + (\mathd \mbf{r}\cdot \bnb) \mbf{A}
\end{equation*}
据此可有
\begin{equation*}
	\frac{\mathd \mbf{A}}{\mathd t} = \frac{\pl \mbf{A}}{\pl t} + (\mbf{v}\cdot \bnb)\mbf{A}
\end{equation*}
由此可得电荷在给定电磁场中运动的方程为
\begin{equation}
	\frac{\mathd \mbf{p}}{\mathd t} = -\dfrac{e}{c}\frac{\pl \mbf{A}}{\pl t} - e\bnb \phi + \dfrac{e}{c}\mbf{v}\times (\bnb \times \mbf{A})
	\label{chapter3:电荷在给定电磁场中的运动方程}
\end{equation}

式\eqref{chapter3:电荷在给定电磁场中的运动方程}左端即为粒子的动量对时间的导数,因此其右端就是电磁场作用在电荷上的力。这个力可以分为两部分,式\eqref{chapter3:电荷在给定电磁场中的运动方程}右端第一、第二项即为第一部分,这一部分的力与粒子的速度无关。第三项为第二部分,这部分的力与粒子的速度有关,它与速度成正比,而且垂直于速度。

我们将作用于单位电荷上的第一部分的力,称为{\bf 电场强度},记作$\mbf{E}$,于是有
\begin{equation}
	\mbf{E} = - \frac1c\frac{\pl \mbf{A}}{\pl t} - \bnb \phi
	\label{chapter3:电场强度的定义}
\end{equation}
电场强度$\mbf{E}$是极矢量。作用于单位电荷上的第二部分的力中的速度因子,称为{\bf 磁场强度}\footnote{如此定义的一个好处就是,电场强度与磁场强度将有相同的单位。},记作$\mbf{H}$,于是有
\begin{equation}
	\mbf{H} = \bnb \times \mbf{A}
	\label{chapter3:磁场强度的定义}
\end{equation}
磁场强度$\mbf{H}$是轴矢量。

如果在一电磁场中$\mbf{E}\neq \mbf{0}$,但$\mbf{H}=\mbf{0}$,我们就称它为{\bf 电场};如果$\mbf{E}= \mbf{0}$,但$\mbf{H}\neq \mbf{0}$,我们就称它为{\bf 磁场}。在一般情形下,电磁场是电场和磁场的叠加。

由此,一个电荷在电磁场中的运动方程可以写作
\begin{equation}
	\frac{\mathd \mbf{p}}{\mathd t} = e\mbf{E}+\dfrac{e}{c}\mbf{v}\times \mbf{H}
	\label{chapter3:电荷在给定电磁场中的运动方程-Lorentz力}
\end{equation}
等式右端的式子称为{\bf Lorentz力}。其第一部分(电场作用于电荷上的力)与电荷速度无关,并沿着$\mbf{E}$的方向。第二部分(磁场作用于电荷上的力)与电荷速度成正比,而其方向既垂直于速度又垂直于磁场$\mbf{H}$。

粒子在电磁场中的机械能由式\eqref{chapter2:自由实物粒子的能量}决定,即
\begin{equation*}
	\E_{\text{m}} = \frac{mc^2}{\sqrt{1-\dfrac{v^2}{c^2}}}
\end{equation*}
将式\eqref{chapter2:自由实物粒子能量与动量之间的关系}两端对时间求导数,即可得
\begin{equation*}
	\frac{\mathd \E_{\text{m}}}{\mathd t} = \mbf{v}\cdot \frac{\mathd \mbf{p}}{\mathd t}
\end{equation*}
将式\eqref{chapter3:电荷在给定电磁场中的运动方程-Lorentz力}中的$\dfrac{\mathd \mbf{p}}{\mathd t}$代入,可得
\begin{equation}
	\frac{\mathd \E_{\text{m}}}{\mathd t} = e\mbf{E}\cdot \mbf{v}
	\label{chapter3:电磁场中电荷能量的变化}
\end{equation}

电荷的机械能随时间的变化率就是场对粒子做功的功率。对电荷做功的仅仅是电场,磁场不能对在其中运动的粒子做功。

\section{规范不变性}

现在来研究场的势可唯一地确定到什么程度。首先,需要强调的是,场是由它对其内电荷的运动所产生的影响来刻画的。但是在运动方程\eqref{chapter3:电荷在给定电磁场中的运动方程-Lorentz力},而只出现了场强$\mbf{E}$和$\mbf{H}$。所以两个场如果用两个矢量$\mbf{E}$和$\mbf{H}$来描述,在物理上也是完全等同的。

假如给定了四维势$A^i$,则根据式\eqref{chapter3:电场强度的定义}和\eqref{chapter3:磁场强度的定义},$\mbf{E}$和$\mbf{H}$就由它们完全唯一地确定了。但是同一个场可以对应于不同的势。对四维势做变换
\begin{equation}
	A'_i = A_i-\frac{\pl f}{\pl x^i}
	\label{chapter3:四维势的一个变换}
\end{equation}
其中$f$是四维坐标的任意函数。经过这样的改变,在作用量积分\eqref{chapter3:电磁场中电荷的作用量1}中将出现附加项
\begin{equation}
	\dfrac{e}{c}\frac{\pl f}{\pl x^i}\mathd x^i = \mathd \left(\dfrac{e}{c}f\right)
\end{equation}
然而将一个全微分加在作用量积分的被积函数中,运动方程不会受到影响。

这个变换反映在矢势和标势上可以写作
\begin{equation}
	\mbf{A}'=\mbf{A} + \bnb f,\quad \phi' = \phi-\frac1c\frac{\pl f}{\pl t}
\end{equation}
很容易验证,在此变换下,由式\eqref{chapter3:电场强度的定义}和\eqref{chapter3:磁场强度的定义}定义的电场强度和磁场强度并不发生改变。因此,势的变换\eqref{chapter3:四维势的一个变换}并不改变场,所以势没有被唯一地确定,确定矢势仅仅精确到一个任意函数的梯度,而确定标势则仅仅精确到同一个任意函数的时间导数。

只有那些对于四维势变换\eqref{chapter3:四维势的一个变换}为不变的量才有物理意义,特别地,所有方程在这个变换下必须是不变的。这种不变性称为{\bf 规范不变性(Gauge invariance)}。

势缺乏唯一性,使得我们有可能去选择它们,使他们满足我们所选择的附加条件。由于势仅能精确到相差一个任意函数的四维梯度,因此我们能够令四维势的各个分量之间满足一个额外条件,以确定变换\eqref{chapter3:四维势的一个变换}中的任意函数$f$。特别而言,我们总是能够选择势,使得标势$\phi$为零。

\section{恒定电磁场}

{\bf 恒定}电磁场指与时间无关的电磁场。显然,恒定电磁场的势可以选择为与时间无关而仅与坐标有关的函数,恒定电场和恒定磁场即为
\begin{equation*}
	\mbf{E} = -\bnb \phi,\quad \mbf{H} = \bnb \times \mbf{A}
\end{equation*}

由于势没有被唯一确定,我们可以在标势上加一个任意常数而不改变场。通常要给$\phi$加上一个附加条件,即在空间内某一特定点有一个特定的值。通常,我们规定$\phi$在无穷远处的值为零。这样恒定场的标势就被唯一地确定了。

同样地,矢势也没有被唯一确定,我们可以把任意一个坐标的函数的梯度加到矢势上。

对于恒定电磁场,电荷的Lagrange函数中不显含时间。在这种情况下,能量是守恒的,即为其Hamilton量。根据式\eqref{chapter3:电磁场中电荷的Hamilton量-用速度表示的Hamilton量},可有
\begin{equation}
	\E = \frac{mc^2}{\sqrt{1-\dfrac{v^2}{c^2}}}+e\phi
	\label{chapter3:恒定电磁场中电荷的总能量}
\end{equation}
由于场的存在,粒子的能量加上了一项$e\phi$,它就是电荷在电磁场内的势能。此处可以注意到,总能量仅与标势有关,而与矢势无关。这说明磁场不影响电荷的能量,只有电场才能改变电荷的能量。

假如场强在空间所有点上都一样,这样的场称为{\bf 均匀场}。场强为$\mbf{E}$的均匀电场,其标势应为
\begin{equation}
	\phi = -\mbf{E}\cdot \mbf{r}
	\label{chapter3:恒定均匀电场的标势}
\end{equation}
均匀磁场的矢势可以用场强$\mbf{H}$表示为
\begin{equation}
	\mbf{H} = \frac12 \mbf{H}\times \mbf{r}
	\label{chapter3:均匀磁场的矢势1}
\end{equation}
如果选取$z$轴沿着$\mbf{H}$的方向,均匀磁场的矢势也可以选择为以下形式:
\begin{equation}
	A_x = -Hy,\quad A_y = A_z = 0
	\label{chapter3:均匀磁场的矢势2}
\end{equation}
很容易发现,势\eqref{chapter3:均匀磁场的矢势1}和\eqref{chapter3:均匀磁场的矢势2}彼此之间的差为函数$f = -\dfrac12 xyH$的梯度。

\section{在恒定均匀电场中的运动}\label{chapter3:section:在恒定均匀电场中的运动}

现在研究电荷量为$e$的电荷在均匀的恒定电场$\mbf{E}$中的运动。取电场方向为$x$轴,运动显然在一个平面内进行,将其取为$xy$平面。这时,运动方程\eqref{chapter3:电荷在给定电磁场中的运动方程-Lorentz力}变为
\begin{equation*}
\begin{cases}
	\dot{p}_x = eE \\
	\dot{p}_y = 0
\end{cases}
\end{equation*}
所以可得
\begin{equation}
	p_x = eEt,\quad p_y = p_0
	\label{chapter3:恒定电场中电荷的动量}
\end{equation}
将时间参考点选择在$p_x=0$的时刻,$p_0$表示粒子在该时刻的动量。据此,可求得粒子的机械能为
\begin{equation}
	\E_m = \sqrt{m^2c^4+p^2} = \sqrt{m^2c^4+c^2p_0^2+(ceEt)^2} = \sqrt{\E_0^2+(ceEt)^2}
	\label{chapter3:恒定电场中电荷的机械能}
\end{equation}
其中$\E_0$是$t=0$时的机械能。根据式\eqref{chapter2:自由实物粒子能量、动量和速度之间的关系}可得
\begin{equation*}
	\frac{\mathd x}{\mathd t} = \frac{p_xc^2}{\E_m} = \frac{c^2eEt}{\sqrt{\E_0^2+(ceEt)^2}}
\end{equation*}
积分可得
\begin{equation}
	x = \frac{1}{eE}\sqrt{\E_0^2+(ceEt)^2}
	\label{chapter3:均匀恒定电场中电荷的x坐标随时间关系}
\end{equation}
其中我们已经令积分常数等于零。为了求得$y$,考虑
\begin{equation*}
	\frac{\mathd y}{\mathd t} = \frac{p_yc^2}{\E_m} = \frac{p_0c^2}{\sqrt{\E_0^2+(ceEt)^2}}
\end{equation*}
积分可得
\begin{equation}
	y = \frac{p_0c}{eE}\arsinh \frac{ceEt}{\E_0}
	\label{chapter3:均匀恒定电场中电荷的y坐标随时间关系}
\end{equation}

联立式\eqref{chapter3:均匀恒定电场中电荷的x坐标随时间关系}和\eqref{chapter3:均匀恒定电场中电荷的y坐标随时间关系},即可得到轨道方程为
\begin{equation}
	x = \frac{\E_0}{eE}\cosh \frac{eEy}{p_0c}
	\label{chapter3:均匀恒定电场中电荷的轨迹方程}
\end{equation}
由此可见,均匀电场中的一个电荷沿着悬链线运动。

在低速近似下,可令$p_0=mv_0, \E_0 = mc^2$,并将式\eqref{chapter3:均匀恒定电场中电荷的轨迹方程}展开为$\dfrac1c$的幂级数并略去高阶项,可得
\begin{equation*}
	x = \frac{eE}{2mv_0^2}y^2+\text{const}
\end{equation*}
即电荷在低速近似下沿抛物线运动,与经典力学中的结果一致。

\section{在恒定均匀磁场中的运动}

本节研究电荷量为$e$的电荷在均匀的恒定磁场$\mbf{H}$中的运动。取磁场的方向为$z$轴方向,则有
\begin{equation*}
	\frac{\mathd \mbf{p}}{\mathd t} = \dfrac{e}{c}\mbf{v} \times \mbf{H}
\end{equation*}
将速度和动量的关系$\mbf{v} = \dfrac{c^2\mbf{p}}{\E}$代入,由于磁场不改变粒子的能量,可得
\begin{equation}
	\frac{\mathd \mbf{v}}{\mathd t} = \frac{ce}{\E}\mbf{v}\times \mbf{H}
	\label{chapter3:电荷在恒定均匀磁场中的运动方程}
\end{equation}
记$\omega = \dfrac{ceH}{\E}$,将方程\eqref{chapter3:电荷在恒定均匀磁场中的运动方程}写成分量形式,则有
\begin{subnumcases}{}
	\ds \dot{v}_x = \omega v_y \label{chapter3:电荷在恒定均匀磁场中的分量方程1} \\
	\ds \dot{v}_y = -\omega v_x \label{chapter3:电荷在恒定均匀磁场中的分量方程2} \\
	\ds \dot{v}_z = 0 \label{chapter3:电荷在恒定均匀磁场中的分量方程3}
\end{subnumcases}
由式\eqref{chapter3:电荷在恒定均匀磁场中的分量方程3}可以解得
\begin{equation}
	v_z = v_{0z}\quad \text{(常数)}
	\label{chapter3:电荷在恒定均匀磁场中的z速度}
\end{equation}
再由式\eqref{chapter3:电荷在恒定均匀磁场中的分量方程1}和式\eqref{chapter3:电荷在恒定均匀磁场中的分量方程2}消去$v_y$可得方程
\begin{equation*}
	\ddot{v}_x + \omega^2 v_x = 0
\end{equation*}
这个方程的解为
\begin{equation}
	v_x = v_{0t}\cos(\omega t+\alpha)
	\label{chapter3:电荷在恒定均匀磁场中的x速度}
\end{equation}
同时可得
\begin{equation}
	v_y = -v_{0t}\sin(\omega+\alpha)
	\label{chapter3:电荷在恒定均匀磁场中的y速度}
\end{equation}
常数$v_{0t}$和初始相位$\alpha$都是由初始条件来决定的。由于
\begin{equation*}
	v_{0t} = \sqrt{v_x^2+v_y^2}
\end{equation*}
即,$v_{0t}$是粒子在$xy$平面内速度的分量大小。由于运动过程中粒子的能量是不变的,故这个分量也是不变的。

再次积分式\eqref{chapter3:电荷在恒定均匀磁场中的x速度}、式\eqref{chapter3:电荷在恒定均匀磁场中的y速度}和式\eqref{chapter3:电荷在恒定均匀磁场中的z速度},可得
\begin{equation}
\begin{cases}
	x = x_0 + r \sin(\omega t+\alpha) \\
	y = y_0 + r \cos(\omega t+\alpha) \\
	z = z_0 + v_{0z}t
\end{cases}
\label{chapter3:电荷在恒定均匀磁场中运动的轨迹方程}
\end{equation}
其中
\begin{equation}
	r = \dfrac{v_{0t}}{\omega} = \dfrac{v_{0t}\E}{ceH} = \frac{cp_t}{eH}
	\label{chapter3:电荷在恒定均匀磁场中运动的轨道半径}
\end{equation}
式\eqref{chapter3:电荷在恒定均匀磁场中运动的轨迹方程}表明,电荷在恒定均匀磁场中沿螺旋线运动,螺旋线的轴沿磁场方向,螺旋线的半径由式\eqref{chapter3:电荷在恒定均匀磁场中运动的轨道半径}决定,在运动过程中,粒子的速度保持不变,其旋转的角速度为$\omega$。

当$v_{0z}=0$,即粒子沿着磁场方向的速度分量为零的特殊情况下,粒子将在与磁场垂直的平面内作匀速圆周运动。此时$\omega$即为粒子圆周运动的角速度。

\begin{example}
将一个带电的空间谐振子置于均匀磁场内,这个振子的本征频率为$\omega_0$。求它在磁场中的振动频率。
\end{example}
\begin{solution}
设磁场方向为$z$方向,因此振子的受迫振动方程为\footnote{此处应用的是经典结论。}
\begin{subnumcases}{}
	\ds \ddot{x} + \omega_0^2 x = \frac{eH}{mc}\dot{y} \label{chapter3:恒定均匀磁场中的带电谐振子方程1} \\
	\ds \ddot{y} + \omega_0^2 y = -\frac{eH}{mc}\dot{x} \label{chapter3:恒定均匀磁场中的带电谐振子方程2} \\
	\ds \ddot{z} + \omega_0^2 z = 0 \label{chapter3:恒定均匀磁场中的带电谐振子方程3}
\end{subnumcases}
将式\eqref{chapter3:恒定均匀磁场中的带电谐振子方程2}乘以虚数单位$\mathi$并与式\eqref{chapter3:恒定均匀磁场中的带电谐振子方程1}相加,令$\xi = x+\i y$,则可得
\begin{equation*}
	\ddot{\xi} + \mathi \frac{eH}{mc} \dot{\xi} + \omega_0^2\xi = 0
\end{equation*}
这个微分方程的通解为
\begin{equation*}
	\xi = A\mathe^{\mathi \left(-\frac{eH}{2mc}+\sqrt{\omega_0^2+\frac14\left(\frac{eH}{mc}\right)^2}\right) t} + B\mathe^{\mathi \left(-\frac{eH}{2mc}-\sqrt{\omega_0^2+\frac14\left(\frac{eH}{mc}\right)^2}\right) t}
\end{equation*}
其中$A, B$是任意的复常数。于是有\small
\begin{align*}
	\ds x & = \Re \xi \\
	& = A_0\cos\left[\left(\sqrt{\omega_0^2 + \frac14\left(\frac{eH}{mc}\right)^2} - \frac{eH}{2mc}\right) t + \phi_1\right] + B_0\cos\left[\left(\sqrt{\omega_0^2 + \frac14\left(\frac{eH}{mc}\right)^2} + \frac{eH}{2mc}\right) t + \phi_2\right] \\
	\ds y & = \Im \xi \\
	& = A_0\sin\left[\left(\sqrt{\omega_0^2 + \frac14\left(\frac{eH}{mc}\right)^2} - \frac{eH}{2mc}\right) t + \phi_1\right] + B_0\sin\left[\left(\sqrt{\omega_0^2 + \frac14\left(\frac{eH}{mc}\right)^2} + \frac{eH}{2mc}\right) t + \phi_2\right]
\end{align*}
\normalsize 由此可得振子在与磁场垂直的平面内的振动频率即为
\begin{equation*}
	\omega = \sqrt{\omega_0^2 + \frac14\left(\frac{eH}{mc}\right)^2} \pm \frac{eH}{2m}
\end{equation*}
磁场很弱的时候,这个结果变为
\begin{equation*}
	\omega \approx \omega_0 \pm \frac{eH}{2mc}
\end{equation*}
\end{solution}

\section{电荷在均匀恒定的电场和磁场中的运动}

本节研究在电场和磁场都存在并且是均匀恒定的情况下,一个电荷如何运动。这些讨论仅局限于粒子的速度远小于光速,即$v\ll c$的情形,因此质点的动量为$\mbf{p}=m\mbf{v}$。

选择磁场$\mbf{H}$的方向为$z$轴方向,矢量$\mbf{H}$和$\mbf{E}$决定的平面为$yz$平面。这时,电荷的运动方程为
\begin{equation}
	m\dot{\mbf{v}} = e\mbf{E} + \frac{e}{c}\mbf{v}\times \mbf{H}
	\label{chapter3:均匀恒定电场和磁场中电荷的运动方程}
\end{equation}
其分量形式可以写作
\begin{subnumcases}{}
	m\ddot{x} = \dfrac{e}{c}H\dot{y} \label{chapter3:均匀恒定电场和磁场中电荷的运动方程分量形式1} \\
	m\ddot{y} = eE_y - \dfrac{e}{c}H\dot{x} \label{chapter3:均匀恒定电场和磁场中电荷的运动方程分量形式2} \\
	m\ddot{z} = eE_z \label{chapter3:均匀恒定电场和磁场中电荷的运动方程分量形式3}
\end{subnumcases}
由式\eqref{chapter3:均匀恒定电场和磁场中电荷的运动方程分量形式3}可得,电荷以恒定加速度沿着$z$轴方向运动,即
\begin{equation}
	z = \frac{eE_z}{2m}t^2+v_{0z}t
\end{equation}
此处已经令积分常数为零。由此,后面可以不考虑这个方向的运动,即考虑电荷在正交的电场和磁场中的运动。

用$\mathi$乘以式\eqref{chapter3:均匀恒定电场和磁场中电荷的运动方程分量形式2}然后与式\eqref{chapter3:均匀恒定电场和磁场中电荷的运动方程分量形式1}相加,并令$\xi = x+\mathi y$可得
\begin{equation}
	\ddot{\xi} + \mathi \omega\dot{\xi} = \mathi \frac{eE_y}{m}
\end{equation}
式中$\omega = \dfrac{eH}{mc}$,其通解为
\begin{equation}
	\dot{\xi} = a\mathe^{-\mathi\omega t}+\frac{cE_y}{H}
\end{equation}
其中$a$是任意复常数。将$a$写成$a=b\mathe^{\mathi \alpha}$的形式,其中$b$和$\alpha$为实数,这说明,适当选择时间原点可以使得复常数$a$的虚部为零。我们这样选取时间原点,使得$a$为实数,由此即有
\begin{equation}
\begin{cases}
	\dot{x} = a\cos\omega t+\dfrac{cE_y}{H} \\
	\dot{y} = -a\sin\omega t
\end{cases}
\label{chapter3:均匀恒定电场和磁场中电荷的速度}
\end{equation}
当$t=0$时,速度沿着$x$轴方向。

由此可以看出,粒子的速度是时间的周期函数,它们的平均值为
\begin{equation*}
	\bar{\dot{x}} = \frac{cE_y}{H},\quad \bar{\dot{y}} = 0
\end{equation*}
电荷在正交的电场和磁场中运动的这个平均速度称为{\bf 电漂移速度},它的方向与两个场都垂直且与电荷的正负无关,它可以改写为矢量形式:
\begin{equation}
	\bar{\mbf{v}} = \frac{c\mbf{E}\times \mbf{H}}{H^2}
	\label{chapter3:电漂移速度}
\end{equation}

这一节的所有公式都假设了粒子的速度远小于光速,即电场和磁场必须满足
\begin{equation}
	\frac{E_y}{H} \ll 1
\end{equation}
即电场强度远小于磁场强度,而$E_y$和$H$的绝对大小是任意的。

将方程\eqref{chapter3:均匀恒定电场和磁场中电荷的速度}再积分一次,并使得当$t=0$时,$x=y=0$,则有
\begin{equation}
\begin{cases}
	\ds x = \frac{a}{\omega}\sin\omega t+\frac{cE_y}{H}t \\
	\ds y = \frac{a}{\omega}(\cos \omega t - 1)
\end{cases}
\label{chapter3:均匀恒定电场和磁场中电荷的运动轨迹}
\end{equation}

将式\eqref{chapter3:均匀恒定电场和磁场中电荷的运动轨迹}看作一条曲线的参数方程,则它们定义了一条次摆线,如图\ref{chapter3:外摆线的情形}、\ref{chapter3:内摆线的情形}和\ref{chapter3:摆线的情形}所示,其具体的形状取决于$|a|$和$\dfrac{cE_y}{H}$之间的大小关系。

\begin{figure}[htb]
\centering
\begin{minipage}[t]{0.48\textwidth}
\begin{asy}
	texpreamble("\usepackage{xeCJK}");
	texpreamble("\setCJKmainfont{SimSun}");
	usepackage("amsmath");
	import graph;
	size(180);
	//
	real x1,x2,y;
	real a,b,omega;
	pair f(real t){
		return (a/omega*sin(omega*t)+b*t,a/omega*(cos(omega*t)-1));
	}
	x1 = -1;
	x2 = 6.5;
	y = 2.5;
	draw(Label("$x$",EndPoint),(x1,0)--(x2,0),Arrow);
	draw(Label("$y$",EndPoint),(0,0)--(0,y),Arrow);
	a = -1;
	b = 0.7;
	omega = 1;
	path p;
	p = graph(f,-2.5,2*pi+2.5);
	draw(p,linewidth(0.8bp));
	add(arrow(p,invisible,FillDraw(black),Relative(0.1)));
	add(arrow(p,invisible,FillDraw(black),Relative(0.93)));
	add(arrow(p,invisible,FillDraw(black),Relative(0.53)));
\end{asy}
\caption{外摆线:$|a|>\dfrac{cE_y}{H}$的情形}
\label{chapter3:外摆线的情形}
\end{minipage}
\hspace{0.1cm}
\begin{minipage}[t]{0.48\textwidth}
\begin{asy}
	texpreamble("\usepackage{xeCJK}");
	texpreamble("\setCJKmainfont{SimSun}");
	usepackage("amsmath");
	import graph;
	size(180);
	//
	real x1,x2,y;
	real a,b,omega;
	pair f(real t){
		return (a/omega*sin(omega*t)+b*t,a/omega*(cos(omega*t)-1));
	}
	x1 = -1;
	x2 = 6.5;
	y = 2.5;
	draw(Label("$x$",EndPoint),(x1,0)--(x2,0),Arrow);
	draw(Label("$y$",EndPoint),(0,0)--(0,y),Arrow);
	a = -0.5;
	b = 0.8;
	omega = 1;
	path p;
	p = graph(f,-2,2*pi+2);
	draw(p,linewidth(0.8bp));
	add(arrow(p,invisible,FillDraw(black),Relative(0.1)));
	add(arrow(p,invisible,FillDraw(black),Relative(0.93)));
	add(arrow(p,invisible,FillDraw(black),Relative(0.53)));
\end{asy}
\caption{内摆线:$|a|<\dfrac{cE_y}{H}$的情形}
\label{chapter3:内摆线的情形}
\end{minipage}\\

\begin{minipage}[t]{0.48\textwidth}
\begin{asy}
	texpreamble("\usepackage{xeCJK}");
	texpreamble("\setCJKmainfont{SimSun}");
	usepackage("amsmath");
	import graph;
	size(180);
	//
	real x1,x2,y;
	real a,b,omega;
	pair f(real t){
		return (a/omega*sin(omega*t)+b*t,a/omega*(cos(omega*t)-1));
	}
	x1 = -1;
	x2 = 6.5;
	y = 2.5;
	draw(Label("$x$",EndPoint),(x1,0)--(x2,0),Arrow);
	draw(Label("$y$",EndPoint),(0,0)--(0,y),Arrow);
	a = -0.8;
	b = 0.8;
	omega = 1;
	path p;
	p = graph(f,-2,2*pi+2);
	draw(p,linewidth(0.8bp));
	add(arrow(p,invisible,FillDraw(black),Relative(0.1)));
	add(arrow(p,invisible,FillDraw(black),Relative(0.93)));
	add(arrow(p,invisible,FillDraw(black),Relative(0.53)));
\end{asy}
\caption{摆线:$|a|=\dfrac{cE_y}{H}$的情形}
\label{chapter3:摆线的情形}
\end{minipage}
\end{figure}

如果$a=-\dfrac{cE_y}{H}$,那么式\eqref{chapter3:均匀恒定电场和磁场中电荷的运动轨迹}变为
\begin{equation}
\begin{cases}
	x = \dfrac{cE_y}{\omega H}(\omega t-\sin \omega t) \\[1.5ex]
	y = \dfrac{cE_y}{\omega H}(1-\cos\omega t)
\end{cases}
\end{equation}
此时,轨道在$xy$平面上的投影是一条摆线,如图\ref{chapter3:摆线的情形}所示。

\begin{example}
求电荷在平行均匀电场和磁场中的相对论运动。
\end{example}
\begin{solution}
取磁场$\mbf{H}$的方向为$z$轴方向,由于磁场对$z$轴方向上的运动没有影响,因而$z$方向上的运动只在电场的作用下发生,于是根据第\ref{chapter3:section:在恒定均匀电场中的运动}节的结果,可得
\begin{equation}
	z = \frac{\E_m}{eE} = \frac{1}{eE}\sqrt{\E_0^2+(ceEt)^2}
	\label{chapter3:例3.2在z方向上的运动}
\end{equation}

对于在$xy$平面内的运动,运动方程为
\begin{equation*}
\begin{cases}
	\dot{p}_x = \dfrac{e}{c}Hv_y \\[1.5ex]
	\dot{p}_y = -\dfrac{e}{c}Hv_x
\end{cases}
\end{equation*}
将其转化为复变数的方程可得
\begin{equation}
	\frac{\mathd}{\mathd t}(p_x+\mathi p_y) = -\mathi \dfrac{eH}{c}(v_x+\mathi v_y)
	\label{chapter3:例3.2动量满足的方程}
\end{equation}
由式\eqref{chapter2:自由实物粒子能量、动量和速度之间的关系}可得
\begin{equation*}
	v_x = \frac{c^2p_x}{\E_m},\quad v_y = \frac{c^2p_y}{\E_m}
\end{equation*}
其中$\E_m = \sqrt{\E_0^2+(ceEt)^2}$。如果记$p=p_x+\mathi p_y$,则方程\eqref{chapter3:例3.2动量满足的方程}化为
\begin{equation}
	\frac{\mathd p}{\mathd t} = -\mathi \frac{ceH}{\E_m}p = -\mathi \frac{ceH}{\sqrt{\E_0^2+(ceEt)^2}}p
	\label{chapter3:例3.2复动量满足的方程}
\end{equation}
将方程\eqref{chapter3:例3.2复动量满足的方程}分离变量并积分,即有
\begin{equation*}
	\int_{p_t}^p \frac{\mathd p}{p} = -\int_0^t \mathi \frac{ceH}{\sqrt{\E_0^2+(ceEt)^2}} \mathd t
\end{equation*}
其中$p_t$为动量在$xy$平面上投影的恒定值。由此则有
\begin{equation*}
	p = p_t\mathe^{-\mathi \frac{H}{E}\arsinh \frac{ceEt}{\E_0}}
\end{equation*}
方便起见,引入一个辅助量$\psi$,它满足
\begin{equation*}
	\psi = \frac{H}{E}\arsinh \frac{ceEt}{\E_0}
\end{equation*}
或者
\begin{equation}
	ct = \frac{\E_0}{eE}\sinh \frac{E}{H}\psi
	\label{chapter3:例3.2辅助量psi与时间的关系}
\end{equation}
它的微分满足关系
\begin{equation*}
	\mathd \psi = \frac{ceH}{\sqrt{\E_0^2+(ceEt)^2}} \mathd t
\end{equation*}
由此即有
\begin{equation*}
	p = p_x+\mathi p_y = p_t\mathe^{-\mathi \psi}
\end{equation*}
再利用式\eqref{chapter2:自由实物粒子能量、动量和速度之间的关系}可得
\begin{equation*}
	p_x+\mathi p_y = \frac{\E_m\dot{x}}{c^2} + \mathi \frac{\E_m\dot{y}}{c^2} = \frac{\E_m}{c^2} \left(\frac{\mathd x}{\mathd \psi} \frac{\mathd \psi}{\mathd t} + \mathi \frac{\mathd y}{\mathd \psi} \frac{\mathd \psi}{\mathd t}\right) = \dfrac{eH}{c}\left(\frac{\mathd x}{\mathd \psi} + \mathi \frac{\mathd y}{\mathd \psi}\right)
\end{equation*}
即有
\begin{equation*}
	\dfrac{eH}{c}\left(\frac{\mathd x}{\mathd \psi} + \mathi \frac{\mathd y}{\mathd \psi}\right) = p_t\mathe^{-\mathi \psi}
\end{equation*}
从中可以解得
\begin{equation}
	x = \frac{cp_t}{eH}\sin \psi,\quad y = \frac{cp_t}{eH}\cos\psi
	\label{chapter3:例3.2在xy平面内的运动}
\end{equation}
由式\eqref{chapter3:例3.2在z方向上的运动}和式\eqref{chapter3:例3.2辅助量psi与时间的关系}可得
\begin{equation}
	z = \frac{\E_0}{eE}\cosh\frac{E}{H}\psi
	\label{chapter3:例3.2在z方向上的运动-辅助量表示}
\end{equation}
式\eqref{chapter3:例3.2在xy平面内的运动}和式\eqref{chapter3:例3.2在z方向上的运动-辅助量表示}一起以参数形式决定了粒子的运动,其轨道为螺旋线,半径为$\dfrac{cp_t}{eH}$,步长单调增加,粒子沿着螺旋线以减小的角速度$\omega = \dfrac{ceH}{\E_m}$运动,其沿$z$轴的速度趋向于$c$。
\end{solution}

\begin{example}
求电荷在互相垂直且数值相等的电场和磁场中的相对论运动。\footnote{互相垂直而数值不等的$\mbf{E}$和$\mbf{H}$场中的运动问题,可以通过适当的参考系变换,化为纯电场或纯磁场中运动的问题,见第\ref{chapter3:section:场的不变量}节。}
\end{example}
\begin{solution}
取$\mbf{H}$的方向为$z$方向,$\mbf{E}$的方向为$y$方向,并考虑到$E=H$可有运动方程为
\begin{subnumcases}{}
	\dfrac{\mathd p_x}{\mathd t} = \frac{e}{c}Ev_y \label{chapter3:例3.3运动方程1} \\
	\dfrac{\mathd p_y}{\mathd t} = eE-\frac{e}{c}Ev_x \label{chapter3:例3.3运动方程2} \\
	\dfrac{\mathd p_z}{\mathd t} = 0 \label{chapter3:例3.3运动方程3} 
\end{subnumcases}
由方程\eqref{chapter3:例3.3运动方程3}可得
\begin{equation}
	p_z = p_{0z}\quad \text{(常数)}
	\label{chapter3:例3.3-z方向动量}
\end{equation}
再由式\eqref{chapter3:电磁场中电荷能量的变化}可得
\begin{equation*}
	\frac{\mathd \E_m}{\mathd t} = eEv_y
\end{equation*}
即有
\begin{equation}
	\E_m-cp_x = \alpha\quad \text{(常数)}
	\label{chapter3:例3.3守恒定律1}
\end{equation}
由式\eqref{chapter2:自由实物粒子能量与动量之间的关系}可得
\begin{equation*}
	\E_m^2 = c^2p_x^2+c^2p_y^2+c^2p_z^2+m^2c^4
\end{equation*}
令$\eps^2 = m^2c^4+c^2p_z^2$(常数),可有
\begin{equation*}
	\E_m^2-c^2p_x^2 = (\E_m+cp_x)(\E_m-cp_x) = c^2p_y^2+\eps^2
\end{equation*}
所以可得
\begin{equation}
	\E_m+cp_x = \frac{1}{\alpha}(c^2p_y^2+\eps^2)
	\label{chapter3:例3.3能量动量关系1}
\end{equation}
联立式\eqref{chapter3:例3.3守恒定律1}和式\eqref{chapter3:例3.3能量动量关系1}可得
\begin{subnumcases}{}
	\ds \E_m = \frac{\alpha}{2}+\frac{c^2p_y^2+\eps^2}{2\alpha} \label{chapter3:例3.3机械能和y方向动量的关系} \\
	\ds p_x = -\frac{\alpha}{2c}+\frac{c^2p_y^2+\eps^2}{2\alpha c} \label{chapter3:例3.3x方向动量和y方向动量的关系}
\end{subnumcases}
进一步可有
\begin{equation*}
	\E_m \frac{\mathd p_y}{\mathd t} = eE\left(\E_m - \frac{\E_mv_x}{c}\right) = eE(\E_m-cp_x) = eE\alpha
\end{equation*}
结合式\eqref{chapter3:例3.3机械能和y方向动量的关系}可得
\begin{equation}
	2eEt = \left(1+\frac{\eps^2}{\alpha^2}\right)p_y + \frac{c^2}{3\alpha^2} p_y^3
	\label{chapter3:例3.3时间与y方向动量的关系}
\end{equation}

下面,为了确定轨道方程,首先在方程
\begin{equation}
	\frac{\mathd x}{\mathd t} = \frac{c^2p_x}{\E_m},\quad \frac{\mathd y}{\mathd t} = \frac{c^2p_y}{\E_m},\quad \frac{\mathd z}{\mathd t} = \frac{c^2p_z}{\E_m}
\end{equation}
中,作变量代换
\begin{equation*}
	\mathd t = \frac{\E_m}{eE\alpha}\mathd p_y
\end{equation*}
可得以$p_y$为参数的轨道方程
\begin{equation}
\begin{cases}
	\ds x = \frac{c}{2eE}\left(-1+\frac{\eps^2}{\alpha^2}\right)p_y + \frac{c^3}{6\alpha^2eE}p_y^3 \\[1.5ex]
	\ds y = \frac{c^2}{2\alpha eE}p_y^2 \\[1.5ex]
	\ds z = \frac{p_zc^2}{eE\alpha}p_y
\end{cases}
\end{equation}
\end{solution}

\section{电磁场张量}

在第\ref{chapter3:section:场中电荷的运动方程}节中,我们从三维形式的Lagrange函数\eqref{chapter3:电磁场中电荷的Lagrange函数}导出了电荷在场内的运动方程。本节将直接从四维形式的作用量\eqref{chapter3:电磁场中电荷的作用量1}中导出同样的方程。

Hamilton原理给出:
\begin{equation}
	\delta S = \delta \int_a^b \left(-mc\mathd s-\frac{e}{c}A_i\mathd x^i\right)
	\label{chapter3:电磁场张量-电磁场的Hamilton原理1}
\end{equation}
注意到$\mathd s^2 = \mathd x_i\mathd x^i$,可求得
\begin{equation*}
	\delta S = -\int_a^b \left(mc\frac{\mathd x_i\mathd \delta x^i}{\mathd s} + \frac{e}{c}A_i\mathd \delta x^i + \frac{e}{c}\delta A_i\mathd x^i\right)
\end{equation*}
分部积分可得
\begin{align}
	\delta S & = -\int_a^b \left(mu_i\mathd \delta x^i + \frac{e}{c}A_i\mathd \delta x^i + \frac{e}{c}\delta A_i\mathd x^i\right) \nonumber \\
	& = \int_a^b \left(m\mathd u_i \delta x^i + \frac{e}{c}\frac{\pl A_i}{\pl x^k}\mathd x^k\delta x^i - \frac{e}{c} \frac{\pl A_i}{\pl x^k}\mathd x^i \delta x^k\right) - \left(mu_i+\frac{e}{c}A_i\right)\delta x^i\bigg|_a^b \nonumber \\
	& = \frac1c \int_a^b \left[m\frac{\mathd u_i}{\mathd \tau} - \frac{e}{c}\left(\frac{\pl A_k}{\pl x^i} - \frac{\pl A_i}{\pl x^k}\right) u^k\right] \delta x^i\mathd s - \left(mu_i+\frac{e}{c}A_i\right)\delta x^i\bigg|_a^b
	\label{chapter3:电磁场张量-电磁场的Hamilton原理2}
\end{align}

在固定边界条件下,式\eqref{chapter3:电磁场张量-电磁场的Hamilton原理2}的第二项为零,由于$\delta x^i$的任意性,可得式\eqref{chapter3:电磁场张量-电磁场的Hamilton原理2}的第一项中被积函数必须为零,即
\begin{equation*}
	m\frac{\mathd u_i}{\mathd \tau} - \frac{e}{c}\left(\frac{\pl A_k}{\pl x^i} - \frac{\pl A_i}{\pl x^k}\right) u^k = 0
\end{equation*}

现在我们引入下面的符号:
\begin{equation}
	F_{ik} = \frac{\pl A_k}{\pl x^i} - \frac{\pl A_i}{\pl x^k}
	\label{chapter3:电磁场张量的定义}
\end{equation}
其中反对称张量$F_{ik}$称为{\bf 电磁场张量},由此,运动方程具有下面的形式
\begin{equation}
	m\frac{\mathd u_i}{\mathd \tau} = \frac{e}{c}F_{ik}u^k
	\label{chapter3:电荷在电磁场中运动方程的四维形式}
\end{equation}

将$A_i=(\phi,-\mbf{A})$代入式\eqref{chapter3:电磁场张量的定义},可以得出张量$F_{ik}$各分量与场量之间的联系。其结果可以写成一个矩阵,其中指标$i=0,1,2,3$表示行,$k=0,1,2,3$表示列,即
\begin{equation}
	F_{ik} = \begin{pmatrix} 0 & E_x & E_y & E_z \\
							-E_x & 0 & -H_z & H_y \\
							-E_y & H_z & 0 & -H_x \\
							-E_z & -H_y & H_x & 0 \end{pmatrix},\quad
	F^{ik} = \begin{pmatrix} 0 & -E_x & -E_y & -E_z \\
							E_x & 0 & -H_z & H_y \\
							E_y & H_z & 0 & -H_x \\
							E_z & -H_y & H_x & 0 \end{pmatrix}
	\label{chapter3:电磁场张量的分量形式}
\end{equation}
这个结果也可以简单地写作
\begin{equation}
	F_{ik} = (\mbf{E},\mbf{H}),\quad F^{ik} = (-\mbf{E},\mbf{H})
\end{equation}
由此可见,电场强度与磁场强度的分量是同一四维电磁场张量的分量。

容易验证,运动方程\eqref{chapter3:电荷在电磁场中运动方程的四维形式}的三个空间分量($i=1,2,3$)与矢量运动方程\eqref{chapter3:电荷在给定电磁场中的运动方程-Lorentz力}相同,而时间分量($i=0$)则给出功方程\eqref{chapter3:电磁场中电荷能量的变化}。后者是运动方程的推论。

假如在变分$\delta S$中,只考虑真实的轨道,那么式\eqref{chapter3:电磁场张量-电磁场的Hamilton原理2}的第一项就恒等于零。此时积分上限是变化的,由此可得作用量作为坐标函数的微分为
\begin{equation}
	\delta S = -\left(mu_i+\frac{e}{c}A_i\right)\delta x^i
	\label{chapter3:作用量对坐标的微分}
\end{equation}
由此可得
\begin{equation}
	-\frac{\pl S}{\pl x^i} = mu_i + \frac{e}{c}A_i = p_i+\frac{e}{c}A_i
	\label{chapter3:电磁场中电荷的四维动量}
\end{equation}
四维矢量$-\dfrac{\pl S}{\pl x^i}$即为粒子的四维动量矢量$P_i$。将四维机械动量$p_i$和四维势$A_i$代入,可得
\begin{equation}
	P^i = \left(\frac{\E_m+e\phi}{c},\mbf{p} +\frac{e}{c}\mbf{A}\right)
	\label{chapter3:电磁场中电荷的四维动量矢量}
\end{equation}
这个四维矢量$P^i$的三个空间分量构成三维广义动量矢量\eqref{chapter3:电磁场中电荷的广义动量},其时间分量则为$\dfrac{\E}{c}$,其中$\E$是电荷在场中的总能量。

\section{场的Lorentz变换}

四维势的变换公式可以根据式\eqref{chapter1:四维矢量协变分量的变换关系式}得出,结合$A^i = (\phi,\mbf{A})$可得
\begin{equation}
	\phi = \frac{\phi'+\dfrac{V}{c}A'_x}{\sqrt{1-\dfrac{V^2}{c^2}}},\quad
	A_x = \frac{A'_x+\dfrac{V}{c}\phi'}{\sqrt{1-\dfrac{V^2}{c^2}}},\quad
	A_y = A'_y,\quad A_z = A'_z
\label{chapter3:四维势的变换}
\end{equation}

电磁场张量的变换公式可以根据式\eqref{chapter1:四维张量的变换规则}得出,由此可以得出电场的变换公式:
\begin{equation}
	E_x = E'_x,\quad E_y = \frac{E'_y+\dfrac{V}{c}H'_z}{\sqrt{1-\dfrac{V^2}{c^2}}},\quad E_z = \frac{E'_z-\dfrac{V}{c}H'_y}{\sqrt{1-\dfrac{V^2}{c^2}}}
	\label{chapter3:电场的变换}
\end{equation}
以及磁场的变换公式
\begin{equation}
	H_x = H'_x,\quad H_y = \frac{H'_y-\dfrac{V}{c}E'_z}{\sqrt{1-\dfrac{V^2}{c^2}}},\quad H_z = \frac{H'_z+\dfrac{V}{c}E'_y}{\sqrt{1-\dfrac{V^2}{c^2}}}
	\label{chapter3:磁场的变换}
\end{equation}

因此,电场和磁场与很多其他物理量一样,是相对的。它们在不同的参考系中有不同的特性。特别地讲,电场或磁场可以在某一个参考系中为零,而在另外的参考系中又存在。

变换式\eqref{chapter3:电场的变换}和\eqref{chapter3:磁场的变换}在$V\ll c$的情形下将大大简化,精确至$\dfrac{V}{c}$数量级,可得
\begin{align*}
	& E_x = E'_x,\quad E_y = E'_y+\frac{V}{c}H'_z,\quad E_z = E'_z-\frac{V}{c}H'_y \\
	& H_x = H'_x,\quad H_y = H'_y-\frac{V}{c}E'_z,\quad H_z = H'_z+\frac{V}{c}E'_y
\end{align*}
这些变换式可以用矢量形式表示为:
\begin{equation}
	\mbf{E} = \mbf{E}'+\frac1c \mbf{H}'\times \mbf{V},\quad \mbf{H} = \mbf{H}'-\frac1c \mbf{E}'\times \mbf{V}
	\label{chapter3:低速近似下的电磁场变换公式}
\end{equation}

以上公式相应的逆变换公式可以通过直接改变$V$的符号来获得。

如果在$K'$系中磁场$\mbf{H}'=\mbf{0}$,那么,根据式\eqref{chapter3:电场的变换}和\eqref{chapter3:磁场的变换}可以验证,在$K$系中,电场和磁场之间存在如下关系:
\begin{equation}
	\mbf{H} = \frac1c\mbf{V}\times \mbf{E}
\end{equation}
假如在$K'$系中,$\mbf{E}'=\mbf{0}$,那么在$K$系中,电场和磁场之间存在如下关系:
\begin{equation}
	\mbf{E} = -\frac1c\mbf{V}\times \mbf{H}
\end{equation}
在上面的两种情况下,电场与磁场在$K$系中都是互相垂直的。

这个结论是很有意义的,如果场$\mbf{E}$和$\mbf{H}$在某个参考系$K$中互相垂直(但数值不等),那么必然存在一个参考系$K'$,其中的场是纯电场或纯磁场。这个参考系相对于$K$的速度$\mbf{V}$垂直于$\mbf{E}$和$\mbf{H}$,当$H<E$时,其大小为$\dfrac{cH}{E}$;当$E<H$时,其大小为$\dfrac{cE}{H}$。

\section{场的不变量}\label{chapter3:section:场的不变量}

从电磁场张量可以构造出一些不变量,这些不变量在从一个惯性系变换到另一个惯性系时保持不变。

从场的四维表示出发,用反对称四维张量$F^{ik}$可以得出这些不变量的形式。由这个张量的分量可以构造如下不变量:
\begin{align}
	F_{ik}F^{ik} & = \text{不变量} \label{chapter3:场的不变量1-四维形式} \\
	e^{iklm}F_{ik}F_{lm} & = \text{不变量} \label{chapter3:场的不变量2-四维形式}
\end{align}
不变量\eqref{chapter3:场的不变量1-四维形式}是一个标量,而不变量\eqref{chapter3:场的不变量2-四维形式}则是赝标量(张量$F^{ik}$同其对偶张量的乘积\footnote{赝标量\eqref{chapter3:场的不变量2-四维形式}也可以表示为一个四维散度
\begin{equation*}
	e^{iklm}F_{ik}F_{lm} = 4\frac{\pl}{\pl x^i}\left(e^{iklm}A_k\frac{\pl A_m}{\pl x^l}\right)
\end{equation*}
})。

利用式\eqref{chapter3:电磁场张量的分量形式}可以将不变量用场变量$\mbf{E}$和$\mbf{H}$表示出来,在三维形式下可有
\begin{align}
	H^2-E^2 & = \text{不变量} \label{chapter3:场的不变量1} \\
	\mbf{E} \cdot \mbf{H} & = \text{不变量} \label{chapter3:场的不变量2}
\end{align}
其中第二个不变量是极矢量$\mbf{E}$和轴矢量$\mbf{H}$的乘积,因而是一个赝标量。

由不变量\eqref{chapter3:场的不变量2}可知,如果电场和磁场在某一惯性系中互相垂直,即$\mbf{E}\cdot\mbf{H}=0$,那么它们在其他所有惯性系中也垂直。由不变量\eqref{chapter3:场的不变量1}可知,如果电场和磁场在某一惯性系中相等,即$H=E$,那么它们在任何其他所有惯性系中也相等。

如果在某一参考系中$E>H$(或$H>E$),那么在其它所有惯性系中也有$E>H$(或$H>E$)。如果在某个参考系中矢量$\mbf{E}$和$\mbf{H}$成锐角(或钝角),那么它们在每个其它参考系也成锐角(或钝角)。

借助Lorentz变换,我们总可以给予$\mbf{E}$和$\mbf{H}$任何给定的值,唯一的附加条件是$E^2-H^2$和$\mbf{E}\cdot\mbf{H}$具有固定值。特别地,总可以找到一个惯性系,其中电场和磁场在给定点彼此平行。在这个参考系中$\mbf{E}\cdot\mbf{H}=EH$,并且由
\begin{equation*}
	E^2-H^2=E_0^2-H_0^2,\quad EH = \mbf{E}_0\cdot\mbf{H}_0
\end{equation*}
可以求出$\mbf{E}$和$\mbf{H}$在这个参考系中的值($\mbf{E}_0$和$\mbf{H}_0$是在原来参考系中的电场和磁场)。

在两个不变量都为零的情形下,$\mbf{E}$和$\mbf{H}$在所有的参考系中都相等并且互相垂直。

如果$\mbf{E}\cdot\mbf{H}=0$,我们总可以找到一个参考系,其中$\mbf{E}=\mbf{0}$或者$\mbf{H}=\mbf{0}$(依$E^2-H^2$的正负而定),即为纯电场或纯磁场。反之,如果在某一坐标系中$\mbf{E}=\mbf{0}$或者$\mbf{H}=\mbf{0}$,则它们在每个其它参考系中都互相垂直。